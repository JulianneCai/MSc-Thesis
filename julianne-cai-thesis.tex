\documentclass[a4paper]{report}
\usepackage[usenames,dvipsnames]{xcolor}
\usepackage{hyperref, lineno} % TODO REMOVE
\usepackage{mystyle}
%
% \graphicspath{{images/}}
% \SetWatermarkAngle{90} % TODO REMOVE
% \SetWatermarkScale{0.3} % TODO REMOVE
% \SetWatermarkHorCenter{1cm} % TODO REMOVE
\setcounter{tocdepth}{3} % SUBSECTIONS IN TOC
%
\author{
  Julianne Cai, 1413991\\
} 
\title{}
\date{\today}

\usepackage[myheadings]{fullpage}
\usepackage{fancyhdr}
\usepackage{lastpage}
\usepackage{wrapfig, subcaption, setspace, booktabs}
\usepackage[protrusion=true, expansion=true]{microtype}
\usepackage[english]{babel}
\usepackage{sectsty}
\usepackage{titling}

\newtheorem{theorem}{Theorem}
\theoremstyle{theorem}
\theoremstyle{definition}
\newtheorem{definition}{Definition}
\theoremstyle{remark}
\newtheorem{remark}{Remark}
\theoremstyle{proposition}
\newtheorem{proposition}{Proposition}
\theoremstyle{conjecture}
\newtheorem{conjecture}{Conjecture}
\theoremstyle{lemma}
\newtheorem{lemma}{Lemma}
\theoremstyle{corollary}
\newtheorem{corollary}{Corollary}
\theoremstyle{exercise}
\newtheorem{exercise}{Exercise}
\theoremstyle{example}
\newtheorem{example}{Example}

\newcommand{\R}{\mathbb{R}}
\newcommand{\C}{\mathbb{C}}

\newcommand{\mcal}{\mathcal}

\newcommand{\Ell}{\mcal{E}\ell\ell}

\newcommand{\on}{\operatorname}
\newcommand{\diff}{\,\,\mathrm{d}}

\newcommand{\spec}{\on{Spec}}
\newcommand{\proj}{\on{Proj}}
\newcommand{\bspec}{\on{\mathbf{Spec}}}
\newcommand{\sym}{\on{Sym}}

\newcommand{\D}{\on{D}}
\newcommand{\bounded}{\on{b}}

\newcommand{\qcoh}{\on{\mathbf{QCoh}}}
\newcommand{\coh}{\on{\mathbf{Coh}}}
\newcommand{\Vect}{\on{\mathbf{Vect}}}
\newcommand{\fin}{\on{fin}}
\newcommand{\lmod}{\on{\mathbf{-Mod}}}
\newcommand{\qTor}{\on{\mathbf{qTor}}}

\newcommand{\fuch}{\on{fuch}}
\newcommand{\DiffEq}{\on{\mathbf{DiffEq}}}
\newcommand{\DiffMod}{\on{\mathbf{DiffMod}}}

\newcommand{\supp}{\on{supp}}

\newcommand{\Hom}{\on{Hom}}
\newcommand{\RHom}{\on{\mathit{R}Hom}}
\newcommand{\Ext}{\on{Ext}}
\newcommand{\qKZ}{\on{\mathit{q}KZ}}
\newcommand{\qRH}{\on{\mathit{q}RH}}
\newcommand{\dyn}{{\on{dyn}}}

\newcommand{\xdashrightarrow}[2][]{\ext@arrow 0359\rightarrowfill@@{#1}{#2}}

\newcommand{\HRule}[1]{\rule{\linewidth}{#1}}

%\title{ \normalsize \textsc{A collection of notes}
%		\\ [2.0cm]
%		\HRule{0.5pt} \\
%		\LARGE \textbf{Notes for Thesis}
%		\HRule{2pt} \\ [0.5cm]
%        }
%
%\date{}
%\author{
%		Julianne Cai \\\\
%        }
%\maketitle
%\newpage
%
%\tableofcontents
%\newpage

\begin{document}

\begin{titlepage}

  \begin{center}
      \vspace*{1cm}
      \huge
      \textbf{On the Monodromy of the Double Affine Hecke Algebra}\\
  %    \textbf{Monodromy of $q$-Difference Equations and Double Affine Hecke Algebra Representations}\\
      \vspace{2cm}
      \Large
      \text{\textbf{Author}: Julianne Cai}\\
      \text{\textbf{Supervisors}: \href{https://sites.google.com/site/gufangzhao/}{Gufang Zhao}, \href{https://researchers.ms.unimelb.edu.au/~nganter@unimelb/}{Nora Ganter}}\\
      \includegraphics[scale=0.5]{unimelb-logo.jpg}\\
      A thesis submitted in partial fulfillment of the\\
      requirements for the degree of\\
      Master of Science\\
      in the\\
      School of Mathematics and Statistics\\
      at\\
      The University of Melbourne\\
      %Month Year
  \end{center}
  
  \end{titlepage}
  
  \pagenumbering{roman}
  
  \begin{abstract}
      We study the monodromy of the double affine Hecke algebra (DAHA) $\mathbf{\mathbf{\ddot{\mathbf{H}}}}$, and use 
      the $q$-Riemann-Hilbert correspondence of Sauloy in \cite{sauloy03} to produce representations 
      of the \emph{elliptic affine Hecke algebra} (ellAHA) $\mcal{H}^{\on{ell}}$ of Ginzburg-Kapranov Vasserot in \cite{gkv95}.\\\\ 
      In the rank one case, we construct the \emph{quantum Knizhnik-Zamolodchikov} (qKZ) functor 
      $\qKZ : \mcal{O}_{\mathbf{\ddot{\mathbf{H}}}} \to \coh^{\on{flat}}(\mcal{H}^{\on{ell}})$ from Cherednik's \cite{che90} 
      category $\mcal{O}$ for DAHA to a representation category of ellAHA, giving a $q$-analogue of the 
      trigonometric and rational Knizhnik-Zamolodchikov functors studied by Varagnolo-Vasserot in \cite{vv04} and
      Ginzburg-Guay-Opdam-Rouquier in \cite{ggor03}, respectively. \\\\
      Our techniques rely on an analysis of the $q$-difference equations arising from standard modules of DAHA, and we expect
      that their monodromy yields modules over the ellAHA. Along the way, we find generalisations of results of \cite{ggor03} to $\mcal{O}_{\mathbf{\ddot{\mathbf{H}}}}$.
      %We introduce a notion of category $\mcal{O}$ of the DAHA, which we denote by $\mcal{O}_{\mathbf{\ddot{\mathbf{H}}}}$.
      %We study basic properties of objects in $\mcal{O}_{\mathbf{\ddot{\mathbf{H}}}}$. This generalises certain results 
      %of Ginzburg-Guay-Opdam-Rouquier in \cite{ggor03}. Then, using a relation between the DAHA and the 
      %quantum torus of Baranovsky-Evans-Ginzburg in \cite{beg00}, we construct for each object in 
      %$\mcal{O}_{\mathbf{\ddot{\mathbf{H}}}}$ a $q$-difference system satisfying the fuchsian condition of 
      %Sauloy in \cite{sauloy03}.\\\\ 
      %We study examples of monodromy of certain DAHA modules, called \emph{standard modules}. The $q$-difference 
      %equations arising from these modules are identified with the qKZ functor, and we show that the 
      %monodromy of such modules are modules of the elliptic affine Hecke algebra $\mcal{H}^{\on{ell}}$.
  \end{abstract}
  
  \chapter*{Acknowledgements}
  
  I would like to express my immense gratitude to my thesis supervisors Gufang Zhao and Nora Ganter for their guidance 
  throughout my mathematical education, for the life advice they have given me, and their endless patience during the writing 
  of this thesis. My mathematical journey has benefited tremendously from their insights, and continuous support.\\\\
  My gratitude also extends to Valerio Toledano Laredo, who begun working with Gufang on a set of conjectures relating the 
  double affine Hecke algebra and the elliptic affine Hecke algebra in $2018$. 
  This thesis uses many of the ideas they discussed around that time. Their research notes were indispensable to many of the proofs 
  used in this thesis.\\\\
  I would also like to thank 
  Ali, Anthony, Davood, Erin, Fei, Grace, Haris, Jake, Jonah, Oliver, Yanchao, and Yuyang
  for their friendship, their interest in my research,
  and all the hours we have spent crowded in front of whiteboards talking about mathematics.
  I would also like to extend my gratitude to my family for their continued support throughout my master's.\\\\ 
  Finally, I want to thank Vivian Morrice: my best friend, and the love of my life. 
  
  \tableofcontents
  
  
  \chapter*{Introduction}
  \addcontentsline{toc}{chapter}{Introduction}
  
  Hecke algebras are an important tool for understanding representations of algebraic groups. The first type of Hecke algebra that one encounters is the finite Hecke algebra $\mathbf{H}$, which arises 
  as a $q$-deformation of the Weyl group $W$. Let $\mathbf{G}$ be a split, 
  reductive group scheme with Borel subgroup $\mathbf{B}$. 
  Let $\mathbb{F}_p$ be a 
  finite field and let $V$ be an irreducible, admissible representation 
  of the algebraic group $\mathbf{G}(\mathbb{F}_p)$. Then, the subspace of 
  $\mathbf{B}(\mathbb{F}_p)$-fixed points $V^{\mathbf{B}(\mathbb{F}_p)}$ can be equipped with the structure
  of a $\mathbf{H}$-module. In particular, there is an equivalence of categories
  \begin{equation}\label{eqn_equiv_cat}
      \mathbf{Rep}_\C(\mathbf{G}(\mathbb{F}_p)) \longrightarrow \mathbf{H}\lmod,\quad V \longmapsto V^{\mathbf{B}(\mathbb{F}_p)},
  \end{equation}
  which sends irreducible objects to irreducible objects (see \cite{bum10}).\\\\
  Passing to the setting of affine spaces, one extends the finite Weyl group by the weight or coweight lattice to 
  produce the \emph{affine Weyl group} $\widetilde{W}$. The
  \emph{affine Hecke algebra} $\mathbf{\dot{H}}$ arises as a $q$-deformation
  of $\widetilde{W}$. 
  The AHA was first described in \cite{im65}, 
  who constructed it as a convolution algebra over the algebraic group $\mathbf{G}(\mathbb{Q}_p)$ 
  (see \cite{hkp10}, \cite{im65}, \cite{cai22}). One may recover the finite
  Hecke algebra $\mathbf{H}$ from this convolution construction by considering a similar convolution algebra on the residue
  field $\mathbf{G}(\mathbb{F}_p)$ (see \cite{bum10}). \\\\
  The affine Hecke algebra plays an important role in studying representations of
  $p$-adic groups. In \cite{bor76}, Borel proved a $p$-adic analogue
  of \eqref{eqn_equiv_cat} by replacing the Borel subgroup by the \emph{Iwahori 
  subgroup} $\mathbf{I}$, which was first
  introduced in \cite{im65}, as the pre-image of $\mathbf{B}(\mathbb{F}_q)$ under the map 
  $\mathbf{G}(\mathbb{Z}_p) \to \mathbf{G}(\mathbb{Q}_p)$, where $\mathbb{Z}_p$
  are the $p$-adic integers. 
  In this case, given any irreducible, admissible
  representation $V$ of $\mathbf{G}(\mathbb{Q}_p)$, the subrepresentation
  $V^{\mathbf{I}}$ has the structure of a finite-dimensional $\mathbf{\dot{H}}$-module, and is 
  irreducible if $V$ is irreducible. 
  Borel in \cite{bor76} showed that there is an equivalence of categories
  $$\mathbf{Rep}_\C(\mathbf{G}(\mathbb{Q}_p)) \longrightarrow \mathbf{\dot{H}}\lmod,\quad V\longmapsto V^{\mathbf{I}}.$$
  The $\mathbf{G}(\mathbb{Q}_p)$-representation $V$ is typically infinite-dimensional \cite{bum10}, 
  whereas $V^{\mathbf{I}}$ is finite-dimensional. By using the affine Hecke algebra, one reduces an 
  infinite-dimensional problem to a finite-dimensional one.\\\\
  Casselman, in\cite{cas80}, showed that under the correspondence established by 
  Borel, all irreducible admissible representations of $\mathbf{G}(\mathbb{Q}_p)$ arising from an
  irreducible $\mathbf{\dot{H}}$-module arise from a particular class of 
  parabolically induced $\mathbf{G}(\mathbb{Q}_p)$-representations called 
  \emph{unramified principal series representations}, which play an important
  role in the local Langlands program (see \cite{gh24}, \cite{bh06}, \cite{rw21}). All irreducible representations of 
  $\mathbf{\dot{H}}$ have since been classified by Kazhdan and Lusztig in \cite{kl87}, using the ${}^L\mathbf{G}(\C)\times \C^\times$-equivariant 
  $K$-theory of Springer fibres. We refer the interested reader to \cite{cg09} for an 
  exposition of this construction. \\\\
  Hecke algebras appear in many other fields of mathematics as well.
  In \cite{kl87}, Kazhdan and Lusztig used affine Hecke algebras to define 
  Kazhdan-Lusztig polynomials, which are combinatorial objects that encode
  deep representation theoretic information, and are still extensively studied
  to this day. In \cite{jon87}, affine Hecke algebras were used to define Jones
  polynomials, which is also an object of extensive study in relation to quantum
  toplogy and knot theory. \\\\
  The \emph{double affine Hecke algebra} (DAHA) $\mathbf{\ddot{\mathbf{H}}}$ was first
  introduced by Cherednik in \cite{che90}, and arises as a non-trivial extension
  of the AHA by the (co)-weight lattice.
  The representations of $\mathbf{\ddot{\mathbf{H}}}$ are generally not well-understood, though 
  there is a classification of the simple, integrable modules of the DAHA using perverse
  sheaves given in \cite{vas05}. The usual representation-theoretic convolution construction that one has 
  for AHA using an algebraic group, however, is not available in the DAHA case.
  This problem has been partially resolved by Braverman and Kazhdan 
  in \cite{bk11} for the spherical case. In their paper, the usual reductive 
  $p$-adic group is replaced by an affine Kac-Moody group, to which they applied
  the convolution algebra construction for the spherical Hecke algebra.\\\\
  The \emph{elliptic affine Hecke algebra} (ellAHA) $\mcal{H}^{\on{ell}}$ --- first introduced in \cite{gkv95} --- is an object that is closely related
  to the double affine Hecke algebra. A Deligne-Langlands classification of the irreducible representations 
  of $\mcal{H}^{\on{ell}}$ was done by Zhao-Zhong in \cite{ZZ21} using the 
  ${}^L\mathbf{G}(\C) \times \C^\times$-equivariant elliptic cohomology of the Springer fibre.\\\\
  One may obtain three more Hecke algebras through various degenerations of the DAHA. One may view the DAHA as an subalgebra of $\C[\mathbf{X}^\pm][\mathbf{Y}^\pm]$, where $\mathbf{X}$ and 
  $\mathbf{Y}$ are the weight and coweight lattices, respectively.
  The trigonometric DAHA arises from the DAHA by degenerating one of the weight lattices, giving us two copies of the trigonometric DAHAs, denoted by 
  $\mathbf{\mathbf{\ddot{\mathbf{H}}}}_{\mathbf{X}}'$ and $\mathbf{\ddot{\mathbf{H}}}_{\mathbf{Y}}$, which are 
  subalgebras of $\C[\mathbf{X}][\mathbf{Y}^\pm]$ and $\C[\mathbf{X}^\pm][\mathbf{Y}]$, respectively. The module categories of these $\mathbf{\ddot{\mathbf{H}}}'_\mathbf{X}$ 
  and $\mathbf{\ddot{\mathbf{H}}}'_\mathbf{Y}$ 
  are related by the Fourier transform of Evens-Mirkovic in \cite{em97}. Zhao-Zhong in \cite{ZZ24} proved an elliptic analogue of this, and showed that the Fourier-Mukai 
  transformation takes the elliptic affine Hecke algebra $\mcal{H}^{\on{ell}}$ to its Langlands dual ${}^L\mcal{H}^{\on{ell}}$.\\\\
  A particularly interesting class of representations are known as \emph{monodromy representations}, 
  which are representations arising from the monodromy of systems of partial differential equations.
  Guay-Ginzburg-Opdam-Rouquier \cite{ggor03}, and Vasserot-Varagnolo \cite{vv04} studied systems of 
  partial differential equations arising from the rational and trigonometric DAHA, respectively (see Appendix \ref{app_B}). 
  One then obtains the rational DAHA by degenerating 
  the other weight lattice in the trigonometric DAHA. 
  They found that the rational and trigonometric DAHA both give rise to a specific system of PDEs called 
  the \emph{Knizhnik-Zamolodchikov (KZ) connection}, which are a family of holonomic partial differential equations first 
  introduced by Knizhnik-Zamolodchikov in \cite{kz84} in the study of two-dimensional quantum field theories.\\\\
  In the trigonometric DAHA case, the monodromy representations of the KZ equations gives a functor from the 
  category $\mcal{O}$ of the trigonometric DAHA to the representation category of the affine braid group. As a byproduct, 
  one obtains a functor from the category $\mcal{O}$ to the representation category of the AHA by taking a categorical 
  quotient of the representation category of the affine braid group. In the rational case, \cite{ggor03} showed that the monodromy factors through the 
  representation category of the finite Hecke algebra. These functors were called the trigonometric and rational Knizhnik-Zamolodchikov functors, 
  respectively. \\\\
  Frenkel-Reshetikhin \cite{fr92} later derived a $q$-analogue of the Knizhnik-Zamolodchikov equations 
  associated to a quantum affine algebra, called the \emph{quantum Knizhnik-Zamolodchiov} (qKZ) equations.
  Rather than PDEs, the qKZ equations are a system of linear $q$-difference equations, which are 
  $q$-analogues of the usual partial derivative. Taking the formal limit $q\to 1$ recovers the usual PDEs defining the 
  Knizhnik-Zamolodchikov equations of \cite{kz84}.\\\\
  Cherednik, in \cite{che92}, constructed the quantum affine Knizhnik-Zamolodchikov equations associated 
  to an affine Hecke algebra module and showed that it formed a holonomic system of $q$-difference equations.
  Stokman, in \cite{sto10}, related the quantum affine KZ equations to the case when the AHA module is 
  induced from a character of the parabolic subalgebra of the AHA. Moreover, \cite{sto10} also constructs a link between 
  the double affine Hecke algebra and the qKZ equations in \cite[\S 3]{sto10}.\\\\
  In this thesis, we study the monodromy of the qKZ equation arising from standard modules of the DAHA. The techniques 
  used in \cite{vv04} and \cite{ggor03} do not generalise immediately, as the mondromy of $q$-difference equations is 
  less well-known.
  The main theorem employed in both the papers of \cite{vv04} and \cite{ggor03} to compute monodromy is the \emph{Riemann-Hilbert correspondence}.
  A $q$-analogue of the Riemann-Hilbert correspondence is known, but only for the case of functions of one 
  variable. The proof of this result is due to Sauloy in \cite{sauloy03}. The $q$-Riemann-Hilbert for $q$-difference systems
  of more than one variable is still conjectural.\\\\
  The rank one DAHA gives rise to $q$-difference equations of one variable, which allows us to apply the results of 
  \cite{sauloy03} to study the monodromy of DAHA. We consider certain induced modules in the category $\mcal{O}$ of DAHA,
  and show that these modules gives rise to the qKZ equations of \cite{fr92} corresponding to $\mcal{U}_q(\widehat{\mathfrak{sl}_2})$.
  The monodromy of the qKZ equations are 
  well-studied in \cite{fr92}, \cite{efk98}, and we use these results together with the $q$-Riemann-Hilbert correspondence 
  of \cite{sauloy03} to outline how one may produce representations of the ellAHA.
  We notate the Hecke algebras we've described in the following way:\\
  \begin{table}[!ht]
      \centering
      \begin{tabular}{|l|l|l|l|l|l|}
      \hline
          ~ & Finite & Affine & Double Affine & Elliptic & Dynamical \\ \hline
          Full & $\mathbf{H}$ & $\mathbf{\dot{H}}$ & $\mathbf{\ddot{\mathbf{H}}}$ & $\mcal{H}^{\on{ell}}$ & $\mcal{H}^\dyn$ \\ \hline
          Trigonometric & ~ & $\mathbf{\dot{H}}'$ & $\mathbf{\ddot{\mathbf{H}}}'$ & ~ & ~ \\ \hline
          Rational & ~ & ~ & $\mathbf{\ddot{\mathbf{H}}}''$ & ~ & ~ \\ \hline
      \end{tabular}
  \end{table}\\
  This entire body of work can be summarised using the diagram:
  $$\begin{tikzcd}
      \mcal{H}^{\on{ell}} \arrow[rrrrrr, leftrightarrow, "{\text{Fourier-Mukai Transform, \cite{ZZ24}}}"]                                                                                                                                              &  &                                                                       &                                                                                        &                                        &  & {}^L\mcal{H}^{\on{ell}}                                                                                                                     \\
      \mathbf{\ddot{\mathbf{H}}} \arrow[d, "\text{Weight Degeneration}"'] \arrow[rrrrrr, leftrightarrow, "\text{Cherednik's $S$-Duality}"] \arrow[u, "\textcolor{red}{\qKZ}"]                                                                          &  &                                                                       &                                                                                        &                                        &  & \mathbf{\ddot{\mathbf{H}}} \arrow[d, "\text{Coweight Degeneration}"] \arrow[u, "\textcolor{red}{\qKZ}"']                                    \\
      \mathbf{\mathbf{\ddot{\mathbf{H}}}}'_{\mathbf{X}} \arrow[rr, "{\text{Trig KZ, \cite{vv04}}}"] \arrow[rrrrrr, leftrightarrow, "{\text{Fourier Transform, \cite{em97}}}"', bend right] \arrow[rrrddd, "\text{Coweight Degeneration}"', bend right] &  & \mathbf{\mathbf{\dot{H}}}^{\mathbf{X}} \arrow[rr, "\text{Not known}"] &                                                                                        & \mathbf{\mathbf{\dot{H}}}^{\mathbf{Y}} &  & \mathbf{\ddot{\mathbf{H}}}'_\mathbf{Y} \arrow[ll, "{\text{Trig KZ, \cite{vv04}}}"'] \arrow[lllddd, "\text{Weight Degeneration}", bend left] \\
                                                                                                                                                                                                                                       &  &                                                                       &                                                                                        &                                        &  &                                                                                                                                             \\
                                                                                                                                                                                                                                       &  &                                                                       &                                                                                        &                                        &  &                                                                                                                                             \\
                                                                                                                                                                                                                                       &  &                                                                       & \mathbf{\mathbf{\ddot{\mathbf{H}}}}'' \arrow[d, "{\text{Rational KZ, \cite{ggor03}}}"] &                                        &  &                                                                                                                                             \\
                                                                                                                                                                                                                                       &  &                                                                       & \mathbf{H}                                                                             &                                        &  &                                                                                                                                            
  \end{tikzcd}$$
  The red arrows labelled \textcolor{red}{"$\qKZ$"} denotes the portion of the literature advanced by this thesis.\\\\
  Our results fit into a larger body of work concerning the monodromy of $q$-difference equations in relation to 
  elliptic cohomology.
  Aganagic-Okounkov in \cite{ao16} studied $q$-difference equations originating from the quantum $K$-theoretic counts 
  of rational curves in a Nakajima variety $X$, and described its monodromy in terms of the equivariant elliptic 
  cohomology of $X$. In the case for which $X = T^\ast\mathbb{P}^1$, the $q$-difference equations that 
  arise are precisely the rank one quantum Knizhnik-Zamolodchikov equations, which we study extensively 
  in this thesis.\\\\ 
  In Okounkov-Smirnov's paper \cite{os22}, the action of the quantum dynamical Weyl group on the equivariant quantum $K$-theory of a Nakajima variety $X$
  is considered. In the case for which $X = T^\ast\mathbb{P}^1$, we once again see the qKZ equations appearing once again 
  in \cite[(122)]{os22}.\\\\ 
  Our contributions in this direction can be summarised in the following diagram:
  $$\begin{tikzcd}
                                                                  &  & \text{DAHA} \arrow[rrd, "\text{This Thesis}"] \arrow[lld, "\cite{beg00}"'] &  &                                                                                     \\
  \text{Quantum Torus} \arrow[d]                                  &  &                                                                                                   &  & \text{Elliptic AHA}                                                                 \\
  \text{$q$-Difference Equations} \arrow[rrd, "\cite{sauloy03}"'] &  &                                                                                                   &  & \begin{aligned}
      \text{Elliptic Cohomology}\\
      \text{of Nakajima Varieties}
  \end{aligned}\arrow[lld, "\cite{ao16}"] \arrow[u] \\
                                                                  &  & \text{Vector Bundles on $E$}                                                                      &  &                                                                                    
  \end{tikzcd}$$
  Moreover, \cite{ZZ21} showed that the irreducible representations of $\mcal{H}^{\on{ell}}$ 
  are in one-to-one correspondence with certain nilpotent Higgs bundles on elliptic curves. 
  As a corollary, our result will give a parametrisation of these Higgs bundles in terms of 
  monodromy DAHA representations.
  
  \section*{Statement of Main Theorem and Conjectures}
  
  We make substantial progress on the following conjecture:
  \begin{conjecture}\label{main_thm1}
      There exists a functor 
      $$\qKZ : \mcal{O}_{\mathbf{\ddot{\mathbf{H}}}} \longrightarrow \coh^{\on{flat}}(\mcal{H}^{\on{ell}}),$$
      called the \emph{quantum Knizhnik-Zamolodchikov} (qKZ) functor.
      Moreover, the qKZ functor factors through the Serre quotient 
      $\mcal{O}_{\mathbf{\ddot{\mathbf{H}}}}/\mcal{O}_{\ddot{\mathbf{H}}}^{\on{tor}}$,
      and induces a map to the module category $\coh^{\on{flat}}(\mcal{H}^{\on{ell}})$:
      $$\begin{tikzcd}
          \mathcal{O}_{\mathbf{\ddot{\mathbf{H}}}} \arrow[rd, two heads] \arrow[rr, "\qKZ"] &                                                                                                       & \coh^{\on{flat}}(\mathcal{H}^{\on{ell}}) \\
                                                                              & \mathcal{O}_{\mathbf{\ddot{\mathbf{H}}}}/\mathcal{O}_{\ddot{\mathbf{H}}}^{\operatorname{tor}} \arrow[ru, dotted] &                                   
  \end{tikzcd}$$
  \end{conjecture}
  
  We have further conjectures about certain properties that the $\qKZ$ functor should satisfy:
  
  \begin{conjecture}\label{main_thm2}
  The restriction of $\qKZ$ to 
  $\mcal{O}_{\mathbf{\ddot{\mathbf{H}}}}/\mcal{O}_{\ddot{\mathbf{H}}}^{\on{tor}}$ is a 
  fully faithful, essentially surjective functor.
  \end{conjecture}
  
  Additionally, we expect to be able to extend this work to representations of the \emph{dynamical affine Hecke algebra} (see Appendix \ref{app_D}, \cite{ZZ21}, \cite{LZZ23}, \cite{ZZ24}).\\\\
  Let ${}^L\mcal{H}^{\on{ell}}$ be the Langlands dual of the elliptic 
  affine Hecke algebra $\mcal{H}^{\on{ell}}$. 
  In \cite[Corollary 5.1]{ZZ24}, it was shown that the Fourier-Mukai 
  transformation gives a functor:
  $$\mcal{FM} : \coh^{\on{fin}}(\mcal{H}^{\on{ell}}) \longrightarrow \coh^{\on{flat}}({}^L\mcal{H}^{\on{ell}}).$$
  Moreover, \cite[Corollary 5.2]{ZZ24} showed that there is an 
  inverse Fourier-Mukai functor $\widehat{\mcal{FM}}$.
  
  \begin{conjecture}\label{main_thm3}
      Let $\mcal{H}^{\on{dyn}}$ be the dynamical Hecke algebra.
      Then, there exists a functor 
      $$\C[\mathbf{Y}]\lmod \longrightarrow \mcal{H}^{\on{dyn}}\lmod,$$
      such that the diagram 
      $$\begin{tikzcd}
  \C[\mathbf{Y}]\lmod \arrow[r, "\on{Ind}_{\C[\mathbf{Y}]}^{\mathbf{\dot{H}}^Y}"] \arrow[d] & \mathbf{\dot{H}}^{\mathbf{Y}}\lmod \arrow[r, "\on{Ind}_{\mathbf{\dot{H}}^Y}^{\mathbf{\ddot{\mathbf{H}}}}"] & \mcal{O}_{\ddot{\mathbf{H}}} \arrow[d, "\qKZ"]                                         \\
  \coh(\mcal{H}^{\on{dyn}}) \arrow[rrdd, "\on{forget}"']                                    &                                                                                                   & \coh^{\on{flat}}(\mcal{H}^{\on{ell}}) \arrow[dd, "\widehat{\mcal{FM}}"', bend right] \\
                                                                                            &                                                                                                   &                                                                                           \\
                                                                                            &                                                                                                   & \coh^{\on{fin}}({}^L\mcal{H}^{\on{ell}}) \arrow[uu, "\mcal{FM}"', bend right]     
  \end{tikzcd}$$
      commutes. 
  \end{conjecture}
  \section*{Structure of this Thesis}
  
  Chapter \ref{chap1} of this thesis introduces all the algebraic objects 
  that we will require. A brief introduction to $q$-difference equations and their monodromy 
  are given, and we introduce the DAHA and outline its relation to $q$-difference equations. Then, the elliptic affine Hecke algebra of \cite{gkv95} is introduced.
  Following \cite{gkv95}, \cite{LZZ23}, \cite{ZZ21}, and \cite{ZZ24}, we detail the construction of the elliptic
  affine Hecke algebra, and construct its representation category $\coh(\mcal{H}^{\on{ell}})$.\\\\
  Chapter \ref{chap2} details the construction of the 
  category $\mcal{O}$ of DAHA -- which we denote by $\mcal{O}_{\mathbf{\ddot{\mathbf{H}}}}$. This definition was originally given by \cite{che95}, and can also be 
  found in \cite{jv19}. We construct $\Delta$-filtrations and standard modules
  in $\mcal{O}_{\mathbf{\ddot{\mathbf{H}}}}$, generalising results of \cite{ggor03} for the category $\mcal{O}$ of the rational 
  DAHA. We classify elements of 
  $\mcal{O}_{\mathbf{\ddot{\mathbf{H}}}}$ in terms of modules admitting $\Delta$-filtrations in 
  Proposition \ref{prop_filtration}.  This is a modification of \cite[Proposition 2.2]{ggor03}.
  We also briefly describe the torsion subcategory $\mcal{O}_{\mathbf{\ddot{\mathbf{H}}}}^{\on{tor}}$ of 
  $\mcal{O}_{\mathbf{\ddot{\mathbf{H}}}}$, and show that it is a Serre category.\\\\
  Chapter \ref{chap3} outlines the construction of the \emph{quantum torus} of \cite{beg00}, which is an 
  algebra generated by $q$-difference equations acting on the algebra of meromorphic functions on the weight lattice.
  \cite[Theorem 7.2]{beg00} proves that there is an isomorphism between the localised DAHA $\mathbf{\ddot{\mathbf{H}}}_{\on{loc}}$,
  and the quantum torus. In Proposition \ref{prop_cat_o_Fuchsian}, 
  we prove that the restriction of the isomorphism of \cite[Theorem 7.2]{beg00} to $\mcal{O}_{\mathbf{\ddot{\mathbf{H}}}}$ 
  lands in the Fuchsian quantum torus. This allows us to apply the $q$-Riemann-Hilbert functor of \cite{sauloy03} to 
  the quantum torus. We also define the \emph{$W$-equivariant connection category},
  which is the connection category of \cite{sauloy03}, but equipped with the action of a Weyl group $W$.
  The $q$-Riemann-Hilbert functor lands in this category when applied to the Fuchsian $W$-equivariant quantum torus.\\\\
  In Chapter \ref{chap4}, we construct the
  quantum Knizhnik-Zamolodchikov (qKZ) functor for a class of induced DAHA modules called standard modules. A key result is Proposition \ref{prop_qkz_daha}, which shows 
  that choosing an appropriate basis for the standard module gives the trigonometric $R$-matrix of the evaluation modules 
  of $\mcal{U}_q(\widehat{\mathfrak{sl}_2})$. This is the 
  coefficient matrix for the qKZ equation of \cite{fr92} corresponding to $\mcal{U}_q(\widehat{\mathfrak{sl}}_2)$. 
  Using techniques from \cite{efk98}, \cite{sauloy03}, we then compute the monodromy matrix of the qKZ equation. 
  We discuss how this monodromy matrix can be used to produce an $\mcal{H}^{\on{ell}}$-module. The only technical condition 
  that remains to be checked is Conjecture \ref{annoying}.
  
  \chapter{Preliminaries}\label{chap1}
  \pagenumbering{arabic}
  \section{$q$-Difference Equations}\label{sec_qdiff}
  
  Throughout, let us fix a complex number $q\in \C^\times$ such that 
  $\vert q\vert < 1$ , and let $E := \C^\times /q^{\mathbb{Z}}$ be an 
  elliptic curve. Let $\mathbb{P}^1 = \on{Proj}\C[x,y]$ be the complex projective
  space. Let $\mcal{M}$ be the sheaf of meromorphic functions over $\mathbb{P}^1$.
  One may think of $q$-difference equations 
  as $q$-deformations of the usual derivative. For some $q\neq 1$, 
  one defines the \emph{$q$-derivative operator} $D_q$ by \cite[\S 3.1]{koe18}:
  $$D_qf(z) = \frac{f(z) - q^zf(qz)}{(1-q)z},\quad z\neq 0,$$
  where $f\in \C(z)$ (c.f. \cite[(1.16)]{am10}, \cite{koe18}).
  Observe that $D_qf(0) = f'(0)$, assuming the derivative exists. One readily checks that
  $D_qf(x) \to f'(x)$ as $q\to 1$. 
  Generally, one can define $q$-difference 
  operators on functions of multiple variables. However, we restrict ourselves to the case 
  of one variable. The general theory for $q$-difference equations 
  of multiple variables is not well-understood. 
  To illustrate this idea of $q$-difference equations being $q$-analogues of differential equations, 
  let us consider some $q$-analogues of classical special functions:
  
  \begin{example}[$q$-Hypergeometric Equation]\label{ex_hypergeom}
      The \emph{hypergeometric equation} is given by:
      $${}_2F_1(a,b,c;z) := \sum_{n=0}^\infty \frac{(a)_n(b)_n}{(c)_n} \frac{z^n}{n!},$$
      where $(a)_n$ is the \emph{Pochhammer symbol} defined by 
      $$(a)_n := \begin{cases}
          1 \quad &\text{if} \quad n=0\\
          a(a+1)\cdots (a+n-1),\quad &\text{if} \quad n > 0
      \end{cases}.$$ The hypergeometric equation satisfies the 
      \emph{hypergeometric equation}, given by 
      $$z(1-z)\frac{\mathrm{d}^2}{\mathrm{d} z^2} f(z) + (c-(a+b+1)z) \frac{\mathrm{d}}{\mathrm{d} z}f(z) - ab f(z) = 0.$$
      There is a well-known $q$-analogue of this special function.
      First, define the \emph{$q$-Pochhammer symbol} to be: 
      $$(a;q)_n := \prod_{n\geq 0} (1-q^na).$$
      Then, the \emph{$q$-hypergeometric equation} is defined by 
      $$\pFq{2}{1}{a,b}{c}{q;z} := \sum_{n\geq 0} \frac{(a;q)_n (b;q)_n}{(q;q)_n(c;q)_n} z^n,$$
      and satisfies the second order $q$-difference equation:
      \begin{equation}\label{eqn_q_hypergeom}
          z(q^c-q^{a+b+1}z) D_q^2f(z) + \left(\frac{1-q^c}{1-q} + \frac{(1-q^a)(1-q^b) - ( 1-q^{a+b+1})}{1-q}z\right) D_qf(z) - \frac{(1-q^a)(1-q^b)}{(1-q)^2}f(z) = 0.
      \end{equation}
      One checks that the above equation tends to the hypergeometric 
      differential equation as $q \to 1$. 
      Substituing the formula for $D_qf(z)$ into \eqref{eqn_q_hypergeom} gives
      us the formula:
      \begin{equation}
          (q^c - q^{a+b}z) f(q^2z) + (-(q^c + q) + (q^a + q^b)z)f(qz) + (q-z)f(z) = 0.
      \end{equation}
  \end{example}
  
  \begin{example}[$q$-Gamma Function]
      The usual Gamma function also has a $q$-analogue, called 
      the \emph{$q$-Gamma function}. It is defined to be:
      $$\Gamma_q(z) = \frac{(q;q)_\infty}{q^z:q)_\infty}(1-q)^{1-z}, \quad 0 < \vert q \vert < 1.$$
      One checks that for $n\in \mathbb{N}$, 
      $$\Gamma_q(n) = \frac{(q;q)_{n-1}}{(1-q)^{n-1}},$$
      and 
      $$\Gamma_q(x+1) = \frac{1-q^x}{1-q}\Gamma_q(x).$$
      Again, taking the formal limit $q\to 1$ recovers the usual Gamma function
      $\Gamma(x)$.
      See \cite[\S 1.6]{am10} for more details and historical remarks.
  \end{example}
  
  There are many more $q$-analogues of these kinds of special 
  functions. We refer the reader to \cite{am10} for a comprehensive treatise of 
  this subject.
  For the purpose of this text it is sufficient to understand the 
  $q$-hypergeometric equation, and we will see that this equation
  arises as a solution to the rank one quantum Knizhnik-Zamolodchikov equation.
  This  is the central object that we will be studying in relation to the
  double affine Hecke algebra (see \ref{chap4}).
  
  \begin{example}[$q$-Power Function]
      The power function $f(z) = (1-z)^{-a}$ can be defined as the solution
      of the differential equation
      $$\frac{\mathrm{d} }{\mathrm{d} z}f(z) = \frac{a}{1-z} f(z).$$
      $q$-Analogously, the $q$-power function 
      $$g(z) := \prod_{n=0}^\infty \frac{1-zq^{n+a}}{1-zq^n},$$
      is a solution to the $q$-difference equation $$g(qz) = \frac{1-z}{1-zq^a}g(z).$$
  \end{example}
  
  Let us formally define the notion of a $q$-difference equation, and what it 
  means to solve them.
  
  \begin{definition}\label{def_extension}
      Let $K$ be a field, and $\sigma$ an automorphism of $K$. Then, the pair
      $(K,\sigma)$ is called a \emph{difference field}. Moreover, if $K'$ is a
      field extension of $K$, and $\sigma'$ is an automorphism of $K'$,
      then $(K',\sigma')$ is an \emph{extension} of $(K,\sigma)$ 
      if $\sigma'\vert_K = \sigma$.
  \end{definition}
  As we saw in Example \ref{ex_hypergeom}, any $q$-difference equation of the
  form $$a_n(q,z)D_q^nf(z) + \cdots + a_0(q,z) f(z) = 0,$$
  can be re-written into an equation of the form 
  $$A_n(q,z) f(q^nz) + \cdots + A_0(q,z)f(z) = 0.$$
  Thus from now on, when we speak of $q$-difference equations, we will 
  speak of the ones of the latter form. 
  Let $K$ be some function field (i.e. $\C(z)$, $\mcal{M}(\C^\times)$), 
  and let $\sigma_q \in \on{End}(K)$ be the 
  \emph{$q$-difference} (or \emph{$q$-shift}) operator acting by 
  $\sigma_q : f(z) \mapsto f(qz)$. Then, the pair $(K,\sigma_q)$ forms a 
  difference field. In particular, note that $(\mcal{M}(\C^\times),\sigma_q)$
  is a field extension of $(\C(z), \sigma_q)$, where the operator $\sigma_q$ is extended 
  from $\C(z)$ to $\mcal{M}(\C^\times)$ in the obvious way. 
  More generally, one may consider $q$-difference equations of the form 
  \begin{equation}\label{eqn_thing}
      \sigma_q X = AX,
      \end{equation}
  where $X = (f_1,\cdots,f_n)^T$, $A \in \on{GL}_n(\mcal{M}(\C^\times))$, and the 
  operator $\sigma_q$ acts on $X$ element-wise. We will call $q$-difference 
  equations written in the form \eqref{eqn_thing} a \emph{$q$-difference system}, 
  with \emph{coefficient matrix} $A$.
  Furthermore, given a difference field $(K,\sigma)$,
  one may form the category of $q$-difference equations, which we denote by 
  $\mathbf{DiffEq}(K,\sigma)$. Its objects are pairs of the form 
  $(K^n,A)$, where $A\in \on{GL}_n(K)$. A morphism $(K^n,A) \to (K^p,B)$ 
  is a matrix $F \in \on{Mat}_{p,n}(K)$ satisfying the relation $(\sigma F)A = BF$.
  One may take tensor products of objects by 
  $$(K^n,A) \otimes (K^p,B) = (K^{np},A\otimes B),$$
  where $\otimes$ in this case denotes the Kronecker product, 
  and thus $A_1\otimes A_2 \in \on{Mat}_{np}(K)$. The tensor product 
  equips $\mathbf{DiffEq}(K,\sigma)$ with the structure of a Tannakian category
  (see \cite[\S 1.1.2]{sauloy03}). In particular, it is a neutral Tannakian
  category with a forgetful functor $(K^n,A) \mapsto K^n$.
  \begin{example}[Quantum Knizhnik-Zamolodchikov Equation]\label{ex_qKZ}
      Let $V$ and $W$ be two weight modules of 
      $\mcal{U}_q(\widehat{\mathfrak{sl}_2})$ with weights $m,n$ respectively. 
      Then, when $m=n=1$,  the trigonometric $R$-matrix of the evaluation modules 
      is given by 
      $$R(z) = E_{11}\otimes E_{11} + E_{22} \otimes E_{22} + \frac{1-z}{q-q^{-1}z} (E_{22}\otimes E_{11} + E_{11}\otimes E_{22}) + \frac{q-q^{-1}}{q-q^{-1}z} (E_{12} \otimes E_{21} + zE_{21}\otimes E_{12}),$$
      which is given in matrix form by:
      $$R(z) = \renewcommand\arraystretch{2}\begin{pmatrix}
          1 & 0 & 0 & 0\\
          0 & \displaystyle\frac{1-z}{q-q^{-1}z} & \displaystyle\frac{z(q-q^{-1})}{q-q^{-1}z} & 0\\
          0 & \displaystyle\frac{q-q^{-1}}{q-q^{-1}z} & \displaystyle\frac{1-z}{q-q^{-1}z} & 0\\
          0 & 0 & 0 & 1
      \end{pmatrix}.$$
      Then, the \emph{quantum Knizhnik-Zamolodchikov} equation is the 
      $q$-difference system given by 
      $$\begin{pmatrix}
          \Phi_1(qz_1,z_2)\\
          \Phi_2(qz_1,z_2) 
          \end{pmatrix} = R\left(\frac{z_1}{z_2}\right) \cdot \begin{pmatrix}
          \Phi(z_1,z_2)\\
          \Phi(z_1,z_2)
      \end{pmatrix}.$$
      (c.f. \cite{efk98}, \cite{fr92}, \cite{sto10}).
      Solutions of this equation, as well as its relation to representations of 
      double affine Hecke algebra and elliptic affine Hecke algebra 
      will be explored in Chapter \ref{chap4}.
  \end{example}
  
  
  \begin{definition}
      Let $\sigma_qX = AX$ be a $q$-difference system, and let $P \in \on{GL}_n(K)$ be a matrix. Then, a \emph{gauge transform} of $A$ by the \emph{gauge 
      transformation matrix} $P$ is the matrix 
      $$P \cdot [A]_q = (\sigma_qP) \cdot A \cdot P^{-1}.$$
  \end{definition}
  A particularly nice property of $q$-difference systems is whether or not 
  it is \emph{Fuchsian}. The reasons for this will become clear when we 
  begin explicitly building solutions over the field $\mcal{M}(\C^\times)$ 
  in the next section.
  \begin{definition}
      A $q$-difference system $\sigma_qX = AX$, with $A \in \on{GL}_n(\C(z))$ 
      is called \emph{Fuchsian} at $0$ (respectively, $\infty$) if 
      $A(0) \in \on{GL}_n(\C)$ (respectively, $A(\infty) \in \on{GL}_n(\C)$), 
      or if there exists a meromorphic gauge transformation $P \in \on{GL}_n(\mcal{M}(\C^\times))$ such that $(\sigma_qP)\cdot A \cdot P^{-1}$ is Fuchsian at 
      $0$ (resp. $\infty$).
  \end{definition}
  Let $\mathbf{Fuch}$ be the category of Fuchsian equations. It is a full 
  subcategory of $\on{\textbf{DiffEq}}(\C(z),\sigma_q)$. (TODO: come back and 
  elaborate more on this)
  
  \subsection{Solutions of $q$-difference Equations}\label{sec_qdiff_soln}
  
  So far, we have considered $q$-difference systems over $\C(z)$. However,
  we wish to build solutions in a field extension of $\C(z)$, specifically 
  $\mcal{M}(\C^\times)$. Both $\C(z)$ and $\mcal{M}(\C^\times)$ come equipped
  with a $q$-shift operator $\sigma_q$ that acts on elements in the same way.
  Thus, $(\mcal{M}(\C^\times),\sigma_q)$ is a field extension of $(\C(z),\sigma_q)$.
  By abuse of notation, we use $\sigma_q$ to denote the field automorphism in both categories.
  Observe first that
  $$\mcal{M}(\C^\times)^{\sigma_q} = \mcal{M}(E),$$
  the meromorphic sections of an elliptic curve 
  $E = \C^\times/q^{\mathbb{Z}}$. Note that sections of the structure sheaf $\mcal{O}_E$
  consist of functions over $\C^\times$ for which $f(q^{\mathbb{Z}}u) = f(u)$. Such functions 
  are thus called \emph{elliptic} or \emph{$q$-periodic}. Moreover, $\mcal{M}(E)$ can be viewed as a 
  localisation of the global sections of the structure sheaf $\mcal{O}_E$.
  The solution functor is given by the fibre functor:
  $$\on{Sol} : \mathbf{DiffEq}(\mcal{M}(\C^\times), \sigma_q) \longrightarrow \mathbf{Vect}_{\mcal{M}(\C^\times)}.$$
  Since the map 
  $z\mapsto qz$ has only two fixed points on the Riemann sphere --- 
  specifically $0$ and $\infty$ --- solutions only exist in neighbourhoods 
  of those fixed points \cite{efk98}. 
  Moreover, another reason that solutions at other 
  singular points are not considered is because if $f(z)$ is a solution of 
  the equation $\sigma_qf(z) = A(z)f(z)$, with a singularity at 
  $z_0\neq 0,\infty$, then $f(z)$ has a singularity at any complex number 
  $q^kz$ \cite[Remark 2.7]{RW22}. \\\\
  So generally, given a $q$-difference system 
  $(\C(z),\sigma_q) \in \mathbf{DiffEq}(\C(z), \sigma_q)$, there are three 
  solution functors. There are two solution functors mapping to 
  solutions around $0$ and $\infty$, and one mapping to elliptic solutions,
  because solutions of the equation $\sigma_qf = f$ that are meromorphic
  at $0$ and $\infty$ must be a constant. We denote these solution functors by
  $$\on{Sol}^{(0)},\quad \on{Sol}^{(\infty)},\quad \on{Sol}^{(\on{ell})}.$$
  In practise, we only wish to consider the functors $\on{Sol}^{(0)}$
  and $\on{Sol}^{(\infty)}$.
  Given a $q$-difference equation, we wish to obtain a basis of solutions at 
  $0$ and $\infty$. The following result gives a way to check the linear 
  independence of these solutions:
  \begin{lemma}[Lemma 2.3.3, \cite{sau16}]
      Let $f_1,\cdots, f_n \in \mcal{M}(\C^\times)$. Then, the \emph{$q$-Wronskian}
      matrix is defined as:
      $$W_n(f_1,\cdots,f_n) := \begin{pmatrix}
          f_1 & f_2 & \cdots & f_n\\
          \sigma_qf_1 & \sigma_qf_2 & \cdots & \sigma_qf_n\\
          \vdots & \vdots & \ddots & \vdots\\
          \sigma_q^{n-1} f_1 & \sigma_q^{n-1}f_2 & \cdots & \sigma_q^{n-1}f_n
      \end{pmatrix}.$$
      Then, $f_1,\cdots,f_n$ are linearly dependent if and only if 
      $\det W_n(f_1,\cdots,f_n) = 0$ over $\mcal{M}(\C^\times)^{\sigma_q}$.
  \end{lemma}
  
  \begin{definition}[Definition 2.10, \cite{RW22}]
      A family of solutions $f_1,\cdots,f_n$ is a \emph{fundamental solution}
      if $\det W_n(f_1,\cdots,f_n) \neq 0$.
  \end{definition}
  When we are given a $q$-difference equation, our goal will be to find fundamental
  solutions in neighbourhoods of $0$ and $\infty$.
  We now introduce some special functions that can be used to solve $q$-difference equations (c.f. \cite[\S 1.2.2]{sauloy03}, \cite[\S 2]{RW22}). The \emph{Jacobi theta function} is defined as
  $$\Theta_q(z) = \sum_{n\in \mathbb{Z}} q^{-\frac{n(n+1)}{2}}z^n,$$
  which satisfies the $q$-difference relation $\sigma_q^n\Theta_q(z) = q^{\frac{1}{2}n(n+1)}z^n\Theta_q(z)$, and the famous \emph{Jacobi triple product identity}:
  $$\Theta_q(z) = (q^{-1};q^{-1})_\infty (-q^{-1}z;q^{-1})_\infty (-z^{-1};q^{-1})_\infty.$$ 
  \begin{remark}
      Another definition of the Jacobi theta function seen in the literature is given by:
      $$\Theta_q(u) = \sum_{n \in \mathbb{Z}} q^{\frac{n(n+1)}{2}} z^n,$$ 
      which satisfies the triple product identity 
      $$\Theta_q(u) = (q;q)_\infty (-z;q)_\infty (-qz^{-1};q)_\infty,$$
      and satisfies the $q$-difference relation $\Theta_q(q^nu) = q^{-\frac{1}{2}n(n+1)} u^{-n}\Theta_q(u)$.
  \end{remark}
  Note that $\Theta_q$ has the property that 
  $$\frac{\Theta_q(q^{-1}u^{-1})}{\Theta_q(u^{-1})} = \frac{\Theta_q(au)}{\Theta_q(u)}.$$
  For some $\lambda \in \C^\times$, the \emph{$q$-character} associated
  to divisor $\lambda$ is the function $e_{q,\lambda} \in \mcal{M}(\C^\times)$
  given by
  $$e_{q,\lambda}(z) = \frac{\Theta_q(z)}{\Theta_q(z/\lambda)},$$
  which satisfies the $q$-difference relation $\sigma_qe_{q,\lambda}(z) = \lambda \cdot e_q,\lambda(z)$. 
  As aforementioned, $q$-periodic functions are sections of the structure sheaf, or some localisations of the 
  structure sheaf. Let $\mcal{O}(\lambda)$ denote the line bundle over $E$ arising from a divisor $\lambda$. 
  It follows then that the ratio $e_{q,\lambda}$ of theta functions defines a section of the line bundle 
  $\mcal{O}(\lambda)$. Generally, any function with $q$-period $\lambda$ gives a section of $\mcal{O}(\lambda)$.\\\\
  The \emph{$q$-logarithm} is the function 
  $\ell_q \in \mcal{M}(\C^\times)$ defined by 
  $$\ell_q(z) = z \frac{\Theta_q'(z)}{\Theta_q(z)},$$
  which satisfies the $q$-difference relation $\ell_q(qz) = \ell(z) + 1$
  (\cite[Definition 2.5]{RW22}, \cite[\S 1.2.2]{sauloy03}).\\\\
  We can use this data to build a fundamental solution to $q$-difference 
  equations of the form $\sigma_qX = AX$, given that the system satisfies the 
  \emph{non-resonance} condition:
  \begin{definition}
      Let $\sigma_qX = AX$ be a $q$-difference system defined by the matrix 
      $A$. Let $\lbrace \lambda_i\rbrace_i$ be a collection of eigenvalues for 
      $A(0)$. Then, this $q$-difference system is said to be \emph{non-resonant}
      if if for every $i\neq j$, the ratio $\frac{\lambda_i}{\lambda_j} \not\in q^{\mathbb{Z}\setminus \lbrace0 \rbrace}$.
  \end{definition}
  
  Let $\mcal{M}_0$ and $\mcal{M}_\infty$ denote the stalks of $\mcal{M}$ at $0$ and $\infty$. 
  Given a non-resonant $q$-difference system, one can build a gauge transformation
  $M^{(0)} \in \on{GL}_n(\mcal{M}_0)$ (resp. $M^{(\infty)} \in \on{GL}_n(\mcal{M}_\infty)$) 
  sending the matrix $A(0)$ (resp. $A(\infty)$) to the constant matrix $A(z)$ (see \cite{sau16}). 
  Then, we take
  the Jordan-Chevalley decomposition of $A(0) = A_sA_u$, where $A_s$ is the 
  semisimple part, and $A_u$ is the unipotent part. Since 
  $A_s = Q\on{diag}(\lambda_1,\cdots,\lambda_n)Q^{-1}$, we can define:
  $$e_{q,A_s} := Q \cdot \begin{pmatrix}
      e_{q,\lambda_1} \\
      & \ddots\\
      & & e_{q,\lambda_n}
  \end{pmatrix} \cdot Q^{-1}.$$
  This satisfies the $q$-difference relation $\sigma_q e_{q,A_s} = A_s e_{q,A_s} = e_{q,A_s}A_s$.
  Identifying the $q$-characters as sections of a line bundle, we see that $e_{q,A_s}$ is a section of the 
  vector bundle over $E$: 
  $$\mcal{F} = \mcal{O}(\lambda_1) \oplus \cdots \oplus \mcal{O}(\lambda_n).$$
  For the nilpotent part, we first obeserve that since $A_u$ is unipotent,
  $N = A_u - I_n$ is nilpotent. Thus, we may define:
  $$A_u^{\ell_q} := \sum_{k\geq 0} \begin{pmatrix}
      \ell_q\\
      k
  \end{pmatrix} N^k,$$
  where $$\begin{pmatrix}
      \ell_q\\
      k 
  \end{pmatrix} := \frac{\ell_q(\ell_q-1)\cdots (\ell_q-(k-1))}{k!}.$$
  Then, we set $$e_{q,A_u} := A_u^{\ell_q}.$$
  Let us write $$e_{q,A(0)} := e_{q,A_s}\cdot e_{q,A_u}.$$
  One checks then that $e_{q,A(0)}$ is a solution to the constant coefficient $q$-difference system:
  $$\sigma_qX = A(0) X.$$
  On constructs $e_{q,A(\infty)}$ analogously.
  Then, the \emph{fundamental solution} around $0$ is given by: $$\mathbf{X}^{(0)} := M^{(0)} e_{q,A_s} \cdot e_{q,A_u}.$$ 
  One constructs $\mathbf{X}^{(\infty)}$ analogously.
  We have proven the following:
  \begin{theorem}[Theorem 3.3.1, \cite{sau16}]\label{thm_const_coeff_solns}
      The $q$-difference system $\sigma_qX = AX$ admits a basis of fundamental
      solutions at $0$ and $\infty$ given by:
      $$\mathbf{X}^{(0)} = M^{(0)}e_{q,A(0)},\quad \mathbf{X}^{(\infty)} = M^{(\infty)} e_{q,A(\infty)}.$$
  \end{theorem}
  
  
  \subsection{The Connection Data}
  
  We follow \cite[Chapter 12]{efk98} for this section. Consider a $q$-difference
  system given by:
  \begin{equation}\label{eqn_q_diff_system}
      \Phi(qz) = A(z) \Phi(z),
  \end{equation}
  where $\Phi$ is a column vector of meromorphic function. Then, we have the following
  result:
  \begin{theorem}[Theorem 12.1.1, \cite{efk98}]\label{thm_asymptotic_soln}
      Let $v_1,\cdots,v_n$ be an eigenbasis of $A(0)$, and 
      $u_1,\cdots,u_n$ be an eigenbasis of $A(\infty)$. Fix complex numbers
      $a_1,\cdots, a_n$, and $b_1,\cdots,b_n$ such that 
      $A(0) v_j = e^{ta_j}v_j$, and $u_j = e^{tb_j}u_j$. Then, there exists a 
      unique basis of solutions $\Phi_1^{(0)},\cdots, \Phi_n^{(0)}$, and 
      $\Phi_1^{(\infty)},\cdots,\Phi_n^{(\infty)}$ of \eqref{eqn_q_diff_system}
      of the form
      $$\Phi_j^{(0)}(z) = z^{a_j} \varphi_j(z),\quad \Phi_j^{(\infty)} = z^{b_j}\psi(z),$$
      where $\varphi_j$ are regular at $0$, and $\psi_j$ are regular at $\infty$,
      and $\varphi_j(0) = v_j$, and $\psi(\infty) = u_j$.
  \end{theorem}
  
  \begin{definition}
      Solutions of the form $\Phi_i^{(0)}$ and $\Phi^{(\infty)}$ seen in 
      Theorem \ref{thm_asymptotic_soln} will be called 
      \emph{asymptotic solutions} near $0$ and $\infty$, respectively.
  \end{definition}
  
  The \emph{connection matrix} is an elliptic matrix that relates the two asymptotic solutions at 
  $0$ and $\infty$ in the following way:
  
  \begin{definition}
      Let $\Phi(qz) = A(z)\Phi(z)$ be a $q$-difference system, with 
      asymptotic solutions $\Phi^{(0)}$ and $\Phi^{(\infty)}$ near 
      $0$ and $\infty$, respectively. Then, a \emph{connection matrix}
      matrix $\mathbf{X}$ is an element of $\on{GL}_n(\mcal{M}(\C^\times)$ 
      satisfying the relation:
      $$\Phi^{(0)} = \mathbf{X} \cdot \Phi^{(\infty)},$$
      and $\sigma_q \mathbf{X} = \mathbf{X}$.
  \end{definition}
  
  The \emph{Birkhoff connection matrix} of \cite{sauloy03} is defined to be: 
  $$\left(\mathbf{X}^{(\infty)}\right)^{-1}\cdot \mathbf{X}^{(0)},$$
  where $\mathbf{X}^{(0)}$ and $\mathbf{X}^{(\infty)}$ are the fundamental
  solutions. Since $\mathbf{X}$ is $q$-periodic, one easily checks that 
  $$\left(\mathbf{X}^{(\infty)}\right)^{-1}\cdot \mathbf{X}^{(0)} = \mathbf{X}^T.$$
  As aforementioned, one builds a meromorphic gauge transformation 
  $M^{(0)} \in \on{GL}_n(\mcal{M}(\C^\times)$ (resp. $M^{(\infty)}$) that takes $A(0)$ (resp. $A(\infty)$) to $A(z)$. It follows then that the matrix 
  $$M := \left(M^{(\infty)}\right)^{-1} \cdot M^{(0)},$$
  is a suitable meromorphic gauge transformation that takes $A(0)$ to $A(\infty)$. Using the formula 
  for the fundamental solutions from Theorem \ref{thm_const_coeff_solns}, one deduces that $M$ has the property 
  that $M$ gives a $q$-gauge transformation from $\mathbf{X}^{(0)}$ to $\mathbf{X}^{(\infty)}$:
  $$(\sigma_q M) \cdot  e_{q,A(0)} = e_{q,A(\infty)} \cdot M.$$
  We thus call this matrix $M$ the \emph{monodromy matrix} of a $q$-difference system. 
  Note that if we have an explicit formula for the connection matrix $\mathbf{X}$, then we can compute 
  the monodromy matrix by the equation:
  $$M = e_{q,A(\infty)} \cdot \mathbf{X}^T \cdot \left(e_{q,A(0)}\right)^{-1}.$$
  
  \subsection{The $q$-Riemann-Hilbert-Correspondence}
  
  As aforementioned in the introduction, the classical Knizhnik-Zamolodchikov equations of 
  \cite{kz84}, can be viewed as vector bundles over $\mathbb{P}^1$ equipped with a connection. 
  Using this viewpoint, one may apply algebraic geometric techniques to study its properties. 
  \cite{sauloy03} shows that something similar can be done for $q$-difference systems. \\\\
  Associated to any $q$-difference system we may associate a triple 
  $(e_{q,A(0)}, e_{q,A(\infty)}, M)$, which \cite{sauloy03} calls 
  the \emph{connection data}. A morphism between the connection data amounts to a 
  change of basis for $\mathbf{X}^{(0)}$, $\mathbf{X}^{(\infty)}$, and $M$ (see 
  \cite[\S 3]{sauloy03}). Thus, there is a category of connection data, which we 
  denote by $$\mcal{C}onn.$$
  \cite[Theorem 2.3.2.1]{sauloy03} shows that there is a functor:
  \begin{equation}
      (e_{q,A(0)}, e_{q,A(\infty)}, M) \longmapsto (\mcal{F}_0,\mcal{F}_\infty, \varphi),
  \end{equation}
  where $\mcal{F}_0$ and $\mcal{F}_\infty$ are locally free coherent sheaves (i.e. vector bundles)
  on an elliptic curve $E = \C^\times/q^{\mathbb{Z}}$, and 
  $\varphi : \mcal{F}_0 \dashrightarrow \mcal{F}_\infty$ is a meromorphic map between the vector bundles. We will use dotted arrows to denote meromorphic 
  maps from now on.
  Moreover, \cite[Theorem 2.3.2.1]{sauloy03} proves that this is in fact an equivalence of monoidal categories.
  The rank of $\mcal{F}_0$ and $\mcal{F}_\infty$ corresponds to 
  the amount of linearly independent asymptotic solutions at $0$ and $\infty$, respectively.
  Recall that there is an isomorphism of coherent sheaves:
  $$\on{\mathscr{H}om}_{\mcal{O}_E}(\mcal{F}_0,\mcal{F}_\infty) \cong \mcal{F}_0\otimes_{\mcal{O}_E}\mcal{F}_\infty^\vee.$$
  In the case where $\mcal{F}_0$ and $\mcal{F}_\infty$ decompose as a direct sum
  of line bundles on $E$, one may identify local sections of $\varphi$ with 
  the matrix $M$ in the connection data category of \cite{sauloy03}, thus 
  giving us a functor in the reverse direction.
  By abuse of notation, we will refer to the category of triples $(\mcal{F}_0,\mcal{F}_\infty,\varphi)$ as the connection category as well, and we will similarly 
  denote it by $\mcal{C}onn$. In particular, the functor 
  $$\qRH : \DiffEq(\C(z),\sigma_q) \longrightarrow \mcal{C}onn,$$
  gives a $q$ analogue of the Riemann-Hilbert correspondence. According, we will call it
  the $q$-Riemann-Hilbert functor, and denote it by $\qRH$.
  \begin{example}
      As an explicit example, if we have a $q$-difference system $\sigma_q X = A(u) X$, such that 
      $A(u)$ is semi-simple, then it follows that $A(0)$ and $A(\infty)$ are both diagonlisable.
      Let $\lambda_1,\cdots,\lambda_n$ be a set of eigenvalues for $A(0)$. Then, the matrix $e_{q,A(0)}$ 
      gives the following vector bundle on $E$: 
      $$\mcal{F}_0 := \mcal{O}(\lambda_1) \oplus \cdots \oplus \mcal{O}(\lambda_n).$$ 
      The procedure for $A(\infty)$ is similar, and we obtain a vector bundle $\mcal{F}_\infty$ 
      built from a direct sum of line bundles: 
      $$\mcal{F}_\infty \cong \mcal{O}(\mu_1) \oplus \cdots \oplus \mcal{O}(\mu_n).$$
      The monodromy map is then given by 
      $$\varphi : \mcal{F}_0 \dashrightarrow \mcal{F}_\infty.$$
      Using the isomorphism $\mathscr{H}om_{\mcal{O}_X}(\mcal{F},\mcal{G}) \cong \mcal{F}^\vee \otimes_{\mcal{O}_X} \mcal{G}$,
      we may identify the meromorphic map $\varphi$ as a section of $$\mcal{F}_0^\vee \otimes_{\mcal{O}_E} \mcal{F}_\infty \cong \mcal{O}(\lambda_1^{-1}\mu_1) \oplus \cdots \oplus \mcal{O}(\lambda_n^{-1}\mu_n).$$
      The monodromy matrix is then given by: 
      $$\varphi = \renewcommand\arraystretch{2.5}\begin{pmatrix}
          M_{11} \cdot \displaystyle\frac{\Theta_q(u)}{\Theta_q(\lambda_1\mu_1^{-1}u)} & \cdots & M_{1n} \cdot \displaystyle\frac{\Theta_q(u)}{\Theta_q(\lambda_n\mu_1^{-1}u)}\\ 
          \vdots & \ddots & \vdots\\ 
          M_{n1} \cdot \displaystyle\frac{\Theta_q(u)}{\Theta_q(\lambda_n\mu_1^{-1}u)} & \cdots & M_{nn} \cdot \displaystyle\frac{\Theta_q(u)}{\Theta_q(\lambda_n\mu_n^{-1}u)}
      \end{pmatrix},$$
      where $M_{ij}$ are $q$-periodic functions.
  \end{example}
  
  \section{The Double Affine Hecke Algebra (DAHA)}
  
  Fix a root system $(\mathbf{X},\Phi,\mathbf{Y},\Phi^\vee)$, where $\mathbf{X}$ and 
  $\mathbf{Y}$ are the weight and coweight lattices, respectively, and 
  $\Phi$ and $\Phi^\vee$ denotes the roots and coroots. Given $\Phi$, one may choose 
  (in many different ways) a set of positive roots $\Phi^+$. Then, the negative roots 
  $\Phi^-$ are given by elements of the form $-\alpha$, where $\alpha \in \Phi^+$.
  One may similarly define positive and negative weights (resp. coweights),
  which we denote by $\mathbf{X}^+$ (resp. $\mathbf{Y}^+$).
  A root is \emph{simple} if it cannot be written as a linear combination of 
  positive roots.
  The \emph{rank} $r$ of a root system is given by the number of simple roots. 
  Denote by  $\mathbf{Q}$ the root lattice, and $\mathbf{P}$ the coroot lattice. 
  They are sublattices of $\mathbf{X}$ and $\mathbf{Y}$, respectively. Let 
  $\C[\mathbf{X}]$ and $\C[\mathbf{Y}]$ denote the group algebras 
  of the weight and coweight lattices. We write their elements as 
  $X^\mu \in \C[\mathbf{X}]$, and $Y^{\lambda^\vee} \in \C[\mathbf{Y}]$. \\\\
  A weight $\lambda$ (resp. coweight $\lambda^\vee$) is \emph{minuscule} if $0 \leq \langle \lambda, \alpha^\vee\rangle \leq 1$
  for each positive root $\alpha \in \Phi^+$ \cite[Definition 2]{kir97}. From \cite[Lemma 2.2]{kir97}, we know 
  that the minuscule weights (resp. coweights) form a set of representatives for $\mathbf{X}/\mathbf{Q}$ (resp. $\mathbf{Y}/\mathbf{P}$).\\\\
  The ambient vector space that the 
  roots lie in is given by $V := \mathbb{R}^r$. Let us define a new vector 
  space $\widetilde{V} := V \oplus \mathbb{R}\delta$.
  Then, the \emph{affine root system} is then given by 
  $\widetilde{\Phi} := \Phi \times \mathbb{Z}\delta$. Then, the affine positive 
  roots are defined to be those of the form $\alpha + k\delta$, for 
  $\alpha \in \Phi^+$, and $k\geq 0$.
  A basis for $\widetilde{\Phi}^+$ is given by the simple roots 
  $\alpha_0 := -\theta + \delta, \alpha_1,\cdots,\alpha_r$, where 
  $\theta$ is the highest root (i.e. $\theta - \alpha \in \mathbb{Q}_+$ for all 
  $\alpha \in \Phi$).
  The affine reflections are defined to be $s_{\widetilde{\alpha}} : \widetilde{\lambda} \mapsto \widetilde{\lambda} - \langle \lambda,\alpha^\vee\rangle \widetilde{\alpha}$, where $\langle -,-\rangle$ is the perfect pairing between
  $\mathbf{X}$ and $\mathbf{Y}$. Denote the \emph{affine Weyl group} 
  $\widetilde{W}
  $as the group generated by affine simple reflections. Then, 
  \cite[Theorem 3.1]{kir97} shows that $\widetilde{W} \cong W \rtimes \mathbf{X}$,
  where $W$ acts on $\mathbf{X}$ in the usual way. We refer the reader 
  to \cite[\S 2, \S 3]{kir97} for more information on affine root systems.\\\\
  The action of $\mu^\vee \in \mathbf{Y}$ on $\widetilde{V}$ is given by:
  \begin{equation}\label{eqn_taumu}
      \tau(\mu^\vee) : \widetilde{\lambda} \longmapsto \widetilde{\lambda} - \langle \mu^\vee, \lambda \rangle \cdot \delta.
  \end{equation}
  We make the following definitions, following \cite[\S 3]{kir97}: 
  \begin{enumerate}
      \item $Y^{\lambda^\vee} = T_{\tau(\lambda^\vee)}$,
      \item if $\lambda^\vee = \mu^\vee - \nu^\vee$, with $\mu^\vee,\nu^\vee \in \mathbf{Y}^+$,
      then $Y^{\lambda^\vee} = Y^{\mu^\vee} (Y^{\nu^\vee})^{-1}$.
  \end{enumerate}
  \begin{theorem}[Theorem 3.7, \cite{kir97}]\label{thm_y_lattice}(ii)
      \leavevmode
      \begin{itemize}
          \item[(i)] $Y^{\lambda^\vee}$ is well-defined for all $\lambda$, and $Y^{\lambda^\vee} \cdot Y^{\mu^\vee} = Y^{\lambda^\vee + \mu^\vee}$,
          \item[(ii)] Let $\tau(\mu^\vee) = \pi_r s_{i_\ell}\cdots s_{i_1}$ be a reduced expression, and let 
          $\alpha^{(1)},\cdots,\alpha^{(\ell)}$ be the associated sequence of affine roots. 
          Then, $$Y^{\mu^\vee} = \pi_r T_{i_\ell}^{\varepsilon_\ell}\cdots T_{i_1}^{\varepsilon_1},$$
          where $\varepsilon_i = 1$ if the corresponding $\alpha^{(i)} = \alpha + k\delta$, for 
          $\alpha \in \Phi^+$, and $\varepsilon_i = -1$ otherwise. 
      \end{itemize}
  \end{theorem}
  Together, such elements generate the algebra of Laurent polynomials $\C[\mathbf{Y}]$ on the coweight lattice 
  $\mathbf{Y}$. $\C[\mathbf{X}]$ is constructed similarly.\\\\
  The \emph{finite Hecke algebra} $\mathbf{H}$ arises as a $q$-deformation of the 
  usual Weyl group. It is a $\C[q^\pm]$-algebra generated by elements of the form 
  $\lbrace T_w : w\in W \rbrace$ satisfying the quadratic relation:
  $$T_{s_\alpha}^2 = (q-1)T_{s_\alpha} + q,$$
  where $s_\alpha$ is a simple reflection about a simple root $\alpha$. 
  Let $t_\alpha \in \C^\times$ be a parameter for each $\alpha \in \Phi$,
  with the property that $t_\alpha = t_{w(\alpha)}$ for each 
  $w\in \widetilde{W}$.
  If $\alpha_i$ is a simple root, we write $t_i := t_{\alpha_i}$, and $T_i := T_{s_{\alpha_i}}$.
  Then, an alternative presentation of this algebra can be given as a 
  $\C[t_\alpha^\pm]$-algebra with quadratic relation 
  $$T_i^2 = (t_i-t_i^{-1})T_i + 1.$$
  Throughout this thesis, we will use the latter presentation. 
  From this, one expects to obtain the \emph{affine Hecke algebra} (AHA) $\mathbf{\dot{H}}$
  as a $q$-deformation of the affine Weyl group $\widetilde{W} = W \rtimes \mathbf{X}$. 
  Note that $$W \rtimes \mathbf{X} \cong W \rtimes \mathbf{Y},$$
  and thus deforming both of these groups should give rise to isomorphic AHAs. We will
  denote the resulting AHAs by $\mathbf{\dot{H}}^X$ and $\mathbf{\dot{H}}^Y$,
  respectively. Moreover, since there are two presentations for the affine Weyl group -- one in terms 
  of a lattice, and another in terms of the reflection about the affine simple root $s_0$ -- one 
  expects to obtain two different presentations of the AHA as well.
  \begin{remark}
      Some authors will refer to the \emph{affine Weyl group} as the semidirect product 
      $W \rtimes \mathbf{P}$, where $\mathbf{P}$ is the coroot lattice, and the
      \emph{extended Weyl group} as the semidirect product of the affine Weyl group and the 
      weight lattice $\mathbf{X}$. 
      Other authors, on the other hand, will refer to the extended Weyl group 
      simply as the affine Weyl group, and the distinction between the two affine
      Weyl groups is made relative to the reductive group which we are using to define the root
      system. This is because a simply-connected, split reductive group (e.g. $\on{SL}_n$), 
      the affine root system constructed from the group will have no zero-length roots. 
      Otherwise, for non-simply connected groups (e.g. $\on{GL}_n$), the affine root system will 
      contain roots of zero length.
  \end{remark}
  The two presentations of the affine Hecke algebra (AHA) are the \emph{Iwahori-Matsumoto presentation} (first introduced in \cite[Theorem 2.24]{im65})
  and the \emph{Bernstein presentation} (first introduced in \cite{lus83}). The Iwahori-Matsumoto presentation gives the AHA as an algebra
  generated by $T_0,\cdots,T_r$, where $T_0$ corresponds to the reflection about the affine root, 
  and the $T_1,\cdots,T_r$ correspond to reflections about the simple roots. Further, there is 
  an additional generator $\pi$ that corresponds to roots of length zero. The generator 
  $\pi$ acts on the $T_i$'s in the following way:
  $$\text{$\pi T_i \pi^{-1} = T_j$, if $\pi(\alpha_i)=\alpha_j$,}$$
  where all indices are taken modulo $r$. Let $\Omega$ be the set of all 
  roots of length zero.
  \begin{definition}[Iwahori-Matsumoto Presentation of AHA]
      The \emph{affine Hecke algebra} $\mathbf{\dot{H}}$ is a 
      $\C[q^\pm,t_\alpha^\pm]$-algebra generated by 
      the elements $\pi \in \Omega$, and the elements $T_0,T_1,\cdots,T_r$
      subject to the following relations:
      \begin{itemize}
          \item[(i)] $T_i^2 = (t_i-t_i^{-1}) T_i + 1$,
          \item[(ii)] For any $w, v \in \widetilde{W}$, 
              $T_w T_{v} = T_{wv}$ if $\ell(wv) = \ell(w) + \ell(v)$.
      \end{itemize}
  \end{definition}
  The Bernstein presentation gives the AHA as a product of the affine Hecke algebra (generated by 
  $T_1,\cdots,T_r$), and a lattice $X^\mu$, where $\mu \in \mathbf{X}$ the coweight lattice. Alternatively, one may also commute the generators $T_1,\cdots,T_r$ with elements of the weight lattice. 
  The $T_i$'s 
  commute in the usual way, and $T_i$ and $X^\mu$ obey the Bernstein relation:
  \begin{equation}\label{eqn_bernstein1}
      T_iX^{\mu} - X^{s_i(\mu)}T_i = (1-q)\frac{X^{s_i(\mu)} - X^{\mu}}{1-X^{-\alpha_i}}.
  \end{equation}
  In terms of the parameters $t_i$, the Bernstein relation is written:
  \begin{equation}\label{eqn_bernstein2}
      T_iX^{\mu} - X^{s_i(\mu)}T_i = (t_i-t_i^{-1})\frac{X^{s_i(\mu)} - X^{\mu}}{1-X^{-\alpha_i}}.
  \end{equation}
  Then, we have the \emph{Bernstein presentation} of the AHA:
  \begin{definition}[Bernstein Presentation of AHA]
      The \emph{affine Hecke algebra} is the $\C[q^\pm, t_\alpha^\pm]$-algebra 
      (resp. $\C[q^\pm]$-algebra)
      generated by 
      $X^{\mu}$, for $\mu \in \mathbf{X}$, and $T_1,\cdots,T_r$ 
      subject to the following relations:
      \begin{itemize}
          \item[(i)] $T_i^2 = (t_i-t_i^{-1})T_i + 1$,
          \item[(ii)] The Bernstein relations \eqref{eqn_bernstein2} (resp. \eqref{eqn_bernstein1}).
      \end{itemize}
  \end{definition}
  Now, one can define another affine Hecke algebra relative to the dual affine root system 
  $\Phi^\vee$, from which one similarly affines $T_i$-terms, and elements $X^\mu$ for $\mu \in X$. 
  The $T_i$'s commute in the usual way, and commutes with $X^\mu$ according to the aforementioned
  Bernstein relation. \\\\
  The double affine Hecke algebra can be defined as an algebra generated by $T_i$, $X^\mu$, and 
  $Y^{\mu^\vee}$. Then, the commutation relations between $X^\mu$ and $Y^\lambda$ will be given by
  taking the product $X^\mu Y^\lambda$ and applying the Bernstein relation iteratively according to
  Theorem \ref{thm_y_lattice}(ii). The resulting relations, however, turn out to be quite complicated. So, often it is best to choose an affine Hecke algebra ---
  either $\mathbf{\dot{H}}^{\mathbf{X}}$ or $\mathbf{\dot{H}}^{\mathbf{Y}}$ ---
  and then take the Iwahori-Matsumoto presentation of it. This gives us 
  generators $\pi, T_0,T_1,\cdots, T_r$, which commute with the the elements
  in the other lattice via the Bernstein presentation. 
  \begin{definition}[Definition 4.1, \cite{kir97}]
      The \emph{double affine Hecke algebra} (DAHA) is the $\C[q^\pm,t_\alpha^\pm]$-algebra generated by the elements $\pi\in \Omega$, $T_0,\cdots,T_r$, and 
      $X^\mu$, where $\mu \in \mathbf{X}$ subject to the following relations:
      \begin{itemize}
          \item[(i)] The Iwahori-Matsumoto relations for $\pi, T_0,\cdots,T_r$,
          \item[(ii)] $X^\mu \cdot X^\nu = X^{\mu + \nu}$,
          \item[(iii)] 
              $$T_i X^\mu = \begin{cases}
                  X^\mu T_i \quad &\text{if} \quad \langle \mu,\alpha_i^\vee\rangle = 0,\\
                  X^{s_i(\mu)}T_i + (t_i - t_i^{-1}) X^\mu \quad &\text{if} \quad \langle \mu,\alpha_i^\vee \rangle = 1,
              \end{cases}$$
          \item[(iv)] $\pi X^\mu \pi^{-1} = X^{\pi(\mu)}$.
      \end{itemize}
  \end{definition}
  Following \cite[pg. 274]{kir97}, we embed the element in $X^\delta$ in the AHA (and 
  hence in the DAHA) in the following way: 
  \begin{equation}\label{eqn_xdelta}
      X^{\delta} = q^{-2}.
  \end{equation}
  Substituting this into \eqref{eqn_dl}, the generator $T_0$ corresponding to 
  the affine simple reflection $s_{\alpha_0}$ becomes:
  $$T_0X^\mu - X^{\mu+\theta}q^2T_0 = (t_0-t_0^{-1}) X^\mu,\quad \langle \mu,\theta^\vee\rangle = 1.$$
  We now record some useful properties of the DAHA that we will
  use:
  \begin{theorem}[Theorem 4.2, \cite{kir97}]\label{thm_pbw_thm}
      Every element $h\in \mathbf{\ddot{\mathbf{H}}}$ can be written uniquely in the form:
      $$\displaystyle\sum_{\substack{\mu \in \mathbf{X}\\ \lambda \in \mathbf{Y}\\w \in W}} a_{\mu\lambda w} X^\mu Y^{\lambda^\vee} T_w, \quad a_{\mu\lambda w} \in \C[q^\pm,t_\alpha^\pm].$$
  \end{theorem}
  Equivalently, Theorem \ref{thm_pbw_thm} states that there is a 
  triangular decomposition of $\mathbf{\ddot{\mathbf{H}}}$:
  \begin{equation}\label{eqn_tri_decomp}
      \mathbf{\ddot{\mathbf{H}}} \cong \C[\mathbf{X}] \otimes_\C \mathbf{H} \otimes_\C \C[\mathbf{Y}].
  \end{equation}
  Note that the triangular decomposition is an isomorphism of $\C$-vector spaces, 
  \emph{not} algebras.
  Moreover, we also have $\C$-vector isomorphisms 
  $$\C[\mathbf{X}] \otimes_\C \mathbf{H} \cong \mathbf{H} \otimes_\C \C[\mathbf{X}],\quad \C[\mathbf{Y}] \otimes_\C \mathbf{H} \cong \mathbf{H} \otimes_\C \C[\mathbf{Y}],$$
  by the Bernstein presentation of AHA. 
  \begin{theorem}[Theorem 4.3, \cite{kir97}]\label{thm_demazure_lusztig}
      The following formulas give a faithful representation of $\mathbf{\ddot{\mathbf{H}}}$
      in $\C[q^\pm,t_\alpha^\pm][X^\pm]$: 
      $$X^\mu \longmapsto X^\mu,$$
      $$\pi : X^\mu \longmapsto X^{\pi(\mu)},\quad \pi\in\Omega,$$
      \begin{equation}\label{eqn_dl}
          T_i \longmapsto t_is_i + (t_i-t_i^{-1})\frac{s_i-1}{X^{-\alpha_i}-1},\quad i=0,\cdots,n.
      \end{equation}
  \end{theorem}
  The formula \eqref{eqn_dl} is called the
  \emph{Demazure-Lusztig operator} (c.f. \cite[(6.1)]{beg00}). 
  Given the group algebra $\C[\mathbf{X}]$, one may construct the algebraic
  torus 
  $$T = \spec \C[\mathbf{X}],$$
  which as a group scheme is isomorphic to 
  $\mathbb{G}_{\on{m}} \otimes_{\mathbb{Z}} \mathbf{X}$. If the root system
  has rank $r$, then $T \cong \mathbb{G}_{\on{m}}^r$. One may identify $T$
  as the maximal torus of a suitable split, reductive algebraic group scheme.\\\\
  Moreover, one may identify $\C[t_\alpha^\pm]$ as the $\C$-rational 
  points of the group
  scheme $T$, and $\C[q^\pm]$ as the $\C$-rational points of the group scheme
  $\mathbb{G}_{\on{m}}$. In this case, the torus $T \times \mathbb{G}_{\on{m}}$
  is the torus corresponding to the affine root system $\widetilde{\Phi}$.
  In particular, it is the maximal torus of a split, $p$-adic
  reductive group (see \cite{cai22}, \cite{hkp10}).\\\\
  Then, with this formalism, one may view 
  the Demazure-Lusztig operator as a rational function in 
  $\C[T \times \mathbb{G}_{\on{m}}]$. Moreover, one may view the elements 
  $X^\mu$ as functions over $T\times \mathbb{G}_{\on{m}}$ as well. 
  This formalism will become more useful
  when we consider the quantum torus in Chapter \ref{chap3}. Alternatively,
  the interested reader can also consult \cite[\S 5, \S6]{beg00}.\\\\
  Recall from \eqref{eqn_taumu} that operator $\tau(\mu^\vee) : \mathbf{X} \to \mathbf{X}$.
  One can extend this to a map on $\C[q^\pm,t_\alpha^\pm][\mathbf{X}] \to \C[q^\pm,t_\alpha^\pm][\mathbf{X}]$.
  Using the identification $X^\delta = q^{-2}$ in \eqref{eqn_xdelta} , we obtain 
  a \emph{$q$-shift operator} given by 
  \begin{equation}\label{eqn_shift_operator}
      \tau(\mu^\vee) : \C[q^\pm,t_\alpha^\pm][\mathbf{X}] \longrightarrow \C[q^\pm,t_\alpha^\pm][\mathbf{X}], \quad X^\lambda \longmapsto q^{2\langle \mu^\vee,\lambda\rangle} X^\lambda,
  \end{equation}
  for some weight $\lambda \in \mathbf{X}$.
  Recall that in terms of the Iwahori-Matsumoto presentation, we can write the lattice term $Y^{\mu^\vee} = T_{\tau(\mu^\vee)}$
  where $\tau(\mu^\vee)$ acts on the affine weights by the formula:
  $\tau(\mu^\vee) : \lambda \mapsto \lambda - \langle \alpha^\vee, \lambda\rangle\delta$. 
  We may consider meromorphic functions on the algebraic torus $T = \spec \C[\mathbf{X}]$, which we will 
  denote by $\C(T)$. By abuse of notation, we will also denote this by $\C(\mathbf{X})$ to emphasise that we 
  are considering meromorphic functions coming from the group scheme $\spec \C[\mathbf{X}]$.
  The action of the 
  $q$-shift operator is preserved under localisation. Moreover, $\C(\mathbf{X})$ 
  can be equipped with a natural $W$-action, and thus one may form the \emph{twisted
  tensor product}, which we denote as 
  $$\C(\mathbf{X}) \rtimes \C[W].$$
  We elaborate more on this in Chapter \ref{chap3}. But this object is what is known as the 
  \emph{quantum torus}, and its construction is due to \cite{beg00}. The action of the 
  $q$-shift operator naturally extends to elements in the quantum torus as well.
  These are examples of $q$-difference equations of multiple variables. As aforementioned, 
  we wish to study $q$-difference equations of one variable, and we must therefore 
  restrict ourselves to the rank one case. \\\\ 
  In this case, there is only one fundamental weight and so $\C(\mathbf{X}) = \C(X)$ -- 
  the algebra of meromorphic functions in one variable $X$.
  As such, one may form the category 
  $$\DiffEq(\C(X) \rtimes \C[W], \tau(\mu^\vee)),$$
  of $q$-difference equations in the sense of \cite{sauloy03}. 
  We will explore this connection between the DAHA and $q$-difference equations 
  in more detail in the later chapters.
  
  \section{The Elliptic Affine Hecke Algebra (EllAHA)}\label{chap_ellAHA}
  
  In this section, we briefly go through the construction of the elliptic affine 
  Hecke algebra, which was originally introduced in \cite{gkv95}. 
  We follow \cite[\S 4]{ZZ21} and 
  \cite[\S 4]{gkv95} for this construction.
  Let $T = \on{Spec}(\C[\mathbf{X}])$ be an 
  algebraic torus, and let $\mathbb{X}^\ast(T)$ be the character lattice,
  and $\mathbb{X}_\ast(T)$ be the co-character lattice. We have $\mathbb{Z}$-module
  isomorphisms $\mathbf{X} \cong \mathbb{X}^\ast(T)$ and $\mathbf{Y} \cong \mathbb{X}_\ast(T)$.\\\\
  Fix an abelian variety $\mathfrak{A} := E \otimes_{\mathbb{Z}} \mathbb{X}^\ast(T)$, where $\mathbb{X}^\ast(T)$ is 
  the weight lattice, and $E$ is an elliptic curve. 
  Note that $E \otimes_{\mathbb{Z}} \mathbb{X}^\ast(T) \cong E^r$, where $r$ is the rank of the root system.
  Again, let $\Phi$ be the finite subset of 
  $X^\ast$ defining a root system. Let us fix an element $\hbar \in E$, and let 
  $$\chi_\alpha : E\otimes_{\mathbb{Z}} \mathbb{X}^\ast(T) \longrightarrow E,\quad \hbar \otimes \mu^\vee \longrightarrow \hbar^{\langle\mu^\vee,\alpha\rangle}.$$
  Then, associated to each $\alpha \in \Phi$, let $$T_\alpha := \ker \chi_\alpha,$$
  be the kernel divisor of the map $\chi_\alpha$, and let 
  $$T_{\alpha,\hbar} := \ker (\chi_\alpha - \hbar),$$
  for some $\hbar \in \mathfrak{A}$.
  The Weyl group acts on the kernel divisor by $wT_\alpha = T_{w^{-1}\alpha}$, and
  $T_{-\alpha} = T_\alpha$. One thinks of these divisors $T_\alpha$ as root 
  hyperplanes.\\\\
  The action of $W$ on $\mathbb{X}^\ast(T)$ naturally extends to an action on 
  $\mathfrak{A}$. Thus, let $\mathfrak{A}/W$ be the set of its $W$-orbits, 
  which is is defined to be the affine GIT quotient
  $\mathfrak{A}/W = \spec (\mcal{O}_\mathfrak{A}(\mathfrak{A})^W)$.
  Let $\pi : \mathfrak{A} \to \mathfrak{A}/W$ be the natural map.\\\\
  The sheaf $\pi_\ast\mcal{O}_{\mathfrak{A}}$ is 
  equipped with a natural action of $W$. Let $$\mcal{O}[W] := \pi_\ast\mcal{O}_{\mathfrak{A}} \rtimes \C[W],$$ where $\rtimes$ denotes the 
  twisted tensor product of $\pi_\ast\mcal{O}_{\mathfrak{A}}$ and $\C[W]$. 
  The algebra structure is given by
  $$(f\otimes w)(g \otimes v) = (f \cdot {}^wg)\otimes (w\cdot v),$$
  where ${}^wg$ is the image of $g$ under the map 
  $$w^{-1} : \pi_\ast\mcal{O}_{\mathfrak{A}}(U) \longrightarrow \pi_\ast\mcal{O}_{\mathfrak{A}}(w^{-1}U).$$
  This equips $\mcal{O}[W]$ with the structure of a sheaf of associative algebras,
  Observe that by our construction, $\mcal{S}$ 
  is a sheaf of modules over $\mcal{S}_W$. On any open subset, let us denote 
  sections of $\mcal{S}_W$ coming from $w\in W$ by $\delta_w$. For any root 
  $\alpha \in \Phi$, we will simply write $\delta_\alpha := \delta_{s_\alpha}$.\\\\
  Now, let 
  $$\mathfrak{A}^{\on{reg}} := \mathfrak{A} \setminus \left(\bigcup_{\alpha\in\Phi} T_\alpha\right),$$
  so that $W$ acts on $\mathfrak{A}^{\on{reg}}$ freely. Indeed, a simple reflection $s_\alpha$ acts by 
  $s_\alpha T_\alpha = T_{s_\alpha(\alpha)} = T_{-\alpha} = T_\alpha$, and thus the stabiliser is non-trivial 
  on each divisor $T_\alpha$. Removing these root hyperplanes from $\mathfrak{A}$ thus allows $W$ to act freely.
  Let $$j : \mathfrak{A}^{\on{reg}}/W \longrightarrow \mathfrak{A}/W,$$
  denote the inclusion. The action of $\mcal{O}[W]$ on $\pi_\ast\mcal{O}_{\mathfrak{A}}$ induces a morphism 
  $$\rho : \mcal{O}[W]\vert_{\mathfrak{A}^{\on{reg}}} \longrightarrow \on{\mathscr{E}nd}(\pi_\ast\mcal{O}_{\mathfrak{A}}\vert_{\mathfrak{A}^{\on{reg}}}).$$
  Pushing forward by $j$, we get a map:
  $$j_\ast \rho : j_\ast\left( \mcal{O}[W]\vert_{\mathfrak{A}^{\on{reg}}}\right) \longrightarrow j_\ast\left(\on{\mathscr{E}nd}(\pi_\ast\mcal{O}_{\mathfrak{A}}\vert_{\mathfrak{A}^{\on{reg}}}\right).$$
  Let $$p : \on{\mathscr{E}nd}(\pi_\ast\mcal{O}_{\mathfrak{A}}) \longrightarrow j_\ast\on{\mathscr{E}nd}(\pi_\ast\mcal{O}_{\mathfrak{A}}\vert_{\mathfrak{A}^{\on{reg}}}),$$
  be the canonical map.
  \begin{definition}\label{def_ellaha}
      Let $f_w$ be local sections of $j_\ast(\pi_\ast\mcal{O}_{\mathfrak{A}}\vert_{\mathfrak{A}^{\on{reg}}})$. Then, following our notation from before, 
      we will write sections of $j_\ast(\mcal{O}[W]\vert_{\mathfrak{A}^{\on{reg}}})$
      as $\sum_{w\in W} f_w\delta_w$.
      The \emph{elliptic affine Hecke algebra} $\mcal{H}^{\on{ell}}$ is 
      the quasi-coherent subsheaf of 
      $$(j_\ast\rho)^{-1}\left(p\left(\on{\mathscr{E}nd}(\pi_\ast\mcal{O}_{\mathfrak{A}}\vert_{\mathfrak{A}^{\on{reg}}}\right)\right),$$
      whose local sections satisfy:
      \begin{itemize}
          \item[(a)] each $f_w$ has a pole of order at most one along $T_\alpha$. 
          \item[(b)] $\on{Res}_{T_\alpha}(f_w) + \on{Res}_{T_\alpha}(f_{s_\alpha w}) = 0$.
          \item[(c)] for any $\alpha \in \Phi(w) := \Phi^+ \cap w^{-1}\Phi^-$, $f_w$ vanishes on $T_{\alpha,\hbar}$
          for some $\hbar \in \mathfrak{A}$.
      \end{itemize}
  \end{definition}
  
  \begin{theorem}[Proposition 4.3, \cite{gkv95}]
      $\mcal{H}^{\on{ell}}$ is a subsheaf of algebras in $j_\ast(\pi_\ast\mcal{O}_{\mathfrak{A}}\vert_{\mathfrak{A}^{\on{reg}}} \rtimes \C[W])$. 
  \end{theorem}
  
  
  \section{The Module Category $\coh(\mcal{H}^{\on{ell}})$}
  
  Consider the object 
  $$\mcal{H}^{\on{ell}} \in \coh(\mathfrak{A}/W).$$
  Over this category, the rational sections of  $\mcal{H}^{\on{ell}}$ can be equipped with the 
  structure of an algebra. This is only possible after taking the pushforward into $\coh_W(\mathfrak{A}/W)$
  (see \cite[\S 4]{ZZ21}).
  Over the category $\coh_W(\mathfrak{A})$, a convolution construction is needed in order to define this 
  algebra structure (see \cite[\S 5]{ZZ21}, \cite[\S 2]{ZZ24}, Appendix \ref{app_C}).
  \begin{definition}
      A \emph{$\mcal{H}^{\on{ell}}$-module} $\mcal{M}$ is an object in
      $\coh(\mathfrak{A}/W)$, together with a multiplication map for which there exists a map:
      $$\mcal{H}^{\on{ell}} \otimes_{\mcal{O}_{\mathfrak{A}/W}} \mcal{M} \longrightarrow \mcal{M},$$
      defined by multiplication, such that for each open set 
      $U \subseteq \mathfrak{A}/W$, each regular sections $\mcal{M}(U)$ has the structure of a $\mcal{H}^{\on{ell}}(U)$-module.
  %    $m,n  \in \mcal{M}(U)$, 
  %    \begin{itemize}
  %        \item[(i)] $f_w \cdot (m+n) = f_w \cdot m + f_w \cdot n$, 
  %        \item[(ii)] $(f_w + g_v)m = f_w \cdot m + g_v \cdot m$,
  %        \item[(iii)] $(f_w \cdot g_v) \cdot m = f_w \cdot (g_v \cdot m)$,
  %        \item[(iv)] $1 \cdot m = m$.
  %    \end{itemize}
  \end{definition}
  Let $$\coh(\mcal{H}^{\on{ell}}),$$ 
  denote the category of $\mcal{H}^{\on{ell}}$-modules. Note that there is a subsheaf of algebras 
  $\pi_\ast\mcal{O}_{\mathfrak{A}} \subseteq \mcal{H}^{\on{ell}}$. Given some $\mcal{M} \in \coh(\mathfrak{A}/W)$,
  and a module action map $$\pi_\ast\mcal{O}_{\mathfrak{A}} \otimes_{\mathfrak{A}/W} \mcal{M} \longrightarrow \mcal{M},$$ 
  there exists some sheaf $\widetilde{\mcal{M}} \in \coh(\mathfrak{A})$ such that $$\mcal{M} \cong \pi_\ast\widetilde{\mcal{M}}.$$
  We say that $\widetilde{\mcal{M}}$ is the \emph{underlying sheaf} of the $\pi_\ast\mcal{O}_{\mathfrak{A}}$-module $\mcal{M}$.
  By abuse of notation, we may consider $\mcal{M}$ as a coherent sheaf on $\mathfrak{A}$.
  Let $$\coh^{\on{fin}}(\mcal{H}^{\on{ell}}),$$
  be the full subcategory of $\coh(\mcal{H}^{\on{ell}})$ whose underlying
  coherent sheaf has zero-dimensional support in $\mathfrak{A}$, and let 
  $$\coh^{\on{flat}}(\mcal{H}^{\on{ell}}),$$
  be the full subcategory of $\coh(\mcal{H}^{\on{ell}})$ whose underlying coherent
  sheaf is a homogeneous vector bundle (i.e. locally free, coherent sheaf)
  on $\mathfrak{A}$. Homogeneous means that $\mcal{M}$ is translation-invariant. That is,
  given left and right translation maps $L_g, R_g : \mathfrak{A} \to \mathfrak{A}$, we have isomorphisms
  $L_g^\ast \mcal{M} \cong \mcal{M} \cong R_g^\ast \mcal{M}$.
  
  \chapter{Category $\mcal{O}$ of the Double Affine Hecke Algebra}\label{chap2}
  
  In this chapter, we study a category $\mcal{O}_{\mathbf{\ddot{\mathbf{H}}}}$ of 
  $\mathbf{\ddot{\mathbf{H}}}$-modules called the \emph{category $\mcal{O}$} of DAHA. 
  This is a full subcategory of $\mathbf{\ddot{\mathbf{H}}}\lmod$ containing certain
  well-behaved $\mathbf{\ddot{\mathbf{H}}}$-modules that we would like to consider.
  The definition of category $\mcal{O}$ we use is due to \cite{che03}, and 
  \cite{jv19}. In particular, we use the definition of \cite{jv19}, who defined 
  $\mcal{O}_{\mathbf{\ddot{\mathbf{H}}}}$ for the type $A$ case. \cite{che03} defines this 
  category in more generality.
  The definition is similar to the one given in \cite{ggor03}
  for the rational DAHA, except the local nilpotency condition is replaced
  with a locally-finiteness condition. \\\\
  The main result of this chapter is Proposition \ref{prop_filtration}, which 
  shows that every object in $\mcal{O}_{\mathbf{\ddot{\mathbf{H}}}}$ admits a particular 
  filtration known as a \emph{$\Delta$-filtration}. 
  Note that Proposition \ref{prop_filtration} is a modification of 
  \cite[Proposition 2.2]{ggor03} for the locally finite case. \\\\
  Further, we also weaken \cite{ggor03}'s definition of a $\Delta$-filtration --
  in particular, our $\Delta$-filtrations can be of infinite length, and our 
  composition
  factors do not have to be precisely isomorphic to DAHA modules induced from 
  irreducible $\mathbf{\ddot{\mathbf{H}}}^{\mathbf{Y}}$-modules (c.f. \cite[Definition 2.3.3]{ggor03}). \\\\
  However, in the next chapter we will construct the \emph{localised} DAHA --
  denoted as $\mathbf{\ddot{\mathbf{H}}}_{\on{loc}}$. In this case, the category 
  $\mcal{O}_{\mathbf{\ddot{\mathbf{H}}}}$ admits $\Delta$-filtrations in 
  the sense of \cite{ggor03}.
  That is, $\Delta$-filtrations over the localised category 
  $\mcal{O}_{\mathbf{\ddot{\mathbf{H}}}}$ all have finite length, and its composition 
  factors are precisely isomorphic to DAHA modules induced from irreducible 
  $\mathbf{\ddot{\mathbf{H}}}^{\mathbf{Y}}$-modules (see Lemma \ref{lem_finite_filt}).
  
  \section{Locally Finite Modules}
  We record the following definition of the category $\mcal{O}$ of 
  DAHA from \cite{jv19}:
  \begin{definition}[Definition 5.1, \cite{jv19}]
      The \emph{category $\mcal{O}$} for DAHA --- denoted $\mcal{O}_{\mathbf{\ddot{\mathbf{H}}}}$ --- is the full subcategory of finitely-generated 
      $\mathbf{\ddot{\mathbf{H}}}$-modules such that for each finitely-generated module 
      $M$, and $m\in M$, the its orbit $\mathbf{Y}\cdot m$ is finite-dimensional.
      We say that such a module $M$ is \emph{$\C[\mathbf{Y}]$-locally finite}.
  \end{definition}
  
  This definition of category $\mcal{O}$ is similar to the one given by Cherednik
  in \cite[\S 6]{che03}. Further, Cherednik showed that 
  the full subcategory of finite-generated $\C[\mathbf{X}]$ or 
  $\C[\mathbf{Y}]$-locally finite $\mathbf{\ddot{\mathbf{H}}}$-modules gives rise 
  to the same category. For our purposes, we will work with 
  $\C[\mathbf{Y}]$-locally finite modules.\\\\
  Let $T^\vee = \spec \C[\mathbf{Y}]$ be an algebraic torus. As a group scheme,
  it is isomorphic to $\mathbb{G}_{\on{m}}\otimes_{\mathbb{Z}} \mathbf{Y}$.
  $T^\vee$ is the dual torus to 
  $T = \spec \C[\mathbf{X}] \cong \mathbb{G}_{\on{m}} \otimes_{\mathbb{Z}} \mathbf{X}$.
  By construction, $T$ 
  is equipped with a Weyl group action. 
  We follow \cite[\S 5]{jv19} for the definitions below.
  Let $M$ be any $\mathbf{\ddot{\mathbf{H}}}$-module. Then, for any $\lambda \in T$, 
  its \emph{$\C[\mathbf{Y}]$-weight space} is defined to be 
  $$M_\lambda := \lbrace v \in M : \text{$Y^{\mu^\vee} v = \lambda(Y^{\mu^\vee})v$, for $Y^{\mu^\vee} \in \C[\mathbf{Y}]$}\rbrace.$$ 
  We call a non-zero vector $v\in M_\lambda$ a \emph{weight vector}. 
  Its generalised weight space is given by:
  $$M_\lambda^{\on{gen}} := \lbrace v\in M : \text{$(Y^{\mu^\vee}-\lambda(Y^{\mu^\vee}))^mv = 0$ for all $Y^{\mu^\vee} \in \C[\mathbf{Y}]$, $m\geq 0$}\rbrace.$$
  Since we are working over $\C$, any  $M \in \mcal{O}_{\mathbf{\ddot{\mathbf{H}}}}$
  splits over $\C$, and thus there is a $\C$-vector space isomorphism:
  $$M \cong \bigoplus_{\lambda \in T}  M_\lambda^{\on{gen}}.$$
  This is, in fact, an isomorphism of $\C[\mathbf{Y}]$-modules as well. 
  Let $[\lambda] \in T/W$. Then, we may construct the weight space 
  $M_{[\lambda]}^{\on{gen}}$, which obtains the structure of a 
  $\mathbf{\ddot{\mathbf{H}}}$-module. Then, for any $\ddot{\mathbf{H}}$-module $M$,
  we have a decomposition of $\mathbf{\ddot{\mathbf{H}}}$-modules:
  $$M \cong \bigoplus_{[\lambda]\in T/W} M_{[\lambda]}^{\on{gen}}.$$
  %We call $\mathbf{\ddot{\mathbf{H}}}$-modules admitting a $\C[\mathbf{Y}]$-module
  %eigendecomposition \emph{$\C[\mathbf{Y}]$-semisimple}, following the 
  %terminology of \cite{jv19}. 
  %As a byproduct of this, we see that $\mcal{O}_{\mathbf{\ddot{\mathbf{H}}}}$ has a decomposition
  %into blocks:
  %$$\mcal{O}_{\mathbf{\ddot{\mathbf{H}}}} = \bigoplus_{[\lambda] \in T/W} \mcal{O}_{[\lambda]},$$
  %where $M \in \mcal{O}_{[\lambda]}$ if the action of $\C[\mathbf{Y}]$ on $M$
  %has eigenvalues that lie in the orbit $W\cdot \lambda$.
  %If $[\lambda]$ lies in a root hyperplane $T_\alpha$ of $T$, then we call its
  %corresponding block a \emph{singular block}. Otherwise, it is a 
  %\emph{regular block}.
  
  \section{Standard Modules}
  
  Here, we construct an appropriate filtration of $\mathbf{\ddot{\mathbf{H}}}$-representations, 
  similar to the ones seen in 
  \cite[\S 2.3]{ggor03} for the rational DAHA, but adapted to the DAHA case. \\\\
  For each eigenvalue $\lambda \in T$ of some element in $\C[\mathbf{Y}]$, the 
  ideal $(Y^{\mu^\vee} - \lambda)$ is maximal in $\C[\mathbf{Y}]$. 
  Denote this maximal ideal by $\mathfrak{m}_\lambda := (Y^{\mu^\vee}-\lambda)$,
  and let us consider the weight space $\C[\mathbf{Y}]/\mathfrak{m}_\lambda$.
  Moreover, the generalised weight space $\C[\mathbf{Y}]/\mathfrak{m}_\lambda^k$ 
  is finite-dimensional, and has the structure of a $\C[\mathbf{Y}]$-module. 
  The \emph{standard module} is the $\mathbf{\dot{H}}^{\mathbf{Y}}$-module
  given by:
  $$\delta_\lambda := \on{Ind}_{\C[\mathbf{Y}]}^{\mathbf{\dot{H}}^{\mathbf{Y}}} \left(\C[\mathbf{Y}]/\mathfrak{m}_\lambda\right).$$
  
  \begin{remark}
      Note that our definition of standard module here is a modification of
      one used in \cite{ggor03}. 
  \end{remark}
  
  \begin{lemma}\label{lem_quot}
      Each irreducible $\mathbf{\dot{H}}^{\mathbf{Y}}$-module $V$ is a quotient 
      of a standard module $\delta_\lambda$ for some $\lambda$.
  \end{lemma}
  
  \begin{proof}
      Let $V$ be an irreducible $\mathbf{\dot{H}}^{\mathbf{Y}}$-module.
      Then, by Frobenius reciprocity, there is an isomorphism
      $$\on{Hom}_{\mathbf{\dot{H}}^{\mathbf{Y}}}(\delta_\lambda, V) \cong \on{Hom}_{\C[\mathbf{Y}]}(\C[\mathbf{Y}]/\mathfrak{m}_\lambda,V).$$
      As a $\C[\mathbf{Y}]$-module, $V$ admits a weight space 
      decomposition $$V \cong \bigoplus_{\lambda \in T} V_\lambda,$$
      and thus there exists a non-zero map 
      $\C[\mathbf{X}]/\mathfrak{m}_\lambda \to V$. It follows therefore that 
      there must exist non-trivial maps on the left-hand side. Moreover, by the 
      irreducibility of $V$ as a $\mathbf{\ddot{\mathbf{H}}}^{\mathbf{Y}}$-module, all 
      such $\mathbf{\ddot{\mathbf{H}}}^{\mathbf{Y}}$-homeomorphisms $\delta_\lambda \to V$
      must be surjective, and thus 
      $V$ arises as some quotient of $\delta_\lambda$.
  
  \end{proof}
  
  Induce the standard module to the DAHA,
  $$\Delta(\delta_\lambda) := \on{Ind}_{\mathbf{\dot{H}}^{\mathbf{Y}}}^{\mathbf{\ddot{\mathbf{H}}}} \delta_\lambda,$$
  and note the following property:
  
  \begin{lemma}\label{lem_delta_exact}
      The functor 
      $\Delta = \on{Ind}_{\mathbf{\dot{H}}^{\mathbf{Y}}}^{\mathbf{\ddot{\mathbf{H}}}}$ 
      is exact.
  \end{lemma}
  
  \begin{proof}
      Let $V$ be any finite-dimensional $\mathbf{\dot{H}}^{\mathbf{Y}}$-module.
      Then, by the PBW theorem of DAHA \eqref{eqn_tri_decomp}, there is an isomorphism of $\C$-vector
      spaces $$\Delta(V) := \mathbf{\ddot{\mathbf{H}}} \otimes_{\mathbf{\dot{H}}^{\mathbf{Y}}} V \cong \C[\mathbf{X}] \otimes_\C V.$$
      It follows that $\Delta(V)$ is a free module over 
      $\mathbf{\dot{H}}^{\mathbf{Y}}$. All free modules are flat, and thus 
      $\Delta$ is exact. 
  \end{proof}
  
  \section{$\Delta$-Filtrations of $\mathbf{\ddot{\mathbf{H}}}$-modules}
  
  \begin{definition}
      A \emph{$\Delta$-filtration} by a $\mathbf{\ddot{\mathbf{H}}}$-module $M$ 
      is a (possibly infinite) filtration
      $$0 = M_0 \subset M_1 \subset \cdots,$$
      such that $\bigcup_i M_i = M$, and
      each composition factor $M_{i+1}/M_i$
      is a quotient of some $\Delta(E_i),$ where $E_i$ is an irreducible
      $\mathbf{\dot{H}}^{\mathbf{Y}}$-module --- that is, 
      $M_i/M_{i-1} \twoheadrightarrow \Delta(E_i)$.
  \end{definition}
  
  \begin{remark}
      We remark that our definition of $\Delta$-filtration here is a weaker version
      of the one given in \cite[\S 2.3.3]{ggor03} --- in particular, the authors in \emph{loc. cit.}
      requires that the $\Delta$-filtation is a finite filtration, and that each 
      composition factor is precisely isomorphic to some $\Delta(V)$, for
      irreducible $V$. This is done in
      order to satisfy the axioms for a highest weight category in the sense of 
      \cite{cps88}.
      However, we do not require such a strong condition in our case. 
      In fact, it is likely that such $\Delta$-filtrations do not 
      exist in general in the category $\mcal{O}$ of DAHA.
  \end{remark}
  
  As a consequence of Lemma \ref{lem_quot}, we have the following:
  
  \begin{corollary}\label{cor_delta_filt_iff}
      A $\mathbf{\ddot{\mathbf{H}}}$-module $M$ admits a $\Delta$-filtration if and only if
      it admits a filtration whose composition factors are quotients of 
      $\Delta(\delta_\lambda)$ for some $\lambda$.
  \end{corollary}
  
  We may define more general standard modules of the form: 
  $$\delta_\lambda^k := \on{Ind}_{\C[\mathbf{Y}]}^{\mathbf{\dot{H}}^{\mathbf{Y}}} \left(\C[\mathbf{Y}]/\mathfrak{m}_\lambda^k\right),$$
  which we then induce to the DAHA:
  $$\Delta(\delta_\lambda^k) = \on{Ind}_{\mathbf{\dot{H}}^{\mathbf{Y}}}^{\mathbf{\ddot{\mathbf{H}}}} \delta_\lambda^k.$$
  From this, we then have the following result:
  
  \begin{lemma}\label{lem_step1}
      $\Delta(\delta_\lambda^k)$ admits a $\Delta$-filtration.
  \end{lemma}
  
  \begin{proof}
      Each generalised standard module $\delta_\lambda^k$ is filtered by 
      the weight space $\C[\mathbf{Y}]/\mathfrak{m}_\lambda^k$ by 
      $$0 \subseteq \C[\mathbf{Y}]/\mathfrak{m}_\lambda \subseteq \C[\mathbf{Y}]/\mathfrak{m}_\lambda^2 \subseteq \cdots,$$
      with composition factors of the form 
      $\mathfrak{m}_\lambda^{i-1}/\mathfrak{m}_\lambda^i$ by the third 
      isomorphism theorem for modules. Each composition factor 
      $\mathfrak{m}_\lambda^{i-1}/\mathfrak{m}_\lambda^i$ has the structure of a 
      vector space over $\C[\mathbf{Y}]/\mathfrak{m}_\lambda$, and thus 
      is isomorphic to a direct sum of copies of $\C[\mathbf{Y}]/\mathfrak{m}_\lambda$.\\\\
      $\Delta(\delta_\lambda^k)$ has a filtration of the form 
      $$0\subseteq \Delta(\delta_\lambda) \subseteq \Delta(\delta_\lambda^2) \subseteq \cdots \subseteq \Delta(\delta_\lambda^k),$$
      whose factors are $$\Delta(\delta_\lambda^i)/\Delta(\delta_\lambda^{i-1}) \cong \frac{\C[\mathbf{X}] \otimes_\C \C[\mathbf{Y}]/\mathfrak{m}_\lambda^i}{\C[\mathbf{X}] \otimes_\C \C[\mathbf{Y}]/\mathfrak{m}_\lambda^{i-1}} \cong \C[\mathbf{X}] \otimes_\C (\mathfrak{m}_\lambda^{i-1}/\mathfrak{m}_\lambda^i) \cong \Delta(\mathfrak{m}_\lambda^{i-1}/\mathfrak{m}_\lambda^i),$$
      where the first and last isomorphisms follows by the PBW theorem 
      \eqref{eqn_tri_decomp},
      and the second isomorphism follows by the exactness of $\Delta$ 
      Lemma \ref{lem_delta_exact}. As aforementioned, the composition factors 
      $\mathfrak{m}_\lambda^{i-1}/\mathfrak{m}_\lambda^i$ is isomorphic to a 
      direct sum of copies of $\C[\mathbf{Y}]/\mathfrak{m}_\lambda$.
      Let $n_{i,\lambda} := \dim_{\C[\mathbf{Y}]/\mathfrak{m}_\lambda}\left(\mathfrak{m}_\lambda^{i-1}/\mathfrak{m}_\lambda^i\right)$.
      Then,
      $$\Delta(\mathfrak{m}_\lambda^{i-1}/\mathfrak{m}_\lambda^i) \cong \Delta(\delta_\lambda)^{\oplus n_{i,\lambda}},$$
      which follows since the induction functor is additive.
      It thus follows that $\Delta(\delta_\lambda^k)$ is filtered by 
      $\Delta(\delta_\lambda)$. Applying Corollary \ref{cor_delta_filt_iff}
      proves the result.
  \end{proof}
  
  \begin{lemma}\label{lem_step2}
      Let $M \in \mcal{O}_{\mathbf{\ddot{\mathbf{H}}}}$. Then, for each $m\in M$, 
      there exists $k \in \mathbb{N}$ and a map 
      $\varphi_m \in \on{Hom}_{\mathbf{\ddot{\mathbf{H}}}}(\Delta(\delta_\lambda^k),M)$ 
      such that $m \in \on{Im}(\varphi_m)$.
  \end{lemma}
  
  \begin{proof}
      Applying Frobenius reciprocity twice gives us isomorphisms:
      $$\on{Hom}_{\mathbf{\ddot{\mathbf{H}}}}(\Delta(\delta_\lambda^k),M) \cong \on{Hom}_{\mathbf{\dot{H}}^{\mathbf{Y}}}(\delta_\lambda^k,M) \cong \on{Hom}_{\C[\mathbf{Y}]}(\C[\mathbf{Y}]/\mathfrak{m}_\lambda^k, M).$$
      As aforementioned, since $M$ is $\C[\mathbf{Y}]$-semisimple, 
      it follows that there exists a non-zero $\C[\mathbf{Y}]$-homomorphism 
      $\widetilde{\varphi}_m : \C[\mathbf{Y}]/\mathfrak{m}_\lambda^k \to M \cong \bigoplus_\lambda M_\lambda^{\on{gen}}$
      satisfying the required properties. Take the image of 
      $\widetilde{\varphi}_m$ in $\on{Hom}_{\mathbf{\ddot{\mathbf{H}}}}(\Delta(\delta_\lambda^k),M)$ and we are done.
  \end{proof}
  
  \begin{lemma}\label{lem_step3}
      Let $\varphi_m$ be defined as in Lemma \ref{lem_step1}.
      Then, each $\mathbf{\ddot{\mathbf{H}}}$-submodule $\on{Im}\varphi_m$ has a filtration
      by quotients of $\Delta(V_i)$, where $V_i$ are irreducible 
      $\mathbf{\dot{H}}^{\mathbf{Y}}$-modules.
  \end{lemma}
  
  \begin{proof}
      This follows from the fact that there exists a surjection
      $\Delta(\delta_\lambda^k) \twoheadrightarrow \on{Im}\varphi_m$, and the 
      fact that $\Delta(\delta_\lambda^k)$ admits a $\Delta$-filtration.
  \end{proof}
  
  \begin{lemma}\label{lem_step3_iter}
      Let $M \in \mcal{O}_{\mathbf{\ddot{\mathbf{H}}}}$, and let $\varphi_m$ be defined as 
      in Lemma \ref{lem_step1}. Then, for some $m\neq m' \in M$, there exists a 
      a submodule $N$ of $M$ such that 
      $N/\on{Im}\varphi_{m} = \on{Im}\varphi_{m'}$.
  \end{lemma}
  
  \begin{proof}
      For each non-zero $m' \in M$, there exists a map 
      $\on{Im}\varphi_{m'} \to M/\on{Im}\varphi_m$. 
      Define a map $\pi : M \twoheadrightarrow M/\on{Im}\varphi_m$. 
      Since $\on{Im}\varphi_{m'} \subseteq M/\on{Im}\varphi_{m'}$, 
      we may consider the pre-image $\pi^{-1}(\on{Im}\varphi_{m'})$, which is a 
      submodule of $M$.
      From this, we also obtain a surjective map
      $\pi^{-1}(\on{Im}\varphi_{m'}) \twoheadrightarrow \on{Im}\varphi_{m'}$,
      whose kernel is $\on{Im}\varphi_m$. Thus, 
      by the first isomorphism theorem, we have that 
      $$\pi^{-1}(\on{Im}(\varphi_{m'})) / \on{Im}\varphi_m \cong \on{Im}\varphi_{m'},$$
      and we are done.
  \end{proof}
  
  And we thus have the following characterisation of $\C[\mathbf{Y}]$-locally
  finite $\mathbf{\ddot{\mathbf{H}}}$-modules:
  \begin{proposition}\label{prop_filtration}
      Let $M$ be a finitely-generated $\mathbf{\ddot{\mathbf{H}}}$-module. Then, the 
      following are equivalent:
      \begin{itemize}
          \item[(i)] $M \in \mcal{O}_{\mathbf{\ddot{\mathbf{H}}}}$,
          \item[(ii)] $M$ admits a $\Delta$-filtration.
      \end{itemize}
  \end{proposition}
  
  \begin{proof}
      (i) $\implies$ (ii). Lemma \ref{lem_step3_iter} shows that $M$ has a 
      filtration by quotients of $\on{Im}(\varphi_m)$. Then, Lemma 
      \ref{lem_step2} shows that each $\on{Im}(\varphi_m)$ admits a 
      $\Delta$-filtration. Thus, $M$ admits a $\Delta$-filtration. \\\\
      (ii) $\implies$ (i). Each composition factor of the filtration is of the form 
      $\Delta(V)$, where $V$ is an irreducible 
      $\mathbf{\dot{H}}^{\mathbf{Y}}$-module. By the PBW theorem \eqref{eqn_tri_decomp},
      there is an isomorphism of $\C$-vector spaces
      $\Delta(V) \cong \C[\mathbf{X}] \otimes_\C V$. Since $M$ is 
      finitely-generated, it follows that for any $m\in M$, the orbit 
      $m$ acts on $\Delta(V)$ from the right, and acts on $V$. 
      $m\cdot \mathbf{Y}$ must also be finite.
  \end{proof}
  
  \section{Torsion Objects in $\mcal{O}_{\mathbf{\ddot{\mathbf{H}}}}$}
  In this section, we define the full subcategory $\mcal{O}_{\mathbf{\ddot{\mathbf{H}}}}^{\on{tor}}$ of torsion objects in $\mcal{O}_{\ddot{\mathbf{H}}}$. This will be 
  important later when we construct the qKZ functor in the next chapter.
  In particular, one of our main theorems shows that the restriction of the qKZ 
  functor to the subcategory of torsion-free
  objects $\mcal{O}_{\mathbf{\ddot{\mathbf{H}}}}/\mcal{O}_{\ddot{\mathbf{H}}}^{\on{tor}}$
  is essentially surjective and fully faithful onto the representation category
  $\coh(\mcal{H}^{\on{ell}})$ of the ellAHA.\\\\
  By the $\C[\mathbf{Y}]$-locally finiteness condition of the category $\mcal{O}$ of DAHA, it follows that given any $M \in \mcal{O}_{\mathbf{\ddot{\mathbf{H}}}}$,
  the action of any element of $\C[\mathbf{Y}]$ on $M$ is torsion. As such, we instead wish to define the torsion objects of $\mcal{O}_{\mathbf{\ddot{\mathbf{H}}}}$ as 
  those objects which are torsion with respect to $\C[\mathbf{X}]$. That is, 
  elements $f(X) \in \C[\mathbf{X}]$ for which $f(X) \cdot M = 0$. Or equiavlently, treating $M$ as a $\C[\mathbf{X}]$-module, $M$ is torsion if 
  $\on{Ann}_{\C[\mathbf{X}]}(M) \neq 0$. Recall that the \emph{support} of $M$ -- denoted $\on{supp}(M)$ --
  is the non-vanishing locus of $\on{Ann}_{\C[\mathbf{X}]}(M)$.
  We thus make the following definition:
  
  \begin{definition}
      An element $M \in \mcal{O}_{\mathbf{\ddot{\mathbf{H}}}}$ is \emph{torsion} if its
      support $\on{supp}(M)$ is a proper subvariety of the algebraic torus 
      $\spec \C[\mathbf{X}]$.
  \end{definition}
  
  Let $\mcal{O}_{\mathbf{\ddot{\mathbf{H}}}}^{\on{tor}}$ be the full subcategory of 
  torsion objects in $\mcal{O}_{\mathbf{\ddot{\mathbf{H}}}}$. Moreover, we show that 
  $\mcal{O}_{\mathbf{\ddot{\mathbf{H}}}}^{\on{tor}}$ forms a Serre subcategory of 
  $\mcal{O}_{\mathbf{\ddot{\mathbf{H}}}}$.
  
  \begin{definition}
      Let $\mathscr{A}$ be an abelian category. Then, a subcategory $\mathscr{C}$
      is a \emph{Serre subcategory} if for any short exact sequence
      $$0\longrightarrow A \longrightarrow B \longrightarrow C \longrightarrow 0,$$
      if $A,C \in \on{Ob}(\mathscr{C})$, then $B \in \on{Ob}(\mathscr{A})$.
  \end{definition}
  
  \begin{lemma}
      The subcategory $\mcal{O}_{\mathbf{\ddot{\mathbf{H}}}}^{\on{tor}}$ is a Serre subcategory.
  \end{lemma}
  
  \begin{proof}
     Consider a short exact sequence 
     $$0 \longrightarrow M' \longrightarrow M \longrightarrow M'' \longrightarrow 0,$$
     of objects in $\mcal{O}_{\mathbf{\ddot{\mathbf{H}}}}$. Suppose that $M'$ and $M''$
     are torsion. Then, $M$ must be torsion since the union of two proper 
     subvarieties is a proper variety.
  \end{proof}
  
  As a result, one obtains the Serre quotient:
  $$\mcal{O}_{\mathbf{\ddot{\mathbf{H}}}}/\mcal{O}_{\ddot{\mathbf{H}}}^{\on{tor}}.$$
  In particular, objects of $\mcal{O}_{\mathbf{\ddot{\mathbf{H}}}}/\mcal{O}_{\ddot{\mathbf{H}}}^{\on{tor}}$ are objects of $\mcal{O}_{\ddot{\mathbf{H}}}$, but the morphisms
  are given by:
  $$\on{Hom}_{\mcal{O}_{\mathbf{\ddot{\mathbf{H}}}}/\mcal{O}_{\ddot{\mathbf{H}}}^{\on{tor}}}(M,N) := \varprojlim_{M' \to M} \on{Hom}_{\mcal{O}_{\ddot{\mathbf{H}}}}(M',N),$$
  where the limit is taken over ``isomorphisms modulo $\mcal{O}_{\mathbf{\ddot{\mathbf{H}}}}^{\on{tor}}$ $M' \to M$, meaning that its kernel and cokernel are both torsion.
  
  
  
  \chapter{The Quantum Torus}\label{chap3}
  
  %As aforementioned, the qKZ functor is a functor 
  %$$\qKZ : \mcal{O}_{\mathbf{\ddot{\mathbf{H}}}} \longrightarrow \coh^{\on{flat}}(\mcal{H}^{\on{ell}}),$$
  %where $\mcal{H}^{\on{ell}}$ is the elliptic affine Hecke algebra of 
  %\cite{gkv95}. 
  This section will outline a key ingredient that we use in the 
  construction of the qKZ functor: the \emph{quantum torus} $\qTor$ of \cite{beg00}.
  We denote by $\C(\mathbf{X})$ the algebra of meromorphic functions on the algebraic torus 
  $T = \spec\C[\mathbf{X}]$.
  The quantum torus is a $\C(\mathbf{X})$-algebra generated by $q$-difference
  operators that is isomorphic to a localisation of the DAHA, which we denote 
  by $\mathbf{\ddot{\mathbf{H}}}_{\on{loc}}$. In particular, there is an equivalence
  of categories 
  \begin{equation}\label{eqn_beg03_isom}
      \mathbf{\ddot{\mathbf{H}}}_{\on{loc}}\lmod \stackrel{\simeq}{\longrightarrow} \qTor_W\lmod,
  \end{equation}
  where $\qTor_W$ is the twisted tensor product $\qTor \rtimes \C[W]$, called 
  the \emph{$W$-equivariant quantum torus}. 
  The isomorphism is due to \cite[Theorem 7.2]{beg00}.\\\\
  A special feature of the category $\mcal{O}_{\mathbf{\ddot{\mathbf{H}}}}$ is that all $\Delta$-filtrations are of finite length, and each 
  composition factor in the filtration is precisely isomorphic to a $\mathbf{\ddot{\mathbf{H}}}$-module
  of the form $\Delta(V)$, where $V$ is an irreducible $\mathbf{\dot{H}}^{\mathbf{Y}}$-module. This is proven in Lemma \ref{lem_finite_filt}.\\\\
  Another feature of $\mcal{O}_{\mathbf{\ddot{\mathbf{H}}}}$ is demonstrated in
  Proposition \ref{prop_cat_o_Fuchsian}, which shows that the restriction of the 
  functor \eqref{eqn_beg03_isom} to the category 
  $\mcal{O}$ gives a functor $$\mcal{O}_{\mathbf{\ddot{\mathbf{H}}}} \longrightarrow \qTor_W^{\on{fuch}}\lmod,$$
  into the Fuchsian subcategory of $\qTor_W$. \\\\
  We give a definition of the qKZ functor, and outline the construction of this functor. 
  In Chapter \ref{chap4} we will outline the construction of this functor explicitly.
  
  \section{The Quantum Torus}
  
  As before, let $T = \spec \C[\mathbf{X}]$, and $T^\vee = \spec \C[\mathbf{Y}]$
  be the algebraic dual torus of $T$. Let $\C(T)$ be the algebra of meromorphic
  functions on $T$. By abuse of notation, we will denote $\C(T)$ by $\C(\mathbf{X})$ to emphasise 
  that we are considering meromorphic functions coming from the torus generated by the weight lattice.
  Similarly, for $\C(\mathbf{Y}) := \C(T^\vee)$.
  The \emph{$q$-shift operator} acts on $\C(T)$ by:
  $$\on{D}^{\mu^\vee}_q f(t) = f(q^{2\mu}t),$$
  which acts on meromorphic functions $f(X)$ on the torus $T$ by 
  $$\on{D}_q^{\mu^\vee} f(X) = f(q^{2\mu^\vee}X),$$ 
  where $X = (X^{\omega_1},\cdots, X^{\omega_i})$, where $\omega_i$ are the fundamental weights 
  forming a basis of $\mathbf{X}$. Then, $q^{2\mu} X = (X^{\langle \mu^\vee,\omega_1\rangle}, \cdots, X^{\langle \mu^\vee,\omega_r\rangle})$.
  For some $X^\mu \in \mathbb{X}^\ast(T)$, we also have the following commutation relation:
  $$\on{D}^{\mu^\vee}_q \cdot X^\lambda \cdot (\on{D}^{\mu^\vee}_q)^{-1}= q^{2\langle \mu^\vee,\lambda\rangle} X^\lambda.$$
  The algebra of meromorphic functions $\C(\mathbf{X})$, together with the 
  $q$-differential operator $\on{D}_q^{\mu^\vee}$, generate an algebra called 
  the \emph{quantum torus} -- denoted by $\qTor$. It has an explicit presentation
  as a $\C(\mathbf{X})$-algebra by:
  $$\qTor = \langle \on{D}_q^{\mu^\vee} : \text{$\on{D}_q^{\mu^\vee} X^\lambda (\on{D}_q^{\mu^\vee})^{-1} = q^{2\langle \mu^\vee,\lambda \rangle} X^\lambda$ for $X^\lambda \in \C(\mathbf{X})$}\rangle.$$
  For simplicity, let us write $\on{D}_q$ for the $q$-shift operator when
  it is not necessary to specify $\mu^\vee$.
  One may define the module category $$\qTor\lmod,$$ in two ways. One is to
  consider objects $M \in \qTor\lmod$ as $\C(\mathbf{X})$-vector spaces,
  and morphisms as $\C(\mathbf{X})$-linear maps that respect the action by
  the $q$-shift operators. However, if we choose a basis \
  $\lbrace e_i \rbrace_{i=1}^n$ for $M$, then elements
  of $\qTor\lmod$ can be identified as pairs 
  $(\C(\mathbf{X})^n, A)$, where $A \in \on{GL}_n(\C(\mathbf{X}))$, and morphisms
  $(\C(\mathbf{X})^n,A) \to (\C(\mathbf{X})^p,B)$ are given by matrices
  $F \in \on{Mat}_{p,n}(\C(\mathbf{X}))$ satisfying the gauge transformation
  relation $(\on{D}_q F)A = BF$. In other words, objects 
  of $\qTor\lmod$ are $q$-difference systems $$\on{D}_qV(X) = A(X) V(X),$$
  where $V(X) \in \C(\mathbf{X})^n$.
  This allows us to identify $\qTor\lmod$ with the category of $q$-difference
  equations $\DiffEq(\C(\mathbf{X}),\on{D}_q)$ of \cite{sauloy03}.\\\\
  This algebra inherits a 
  Weyl group action, which allows us to define the twisted tensor product given 
  by:
  $$\qTor \rtimes \C[W],$$
  with algebra structure given by 
  $$(f\otimes w)\cdot (g\otimes v) = (f \cdot {}^wg) \otimes (wv),$$
  and ${}^wf(t) := f(w^{-1}t)$. 
  In particular, the resulting twisted algebra has elements of the form:
  $$\sum_{\substack{w \in W\\ \mu^\vee \in \mathbf{Y}}} h_{w,\mu^\vee} \on{D}^\mu_q[w] : f \longmapsto \sum_{\substack{w\in W\\ \mu^\vee \in \mathbf{Y}}} h_{w,\mu^\vee} \on{D}^\mu_q({}^wf), \quad h_{w,\mu^\vee}\in \C(T).$$
  We call this resulting algebra the \emph{$W$-equivariant quantum torus},
  and we denote it by 
  $$\qTor_W.$$ 
  Then, $\qTor_W\lmod$ is the category of $W$-equivariant $q$-difference equations
  in $\C(\mathbf{X})$, which we will denote by 
  $\DiffEq_W(\C(\mathbf{X}),\on{D}_q)$.
  
  \section{The $\mathbb{Z}/2\mathbb{Z}$-equivariant Connection Category}
  
  We restrict ourselves to the rank one case for this section, as the connection category of \cite{sauloy03}, 
  only works for $q$-difference systems of one variable, which corresponds to the rank one localised DAHA. 
  In this case, the Weyl group $W$ is $\mathfrak{S}_2 \cong \mathbb{Z}/2\mathbb{Z}$.\\\\
  As aforementioned in Chapter \ref{sec_qdiff}, there exists a connection category 
  of $\mcal{C}onn$ whose objects are triples $(\mcal{F}_0,\mcal{F}_\infty,\varphi)$,
  where $\mcal{F}_0$ and $\mcal{F}_\infty$ are flat vector bundles on an elliptic curve 
  $E$, and $\varphi$ is the monodromy morphism $\varphi : \mcal{F}_0 \dashrightarrow \mcal{F}_\infty$.
  Since we are considering vector bundles $\mcal{F}_0$ and $\mcal{F}_\infty$ coming from 
  fundamental solutions of equations from $\qTor_{\mathfrak{S}_2}$, one needs to take into the account
  the equivariance structure on the sheaves $\mcal{F}_0$ and $\mcal{F}_\infty$ as well.\\\\
  Let $$\mcal{C}onn_W,$$ be the category of $W$-equivariant connection data,
  whose objects are triples $(\mcal{F}_0,\mcal{F}_\infty,\varphi)$, together with the action of the simple reflection $s \in \mathfrak{S}_2$ given by:
  $$s^\ast (\mcal{F}_0,\mcal{F}_\infty, \varphi) := (s^\ast\mcal{F}_\infty, s^\ast\mcal{F}_0, w^\ast\varphi^{-1}),$$
  and a morphism
  $$B_s : (\mcal{F}_0,\mcal{F}_\infty,\varphi) \longrightarrow s^\ast (\mcal{F}_0,\mcal{F}_\infty,\varphi),$$
  given by the commutative diagram
  \begin{equation}\label{eqn_conn_comm}
      \begin{tikzcd}
          \mcal{F}_0 \arrow[d, "\varphi"', dashed] \arrow[r, "B_s^{(0)}", dashed] & s^\ast\mcal{F}_\infty \arrow[d, "s^\ast\varphi^{-1}", dashed] \\
          \mcal{F}_\infty \arrow[r, "B_s^{(\infty)}"', dashed]                         & s^\ast\mcal{F}_0                                             
      \end{tikzcd}
  \end{equation}
  and with the property that $$s^\ast B_s \circ B_s = B_{e}.$$
  Taking the pushforwrad by $\pi : E \to E/\mathfrak{S}_2$, we have that 
  $\left(\pi_\ast B_s\right)^2 = \on{id}$.
  There are forgetful functors:
  $$\mcal{C}onn_W \longrightarrow \mcal{C}onn, \quad \qTor^{\on{fuch}}_W \longrightarrow \qTor^{\on{fuch}}.$$
  By construction, the following holds:
  \begin{corollary}
      Applying the $q$-Riemann-Hilbert functor of \cite{sauloy03} to 
      $\qTor_W^{\on{fuch}}$ gives an equivalence of categories:
      $$\begin{tikzcd}
          \qTor^{\on{fuch}}_W \arrow[r, "\cong"] \arrow[d, "\on{forget}"'] & \mcal{C}onn_W \arrow[d, "\on{forget}"] \\
          \qTor^{\on{fuch}} \arrow[r]                                      & \mcal{C}onn                           
      \end{tikzcd}$$
  \end{corollary}
  
  \section{Quantum Tori to DAHA}
  
  We follow \cite[\S 6]{beg00} for this section. 
  Recall that $T = \mathbb{G}_{\on{m}}\otimes_{\mathbb{Z}} \mathbf{X}$. 
  Since there is an isomorphism $\mathbb{X}^\ast(T) \cong \mathbf{X}$, 
  we may identify elements
  $X^\mu \in \C[\mathbf{X}]$ with functions on $T$.\\\\
  Let $\Phi$ be a root system of rank $r$, and let $\widehat{\Phi}$ 
  denote the affine root system, and let 
  $\widehat{T} = T \times \mathbb{G}_{\on{m}}$ be the corresponding 
  algebraic torus. We may identify the Laurent polynomial ring $\C[q^\pm]$
  with the $\C$-points of the group scheme $\mathbb{G}_{\on{m}}$, and 
  $\C[t_\alpha^\pm]$ with the $\C$-points of $T$. The affine Hecke algebra
   associated to the affine Weyl group $W \rtimes \mathbf{X}$ has a 
   faithful representation in $\C[\widehat{T}]$ via the Demazure-Lusztig
   operators  (c.f. Theorem \ref{thm_demazure_lusztig}, \cite[Theorem 4.3]{kir97},
   \cite[(6.1)]{beg00}):
   $$T_i = t_i s_i + (t_i-t_i^{-1})\frac{s_i-1}{X^{\alpha_i} - 1},\quad i = 0,1,\cdots, r,$$
  satisfying the Iwahori-Matsumoto relations for the AHA. Then, the 
  double affine Hecke algebra $\mathbf{\ddot{\mathbf{H}}}$ can be defined as the 
  subalgebra of $\C[t_\alpha^\pm]$-linear endomorphisms of 
  $\C[\widehat{T}][t_\alpha^\pm]$ generated by multiplication operators 
  of the form $T_i$ and $X^\mu$, that commute via the Bernstein relation.
  The \emph{localised DAHA} is then constructed in the following way:
  $$\mathbf{\ddot{\mathbf{H}}}_{\on{loc}} := \ddot{\mathbf{H}} \otimes_{\C[\widehat{T}]} \C(\widehat{T}).$$
  Recall that the DAHA itself comes equipped with a $q$-difference operator 
  $\tau(\mu^\vee)$ (see \eqref{eqn_shift_operator}), that acts 
  precisely the same way as $\on{D}_q^{\mu^\vee}f(t)$. 
  Thus, we have the following theorem from \cite{beg00}:
  \begin{theorem}[Theorem 7.2, \cite{beg00}]
      There is an isomorphism:
      $$\mathbf{\ddot{\mathbf{H}}}_{\on{loc}} \cong \qTor_W.$$
  \end{theorem}
  Equivalently, there is an equivalence of categories 
  $$\mathbf{\ddot{\mathbf{H}}}_{\on{loc}}\lmod \cong \qTor_W\lmod.$$
  Given any $\mathbf{\ddot{\mathbf{H}}}$-module $M$, let 
  $$M_{\on{loc}} := M \otimes_{\C[\widehat{T}]} \C(\widehat{T}),$$ denote the corresponding 
  $\mathbf{\ddot{\mathbf{H}}}_{\on{loc}}$-module. Denote by $\on{Loc} : \mcal{O}_{\ddot{\mathbf{H}}} \to \qTor_W\lmod$ 
  the \emph{localisation functor} given by the composition:
  $$\begin{tikzcd}
      \mathcal{O}_{\mathbf{\ddot{\mathbf{H}}}} \arrow[r] \arrow[rr, "\operatorname{Loc}"', bend right] & \mathbf{\ddot{\mathbf{H}}}_{\operatorname{loc}}\lmod \arrow[r, "\cong"] & \qTor_W\lmod
  \end{tikzcd}$$
  
  \begin{remark}
      Note that \cite[Theorem 7.2]{beg00} is stated in terms of technical residue
      conditions, similar to the ones seen in \cite{gkv95} for the defintion of 
      ellAHA. However, for our purposes, we unpack \cite{beg00}'s theorems and 
      definitions only to the extent that we require.
  \end{remark}
  
  \section{Recipe for a qKZ Functor}
  
  Since $\qTor_W$ is generated 
  by meromorphic functions on $T$, and $q$-difference operators, it follows
  that the category $\qTor_W\lmod$ can be identified with the category of 
  $q$-difference equations $\mathbf{DiffEq}(\C(T), \on{D}^{\mu^\vee}_q)$
  of \cite{sauloy03}. Let $\qTor_W^{\on{fuch}}\lmod$ be the full subcategory
  of (strictly) Fuchsian $q$-difference equations.
  
  \begin{lemma}\label{lem_finite_filt}
      Let $M_{\on{loc}}$ be the image of $M \in \mcal{O}_{\mathbf{\ddot{\mathbf{H}}}}$ under the localisation 
      functor. 
      Then, $M_{\on{loc}}$ admits a finite $\Delta$-filtration whose composition factors are 
      isomorphic to $\Delta(V)$, where $V$ is an irreducible 
      $\mathbf{\dot{H}}^{\mathbf{Y}}$-module.
  \end{lemma}
  
  \begin{proof}
      Let $M \in \mcal{O}_{\mathbf{\ddot{\mathbf{H}}}}$, and let 
      $$0 = M_0 \subseteq M_1 \subseteq \cdots \subseteq M,$$
      be a $\Delta$-filtration. Then, by definition we have that
      $\Delta(V) \twoheadrightarrow M_i/M_{i-1}$. 
      Observe that if $V$ is an irreducible $\mathbf{\ddot{\mathbf{H}}}^{\mathbf{Y}}$-module,
      then, $$\Delta(V)_{\on{loc}} = (\C[\mathbf{X}] \otimes_\C V)_{\on{loc}} = \C[\mathbf{X}]\otimes_\C V \otimes_{\C(\widehat{T})} \C(\widehat{T}) \cong \C(\widehat{T}) \otimes_\C V = V_{\on{loc}},$$
      which is an irreducible $\ddot{\mathbf{H}}_{\on{loc}}$-module.
      Since $(M_i/M_{i-1})$ is a quotient of $\Delta(V)$, we see that after 
      localisation, either $(M_i/M_{i-1})_{\on{loc}} = 0$, or 
      $(M_i/M_{i-1})_{\on{loc}} \cong (\Delta(V))_{\on{loc}}$. 
      Moreover, since $M$ is finitely-generated, each generator can only lie 
      in finitely many pieces in the filtration. Thus, all but finitely many composition factors are killed 
      off after taking localisation.
  \end{proof}
  
  \begin{lemma}\label{lem_Fuchsian_ext}
      The subcategory of Fuchsian equations is closed under extensions. 
  \end{lemma}
  
  \begin{proof}
      It is sufficient to prove this result for strictly Fuchsian equations.
      By \cite[Lemma 2.1.2.1]{sauloy03}, the full subcategory of strictly
      Fuchsian equations is an essential subcategory. 
      Thus, let $N$ and $N'$ be fuchsian $\qTor$-modules fitting into a short exact sequence:
      $$0 \longrightarrow N \longrightarrow M \longrightarrow N' \longrightarrow 0.$$
      Choose a basis $\lbrace e_1,\cdots,e_n \rbrace$ for $N$. 
      Then, $\on{D}^{\omega_j^\vee}_q \cdot e_i = A_{ij} e_j$, which gives a matrix 
      $A_{ij} \in \on{GL}_n(\C(\mathbf{X}))$ by the Fuchsian condition on $N$. From this, one may extend this 
      to a basis $\lbrace e_1,\cdots,e_n,e_{n+1},\cdots,e_m\rbrace$ of $M$.
      As an $N$-module, we may identify the matrix $[A_{ij}]_{i,j=1}^n$ 
      with the $m\times m$ matrix for which $A_{ij} = 0$ for all $j = n+1,\cdots,m$.
      This gives us an $m\times m$ block upper triangular matrix 
      $$A = \begin{pmatrix}
          \ast & \ast\\
          0 & \ast 
      \end{pmatrix}.$$
      The modules $N$ and $N'$ correspond to the diagonal blocks, and the upper triangular block corresponds to $M$. By the Fuchsian condition of $N$ 
      and $N'$, it follows that the block diagonal matrices 
      are invertible. Thus, the matrix $A$ is invertible, since a block upper triangular matrix with invertible diagonals is invertible.
      $M$ is therefore Fuchsian.
  \end{proof}
  
  \begin{lemma}\label{lem_Fuchsian_quotient}
      Any submodule of a Fuchsian $q$-difference module is Fuchsian.
  \end{lemma}
  
  \begin{proof}
      Repeat the same process as above to obtain a block upper triangular matrix.
      Taking quotients amounts to choosing one of of the block diagonal matrices,
      which, as aforementioned, is invertible.
  \end{proof}
  
  \begin{proposition}\label{prop_cat_o_Fuchsian}
      The localisation functor lands in the full subcategory $\qTor^{\on{fuch}}_W$ of $\qTor_W$. That is, 
      $\on{Loc}$ is a functor:
      $$\on{Loc} : \mcal{O}_{\mathbf{\ddot{\mathbf{H}}}} \longrightarrow \qTor_W^{\on{fuch}}\lmod.$$
  \end{proposition}
  
  \begin{proof}
      Given any $M \in \mcal{O}_{\mathbf{\ddot{\mathbf{H}}}}$, the 
      $\qTor_W$-module $\qTor_W\otimes_\C M_{\on{loc}}$ has a finite filtration with composition factors 
      of the form $\Delta(V)_{\on{loc}}$, where $V$ is an irreducible $\mathbf{\dot{H}}^{\mathbf{Y}}$-module. 
      Choose a basis for $\Delta(V)_{\on{loc}}$, and write 
      the $q$-differential operator $\on{D}^{\mu^\vee}_q$
      in this basis, which produces an invertible square matrix $A(X) \in \on{GL}_n(\C(X))$.
      Then, for an arbitrary element $V \otimes_\C f(X) \in \Delta(V)_{\on{loc}}$, we have:
      $$V \otimes_\C \left(\on{D}_q^{\mu^\vee} f(X)\right) = V \otimes_\C \left(A(X) f(X)\right),$$
      using the fact that $\on{D}_q^{\mu^\vee}$ acts on $\Delta(V)_{\on{loc}}$ by the matrix $A(X)$. But also $\on{D}_q^{\mu^\vee}f(X) = f(q^{2\mu}X)$,
      and thus we have a $q$-difference system 
      $$f(q^{2\mu}X) = A(X) f(X).$$
      Since $A(X)$ is invertible, it follows then that if one takes the limit $X \to 0$ or $X \to \infty$,
      one obtains an invertible scalar matrix, and thus $\Delta(V)_{\on{loc}}$ is Fuchsian.
      Further, $\Delta(V)_{\on{loc}} \subseteq \Delta(\delta_\lambda^k)_{\on{loc}}$ -- 
      that is, $\Delta(V)_{\on{loc}}$ is a quotient of $\Delta(\delta_\lambda^k)_{\on{loc}}$
      Thus, by Lemma \ref{lem_Fuchsian_ext} and \ref{lem_Fuchsian_quotient},
      $\Delta(\delta_\lambda^k)_{\on{loc}}$ is also Fuchsian. Moreover, by 
      Lemma \ref{lem_step1} and \ref{lem_finite_filt},
      $\Delta(\delta_\lambda^k)_{\on{loc}}$ admits a finite $\Delta$-filtration.
  \end{proof}
  
  As a consequence, we have the following:
  
  \begin{corollary}\label{cor_quotient_tor}
      The localisation functor factors through the Serre quotient 
      $\mcal{O}_{\mathbf{\ddot{\mathbf{H}}}} / \mcal{O}_{\ddot{\mathbf{H}}}^{\on{tor}}$ 
      to give a functor:
      $$\mcal{O}_{\mathbf{\ddot{\mathbf{H}}}} / \mcal{O}_{\ddot{\mathbf{H}}}^{\on{tor}} \longrightarrow \qTor_W^{\on{fuch}}\lmod.$$
  \end{corollary}
  
  \begin{proof}
      Immediate consequence of Lemma \ref{lem_Fuchsian_quotient},
      Proposition \ref{prop_cat_o_Fuchsian}, and the universal property of 
      Serre quotients.
  \end{proof}
  
  %Then, \cite[Theorem 2.3.2.1]{sauloy03} tells us that there is a
  %$q$-Riemann-Hilbert functor 
  %$$\qTor_W^{\on{fuch}}\lmod \longrightarrow \mcal{C}onn_W,$$
  %where $\mcal{C}onn_W$ is a category of $W$-equivariant elliptic triples
  %$(\mcal{F}_0,\mcal{F}_\infty,\varphi)$, where $\mcal{F}_0$ and $\mcal{F}_\infty$
  %are degree $0$ vector bundles on an elliptic curve $E$, and 
  %$\varphi : \mcal{F}_0 \to \mcal{F}_\infty$ is a rational morphism. 
  %The $W$-equivariance condition means that there are isomorphisms 
  %$\mcal{F}_0 \cong w^\ast\mcal{F}_0$ and 
  %$\mcal{F}_\infty \cong w^\ast\mcal{F}_\infty$ for each $w\in W$. Our goal will be to show that 
  %the pushforwards of $\mcal{F}_0$ and $\mcal{F}_\infty$ by $\pi : \mathfrak{A} \to \mathfrak{A}/W$ 
  %can be equipped with the structure of a $\mcal{H}^{\on{ell}}$-module. 
  %%Moreover, for the rank one ellAHA $\mcal{H}^{\on{ell}}_{\mathfrak{sl}_2}$, we have an embedding
  %%$$\coh^{\on{flat}}(\mcal{H}^{\on{ell}}_{\mathfrak{sl}_2}) \longrightarrow \mcal{C}onn_W,$$
  %%mapping a $\mcal{H}^{\on{ell}}$-module $\mcal{F}$ into one of 
  %%$\mcal{F}_0$ or $\mcal{F}_\infty$.
  %Given any vector bundle $\mcal{F}$ over $\mathfrak{A}$, we have the following
  %characterisation for when $\mcal{F} \in \coh(\mcal{H}^{\on{ell}})$:
  
  \begin{lemma}\label{lem_module_over_ellaha}
      Let $\mcal{F}$ be a vector bundle over $\mathfrak{A}$. Then, for each 
      $w\in W$, let $\Delta_w$ be a meromorphic morphism 
      $\mcal{F} \dashrightarrow w^\ast \mcal{F}$. Then, $\pi_\ast \mcal{F}$ has the 
      structure of a sheaf of modules over $\mcal{H}^{\on{ell}}$
      if the following hold:
      \begin{itemize}
          \item[(i)] $\pi_\ast\Delta_e = \on{id}_{\pi_\ast\mcal{F}}$, and $\lbrace \pi_\ast\Delta_w\rbrace_{w \in W}$ satisfies the relations of the Weyl group $W$,
          \item[(ii)] $(\pi_\ast\Delta_w + \pi_\ast\Delta_{s_\alpha w})\vert_{T_\alpha}$ vanishes,
          \item[(iii)] $\Delta_w$ could have a pole of order $1$ along 
              the divisor $T_{\alpha,\hbar}$ for each $\alpha \in \Phi(w)$.
      \end{itemize}
  \end{lemma}
  
  \begin{proof}
      Recall that a rational section of $\mcal{H}^{\on{ell}}$ are elements of the twisted 
      group algebra $\mcal{O}[W]$ of the form $T := \sum_{w\in W}f_ww$, for some $w \in W$. 
      It is sufficient to show that the sum $\sum_{w\in W} \pi_\ast(f_w\Delta_w)$ defines a regular morphism 
      $\pi_\ast \mcal{F} \to \pi_\ast\mcal{F}$, given that conditions (i) -- (iii) hold.\\\\
      One may identify each $f_w$ as a rational morphism $f_w : \mcal{F}\dashrightarrow \mcal{F}$. 
      For each $f_ww$, replace the Weyl group elements of with $\Delta_w$ to obtain $f_w\Delta_w$.
      These elements can be identified as morphisms $f_w\Delta_w : \mcal{F} \dashrightarrow w^\ast\mcal{F}$. Note that the sum $\sum_{w\in W} f_w \Delta_w$ 
      is not well-defined since the target of each of the maps $f_w\Delta_w$ is different. However, if 
      we take the pushforward, then we have $$\pi_\ast(f_s\Delta_s) : \pi_\ast\mcal{F} \dashrightarrow \pi_\ast s^\ast\mcal{F}.$$
      Observe that by the $W$-invariance of $\pi_\ast\mcal{F}$, we have that $\pi_\ast s^\ast \mcal{F} \cong \pi_\ast\mcal{F}$.
      Thus, the sum $\sum_{w\in W} \pi_\ast(f_w \Delta_w)$ is a well-defined map $\pi_\ast\mcal{F} \to \pi_\ast\mcal{F}$.
      By abuse of notation, we will write $f_ww$ and $f_w\Delta_w$ for its image in the pushforward $\pi_\ast(f_ww)$ and $\pi_\ast(f_w\Delta_w)$.
      Given another element of the form $g_w \Delta_w$, we may use (i) to define their product is given by: 
      $$\left(\sum_{w \in W} f_w \Delta_w \right)\cdot \left(\sum_{v\in W} g_v \Delta_v\right) = \sum_{w,v\in W} f_w \cdot {}^wf_v \cdot \Delta_{wv},$$
      where the coefficients of $\Delta_{wv}$ are multiplied according to the twisted group algebra structure on 
      $\mcal{O}[W]$. This gives the required multiplication map $\mcal{H}^{\on{ell}} \otimes_{\mcal{O}_{\mathfrak{A}/W}} \pi_\ast\mcal{F} \to \pi_\ast\mcal{F}$ 
      in $\coh(\mathfrak{A}/W)$.
      Thus, condition (ii) implies then that the terms $\Delta_w$ has vanishing order at least $1$ along 
      $T_\alpha$, and thus the terms of the form $f_w\Delta_w$ are regular. \\\\
      Note that 
      \begin{equation}\label{eqn_thingg}
          \left(f_w\Delta_w + f_{s_\alpha w}\Delta_{s_\alpha w}\right)\vert_{T_\alpha} = \left(\Delta_w\vert_{T_\alpha} + \Delta_{s_\alpha w}\vert_{T_\alpha}\right)\cdot f_w\vert_{T_\alpha},
      \end{equation}
      since $f_{s_\alpha w}\vert_{T_\alpha} = f_w \vert_{s\cdot T_\alpha} = f_w\vert_{T_{-\alpha}} = f_w\vert_{T_\alpha}$.
      By \eqref{eqn_thingg}, and the residue condition from Definition \ref{def_ellaha}(b): 
      $$\on{Res}_{T_\alpha}(f_w) + \on{Res}_{T_\alpha}(f_{s_\alpha w}) = 0,$$
      it follows then that terms of the form $f_w\Delta_w + f_{s_\alpha w}\Delta_{s_\alpha w}$ are regular if $(\Delta_w + \Delta_{s_\alpha w})\vert_{T_\alpha} = 0$.
      Definition \ref{def_ellaha}(c) states that $f_w$ vanishes along the divisor $T_{\alpha,\hbar}$ for any $\alpha \in \Phi(w)$. 
      In this case, the term $f_w\Delta_w$ will remain regular even if $\Delta_w$ has a pole of order at most $1$ along $T_{\alpha,\hbar}$. Together, these all imply that 
      we have a regular morphism: $$\sum_{w\in W} f_w\Delta_w : \pi_\ast \mcal{F} \longrightarrow \pi_\ast\mcal{F}.$$ 
  \end{proof}
  
  By Lemma \ref{lem_module_over_ellaha},the problem of determining whether $\pi_\ast\mcal{F}_0$ and $\pi_\ast\mcal{F}_\infty$ 
  are $\mcal{H}^{\on{ell}}$-modules will involve an analysis of the poles of the monodromy map $\varphi : \mcal{F}_0 \to \mcal{F}_\infty$. 
  The maps $B_w^{(0)} : \mcal{F}_0 \to w^\ast\mcal{F}_\infty$, and $B_w^{(\infty)} : \mcal{F}_\infty \to w^\ast\mcal{F}_0$ 
  will also need to be analysed in the general case.\\\\
  As we will see in Chapter \ref{chap4}, in the rank one case there is only 
  one morphism in the $\mathbb{Z}/2\mathbb{Z}$-equivariant connection category, given by $B_s$. Moreover, 
  $\mcal{F}_0 \cong \mcal{F}_\infty^\vee \cong s^\ast\mcal{F}_0$, and thus the morphism $B_s^{(0)}$ and
  $B_s^{(\infty)}$) are constant maps, hence regular. As such, the $\mcal{H}^{\on{ell}}$-module 
  structure on $\pi_\ast\mcal{F}_0$ and $\pi_\ast\mcal{F}_\infty$ is controlled entirely by the poles of the monodromy map 
  $\varphi : \mcal{F}_0 \to \mcal{F}_\infty$.\\\\
  The \emph{quantum Knizhnik-Zamolodchikov} (qKZ) functor is a functor 
  $$\qKZ : \mcal{O}_{\mathbf{\ddot{\mathbf{H}}}} \longrightarrow \coh^{\on{flat}}(\mcal{H}^{\on{ell}}),$$ 
  %mapping elements in $\mcal{O}_{\mathbf{\ddot{\mathbf{H}}}}$ to flat, homogeneous vector bundles 
  %whose image in the composition $\qRH \circ \on{Loc}$ is in $\coh^{\on{flat}}(\mcal{H}^{\on{ell}})$.
  that uniquely factors through the $q$-Riemann-Hilbert functor in the following way:
  $$\begin{tikzcd}
      \qTor^{\on{fuch}}_W\lmod \arrow[rr, "\qRH"] \arrow[rd, "\qKZ"'] &                                                        & \mcal{C}onn_W \\
                                                                      & \coh^{\on{flat}}(\mcal{H}^{\on{ell}}) \arrow[ru, hook] &              
  \end{tikzcd}$$
  In particular, we will find that $\pi_\ast\mcal{F}_0 \in \coh(E/(\mathbb{Z}/2\mathbb{Z}))$ has the structure of a 
  $\mcal{H}^{\on{ell}}$-module, and thus the $\qKZ$ maps an appropriate $\qTor_W^{\on{fuch}}$-module to $\pi_\ast\mcal{F}_0$, 
  which is then embedded into $\mcal{C}onn_{\mathbb{Z}/2\mathbb{Z}}$. Since $\mcal{F}_\infty \cong \mcal{F}_0^\vee$, it follows that 
  $\pi_\ast\mcal{F}_\infty$ also has the structure of a $\mcal{H}^{\on{ell}}$.\\\\
  In Chapter \ref{chap4}, we outline construct vector bundles $\mcal{F}_0$ and $\mcal{F}_\infty$ 
  by studying $q$-difference equations arising from
  standard modules in $\mcal{O}_{\mathbf{\ddot{\mathbf{H}}}}$. We study its monodromy, and we outline a proof of Conjecture \ref{main_thm1}. 
  The only thing that remains to be to proven is a technical condition, outlined in Conjecture \ref{annoying}. 
  Ultimately, we expect to obtain a diagram:
  $$\begin{tikzcd}
  \mcal{O}_{\mathbf{\ddot{\mathbf{H}}}} \arrow[r, "\on{Loc}"] \arrow[d, "\qKZ"'] & \qTor_W^{\on{fuch}}\lmod \arrow[d, "\qRH"] \\
  \coh(\mcal{H}^{\on{ell}}) \arrow[r, hook]                           & \mcal{C}onn_W                                       
  \end{tikzcd}$$
  We observe that given some $M_{\on{loc}} \in \mcal{O}_{\mathbf{\ddot{\mathbf{H}}}}$,
  $M_{\on{loc}} = 0$ if and only if $M$ is torsion. Therefore 
  $\qKZ(M) \neq 0$ if and only if $M \in \mcal{O}_{\mathbf{\ddot{\mathbf{H}}}}/\mcal{O}_{\ddot{\mathbf{H}}}^{\on{tor}}$.
  This shows that the restriction of $\on{qKZ}$ to the Serre quotient is faithful. Moreover, we expect that the restriction to this 
  quotient gives a fully faithful, surjective functor -- this is the content of Conjecture \ref{main_thm2}.\\\\
  %\begin{lemma}\label{lem_module_over_dynaha}
  %    Let $\mcal{F}$ be a vector bundle over $\mathfrak{A} \times \mathfrak{A}^\vee$, and let 
  %    $\pi \times \pi^\vee : \mathfrak{A} \times \mathfrak{A}^\vee \to \mathfrak{A}/W \times \mathfrak{A}^\vee/W^\dyn$
  %    be the natural map. Then, for each $w \in W$, and 
  %    $v^\dyn \in W^\dyn$, let $\Delta_{w,v^\dyn}$ be a rational morphism 
  %    $\mcal{F} \to w^\ast (v^\dyn)^\ast \mcal{F}$. Then, $(\pi \times \pi^\vee)^\ast \mcal{F}$ 
  %    has the structure of a $\mcal{H}^\dyn$-module if the following hold:
  %    \begin{itemize}
  %        \item[(i)] $\Delta_{e,e^\dyn} = \on{id}_\mcal{F}$ and $\lbrace \Delta_{w,v^\dyn} \rbrace_{w\in W, v^\dyn \in W^\dyn}$ 
  %        satisfies the relations of the Weyl groups $W$ and $W^\dyn$,
  %        \item[(ii)] $(\Delta_{w,v^\dyn} - \Delta_{s_\alpha w, s_\alpha^\dyn v^\dyn})\vert_{T_\alpha}$ vanishes,
  %        \item[(iii)] $\Delta_{w,v^\dyn}$ could have a pole of order $1$ along the divisor $T_{\alpha, \hbar}$, 
  %        or $T_{\alpha^\vee, z}$, for each $\alpha \in \Phi(w)$, with no other poles.
  %    \end{itemize}
  %\end{lemma}
  
  \chapter{Construction of the Rank One qKZ Functor}\label{chap4}
  
  To begin, we derive some 
  useful relations about the $q$-shift operator \eqref{eqn_shift_operator} in 
  the first section. 
  The first key result is Proposition \ref{prop_qkz_daha},
  which gives an explicit link between the rank one DAHA representations
  admitting $\Delta$-filtrations and the qKZ equation.\\\\
  Sections 4.4 and 4.5 are dedicated to calculating solutions of the rank one qKZ equation,
  and calculating its monodromy matrix $M$. Using techniques of \cite{sauloy03}, we can compute 
  $M$ up to some $q$-periodic constants. However, using \cite{efk98}, we can compute these 
  $q$-periodic constants explicitly. No originality is claimed in these sections.\\\\
  In the last section, we construct the qKZ functor explicitly for the case of standard modules.
  Given a triple of connection data $(\mcal{F}_0, \mcal{F}_\infty, \varphi)$, we wish 
  to construct morphisms $\Delta_s^{(0)} : \mcal{F}_0 \to s^\ast\mcal{F}_0$, and 
  $\Delta_s^{(\infty)} : \mcal{F}_\infty \to s^\ast\mcal{F}_\infty$, and compute 
  their sections explicitly, so that we can analyse its poles to show that the conditions outline in
  Lemma \ref{lem_module_over_ellaha}. Let $\pi : E \to E/(\mathbb{Z}/2\mathbb{Z})$.
  We then outline how $\pi_\ast\mcal{F}_0$ and $\pi_\ast\mcal{F}_\infty$ may be equipped with the structure of a module over 
  the elliptic affine Hecke algebra $\mcal{H}^{\on{ell}}$. The only technical condition that requires to be checked 
  before we can prove that $\pi_\ast\mcal{F}_0,\pi_\ast\mcal{F}_\infty \in \coh^{\on{flat}}(\mcal{H}^{\on{ell}})$ 
  is Conjecture \ref{annoying}.
  
  \section{Rank One Reduction}
  
  For this chapter, we will restrict ourselves entirely to the rank one case.
  This section details some results that we will use in the remainder of our 
  thesis. In particular, the $q$-shift
  operator arising in \eqref{eqn_shift_operator} is of great interest to us, as this is will 
  play the role of the $q$-difference operator in Chapter \ref{sec_qdiff}.\\\\
  In the rank one case, there is only one simple root -- which we denote by 
  $\alpha$. We choose $\alpha = e_1 - e_2$ as our representative for the simple root,
  where $e_i$ are the canonical basis vectors of $\mathbb{R}^2$.
  The fundamental weight 
  is given by $\omega = \frac{1}{2}\alpha$, and generates $\mathbf{X} \cong \mathbb{Z}$.\\\\
  We write $t := t_\alpha$, as there 
  is now only one parameter.
  So, let us write $Y := Y^{\rho^\vee}$ and $X := X^\rho$. Moreover, the $q$-shift operator
  acts on $X$ by $\tau(\rho)X = qX$. Let $\pi_\rho$ denote the element of zero length.
  \cite[Corollary 3.3(i)]{kir97} tells us that that 
  $\ell(\tau(\rho)) = 2\langle \rho,\rho^\vee\rangle = 1$, and from \cite[Corollary 3.3(iv)]{kir97},
  we know that $\ell(\tau(\rho)s) = \ell(\tau(\rho)) - 1 = 0$. Thus, it follows that 
  $\pi_\rho := \tau(\rho)s$ is the element of zero length.\\\\
  As aforementioned in Theorem \ref{thm_y_lattice}(ii), $Y$ can be written in as 
  \begin{equation}\label{eqn_y_iwahori}
      Y = T_{\tau(\rho)} = \pi_\rho T_\alpha,
  \end{equation}
  using the Iwahori-Matsumoto presentation. We write $t := t_\alpha$, as there is only one simple root.
  The following Lemma shows that the $q$-shift operator $\tau(\rho)$ satisfies the commutation relation 
  outlined in the definition of the quantum torus of \cite{beg00}:
  \begin{lemma}\label{lem_comm_relation}
      Let $f(X) \in \C[q^\pm,t^\pm][X^\pm]$ be a rational function in $X$. Then,
      $$\tau(\rho) f(X) = f(qX) \tau(\rho).$$
  \end{lemma}
  
  \begin{proof}
      We have that $\tau(\rho)X = q^{2\langle \rho, \rho\rangle} X = qX$. Then, 
      \begin{align*}
          \tau(\rho)X\tau(\rho)^{-1} &= qX\tau(\rho)^{-1} = q X s\pi_\rho^{-1} = q s(sXs)\pi_\rho^{-1} = q^2 sX^{-1} \pi_\rho^{-1} = q^2 s\pi_\rho^{-1}(\pi_\rho X^{-1} \pi_\rho^{-1}).
      \end{align*}
      Computing explicitly, 
      $$\pi_\rho X^{-1} \pi_\rho^{-1} = \pi_\rho X^{-\rho} \pi_\rho^{-1} = X^{-\alpha_0/2} = q X,$$
      where the last equality follows from the fact that $\alpha = -\alpha + \delta$, and 
      \eqref{eqn_xdelta}.
      Then, we have the relation:
      $$\tau(\rho) X \tau(\rho)^{-1} = q \tau(\rho)^{-1} (qX) = qX,$$
      and thus:
      \begin{equation}\label{eqn_stuff}
          \tau(\rho)X = qX\tau(\rho).
      \end{equation}
      %$-\alpha_0/2 = \alpha/2 - \delta/2 = \rho - \delta/2,$ which we substitute 
      %into \eqref{eqn_stuff} to obtain:
      %$$\tau(\rho)X = X^{-\alpha_0}\tau(\rho) = qX\tau(\rho).$$ 
      Extending by linearity, it follows then that for any rational function $f(X) \in \C[q^\pm,t^\pm][X^\pm]$, 
      $$\tau(\rho)f(X) = f(qX)\tau(\rho),$$
      as claimed.
  \end{proof}
  
  \begin{lemma}
      The $q$-shift operator $\tau(\rho)$ has a faithful representation in 
      $\C[q^\pm,t_\alpha^\pm][X^\pm]$ given by:
      $$\tau(\rho)^{-1} = \left(\frac{1-X^2}{t-t^{-1}X^2} + \frac{t-t^{-1}}{t-t^{-1}X^2}\right)Y^{-1}.$$
  \end{lemma}
  
  \begin{proof}
      Recall from Theorem \ref{thm_demazure_lusztig} that $T_\alpha$ has a polynomial representation
      given by 
      \begin{equation}\label{eqn_T_poly_repn}
          T_\alpha = ts + (t-t^{-1})\frac{s-1}{X^{-2} - 1},
      \end{equation}
      For simplicity, let us write $T := T_\alpha$.
      We may re-write \eqref{eqn_T_poly_repn} as 
      $$T - t = t(s-1) + (t-t^{-1})\frac{s-1}{X^{-2}-1} = \left(t + \frac{t-t^{-1}}{X^{-2}-1}\right)(s-1) = \frac{tX^{-2} - t^{-1}}{X^{-2}-1}(s-1),$$
      and we obtain the formula 
      \begin{equation}\label{eqn_1-s}
          1-s = \frac{1 - X^{-2}}{tX^{-2}-t^{-1}} (T-t).
      \end{equation} 
      Now, using \eqref{eqn_y_iwahori} and \eqref{eqn_T_poly_repn}, 
      \begin{equation}\label{eqn_thingo}
          Y = \pi_\rho T = \tau(\rho)s \left(ts + (t-t^{-1})\frac{s-1}{X^{-2}-1}\right) = \tau(\rho) \left(t + (t-t^{-1})\frac{1-s}{X^2-1}\right).
      \end{equation}
      Then, we substitute \eqref{eqn_1-s} into \eqref{eqn_thingo} to obtain:
      \begin{align*}
          \tau(\rho)^{-1} &= \left(t + (t-t^{-1}) \frac{1-s}{X^2-1}\right)Y^{-1}\\
                          &= \left(t + (t-t^{-1}) \frac{X^{-2}(X^2-1)(T-t)}{(X^2-1)(tX^{-2}-t^{-1})}\right)Y^{-1}\\
                          &= \left(t + (t-t^{-1}) \frac{X^{-2}(T-t)}{tX^{-2}-t^{-1}}\right)Y^{-1}\\
                          &= \left( \frac{t(tX^{-2}-t^{-1}) + (t-t^{-1})X^{-2}(T-t)}{tX^{-2}-t^{-1}}\right)Y^{-1}\\
                          &=\left(\frac{t^2X^{-2} - 1 + tX^{-2}T -t^2X^{-2} - T^{-1}X^{-2}T + X^{-2}}{tX^{-2}-t^{-1}}\right)Y^{-1}\\
                          &= \left(\frac{(t-t^{-1})X^{-2}T + (X^{-2}-1)}{tX^{-2}-t^{-1}}\right)Y^{-1}\\
                          &= \left(\frac{t-t^{-1}}{t-t^{-1}X^2}T + \frac{1-X}{t-t^{-1}X^2}\right)Y^{-1}
  \end{align*}
      Thus,
      \begin{equation}\label{eqn_taurho}
          \tau(\rho)^{-1} = \left(\frac{1-X^2}{t-t^{-1}X^2} + \frac{t-t^{-1}}{t-t^{-1}X^2}\right)Y^{-1},
      \end{equation}
      as claimed.
  \end{proof}
  
  Let us give a one-dimensional example of how one may monodromy of $q$-difference equations arising 
  from DAHA. In this case, the matrix defining the $q$-difference system 
  is just a scalar matrix -- that is, it is a $1\times 1$ matrix. From this, 
  we wish to derive a gauge transformation that takes us from 
  $A(0)$ to $A(\infty)$. This gauge transformation will serve as our 
  monodromy matrix for this one-dimensional case.
  The logic is the same when we begin studying the monodromy of the 
  qKZ equation in the next section for a two-dimensional representation.
  
  \begin{example}[One-Dimensional Monodromy Matrix]
      Let $V$ be any finite-dimensional $\mathbf{\dot{H}}^\mathbf{Y}$-module.
      Then, 
      $$M := \on{Ind}_{\mathbf{\dot{H}}^\mathbf{Y}}^{\mathbf{\ddot{\mathbf{H}}}} V = \ddot{\mathbf{H}} \otimes_{\mathbf{\dot{H}}^\mathbf{Y}} V \cong V \otimes_\C \C[q^\pm][X^\pm],$$
      where the isomorphism follows from the PBW theorem for DAHA \eqref{eqn_tri_decomp}.
      As before, from (\ref{eqn_taurho}), we have the polynomial 
      representation given by:
      $$\tau(\rho)^{-1} = \left(\frac{1-X^2}{t-t^{-1}X^2} + \frac{t-t^{-1}}{t-t^{-1}X^2}T\right)Y^{-1}.$$
      We also have the polynomial representation for the simple reflection given by
      $$s(X) = \frac{1-X^2}{t-t^{-1}X^2}(T-t)+1.$$
      As before, given some $v\otimes f(X) \in M$, 
      $$v \otimes \tau(\rho)^{-1} f(X) = v\otimes f(q^{-1}(X)),$$ 
      but also 
      $$v \otimes \tau(\rho)^{-1} f(X) = v \otimes \left(\frac{1-X^2}{t-t^{-1}X^2} + \frac{t-t^{-1}}{t-t^{-1}X^2}T\right)Y^{-1} f(X).$$
      where the first equality follows from Lemma \ref{lem_comm_relation}.
      Apply $\tau(\rho)$ to both sides to obtain a second order $q$-difference equation 
      \begin{equation}\label{eqn_one_dim_qdiff}
         f(qX) = \frac{(t-t^{-1}X^2)Y}{(1-X^2) + (t-t^{-1})T} f(X).
      \end{equation}
      Let $A(X)$ be the coefficient of $f(X)$ in this $q$-difference equation.
      Suppose that the solution to \eqref{eqn_one_dim_qdiff} is of the form 
      Then, computing directly, we see that:
      $$A(0) = \frac{t}{1+(t-t^{-1})T}Y = tT^{-2}Y,$$ using the quadratic 
      relation for $T$, and $$A(\infty) = t^{-1}Y.$$
      The fundamental solutions around $0$ and $\infty$ are then given by 
      $e_{q,A(0)}$, and $e_{q,A(\infty)}$ in this case.
      The monodromy matrices $M^{(0)}$ and 
      $M^{(\infty)}$ are trivial in this case.
      We observe that $s(0) = t^{-1}T$, and $s(\infty) = t(T-t)+1$, and we claim 
      that $s(0)$ gives a suitable
      gauge transform that takes us from $A(0)$ to $A(\infty)^{-1}$.
      Computing directly:
      $$s(0) A(0) s(0)^{-1} = t^{-1}T(tT^{-2}Y)tT^{-1} = tT^{-1}YT^{-1} = tY^{-1} = A(\infty)^{-1},$$
      where we use the relation $Y^{-1} = T^{-1}YT^{-1}$. 
      Therefore, we see that $s(0)$ defines a gauge transformation from $A(0)$ to $A(\infty)$. 
      What this shows is that $s(X)$ corresponds to a map of $q$-difference modules $M \to s^\ast M$.
      This property holds for the qKZ equations, as we shall see later.
  \end{example}
  
  In the sections to come, we will consider the case where the $\mathbf{\ddot{\mathbf{H}}}$-module $M$ 
  is a standard module (see Chapter \ref{chap2}). In this case, 
  calculating the fundamental solutions, and the monodromy matrix become a much more non-trivial task.
  Further, in this case the simple reflection $s$ still plays the role of $q$-gauge 
  transforming $q$-difference modules $M$ to the module $s^\ast M$.
  
  \section{The Rank One qKZ Equation}
  
  The \emph{quantum Knizhnik-Zalomodchikov} (qKZ) equation 
  is given by \cite[Chapter 11.5, (11.27)]{efk98}: 
  \begin{equation}\label{eqn_qkz_eqn}
      \Phi(z_1,\cdots,qz_j,\cdots,z_N) = R_{j,j-1}\left(\frac{qz_j}{z_{j-1}}\right) \cdots R_{j1}\left(\frac{qz_j}{z_1}\right) q^{(2\lambda)_j} R_{jN}\left(\frac{z_j}{z_N}\right) \cdots R_{j,j+1}\left(\frac{z_j}{z_{j+1}}\right) \Phi(z_1,\cdots,z_N),
  \end{equation}
  This equation was first derived in \cite[Theorem 5.3]{fr92}, as a $q$-analogue of the 
  Knizhnik-Zamolodchikov equation for quantum affine algebras. Frenkel-Reshetikhin studied the 
  solutions and monodromy of \eqref{eqn_qkz_eqn}. Their solutions are given in \cite[Proposition 6.1]{fr92}, 
  and a general form for the connection matrix is given in \cite[(6.32)]{fr92}.\\\\
  We will consider consider the rank one case. We will follow \cite[\S 11, \S 12]{efk98}, and record 
  the solutions of the qKZ equation, and its connection matrix.
  From \cite[Example 9.6.5]{efk98}, given $V$ and $W$ two representations of 
  $\mathbf{U}_q(\widehat{\mathfrak{sl}_2})$, with weights $m,n$, respectively,
  we have in the case for $m=n=1$, that the $R$-matrix is given by 
  \cite[(9.41)]{efk98}:
  $$R(z) = E_{11}\otimes E_{11} + E_{22}\otimes E_{22} + \frac{1-z}{q-q^{-1}z}(E_{22}\otimes E_{11} + E_{11}\otimes E_{22}) + \frac{q-q^{-1}}{q-q^{-1}z}(E_{12}\otimes E_{21}+zE_{21}\otimes E_{12}).$$
  Written in matrix form, we get:
  $$R(z) = \renewcommand\arraystretch{1.5}\begin{pmatrix}
      1 & 0 & 0 & 0\\
      0 & \displaystyle\frac{1-z}{q-q^{-1}z} & \displaystyle\frac{z(q-q^{-1})}{q-q^{-1}z} & 0\\
      0 & \displaystyle\frac{q-q^{-1}}{q-q^{-1}z} & \displaystyle\frac{1-z}{q-q^{-1}z} & 0\\
      0 & 0 & 0 & 1
  \end{pmatrix}.$$
  The quantum Knizhnik-Zamolodchikov equation then becomes:
  $$\Phi(qz_1,z_2) = q^{(2\lambda)_1} R\left(\frac{z_1}{z_2}\right) \Phi(z_1,z_2),$$
  $$\Phi(z_1,qz_2) = R\left(\frac{z_1}{qz_2}\right)q^{(2\lambda)_2}\Phi(z_1,z_2),$$
  where $\Phi$ in this case takes values in $V\otimes W$, and 
  $q^{(2\lambda)_1}$ is a diagonal matrix whose entries depend on the weights of the 
  weight spaces. In particular,
  we restrict ourselves to the case where $\Phi$ takes values in the 
  $2$-dimensional weight space of $\mcal{U}_q(\widehat{\mathfrak{sl}_2})$ --- 
  that is, the weight space corresponding to $m+n-2$.
  This gives us the following matrix defining the 
  quantum Knizhnik-Zamolodchikov equation:
  $$\frac{q^{(2\lambda)_i}}{q-q^{-1}\displaystyle\frac{z_1}{z_2}} \renewcommand\arraystretch{2.5}\begin{pmatrix}
      1-\displaystyle\frac{z_1}{z_2} & \displaystyle\frac{z_1}{z_2}(q-q^{-1})\\
      q-q^{-1} & 1-\displaystyle\frac{z_1}{z_2}
  \end{pmatrix},\quad i =1,2.$$
  This corresponds to the coefficient matrix seen in \cite[(69)]{os22},
  which is a modification of the quantum Knizhnik-Zamolodchikov 
  equation given by
  $h_{(1)}^\lambda R(a_1/a_2)$, where --- according to \cite[(72)]{os22} --- 
  $h_{(1)}^\lambda$ should be given as a diagonal matrix whose entries 
  are some monomials in \emph{dynamical parameters} $z$ (see \cite{ao16}). Here, the variables 
  $a_1/a_2$ are called \emph{equivariant} parameters. 
  In particular, $$h_{(1)}^\lambda = \begin{pmatrix}
      z \\
      & z^{-1}
  \end{pmatrix}.$$
  From \cite[(122)]{os22}, the $R$-matrix is given as:
  $$R\left(\frac{a_1}{a_2}\right) = 
  \renewcommand\arraystretch{1.5}\begin{pmatrix}
      \displaystyle\frac{(1-a_1/a_2)\hbar^{1/2}}{\hbar - a_1/a_2} & \displaystyle\frac{a_1/a_2(\hbar - 1)}{\hbar - a_1/a_2}\\
      \displaystyle\frac{\hbar - 1}{\hbar - a_1/a_2} & \displaystyle\frac{(1-a_1/a_2)\hbar^{1/2}}{\hbar-a_1/a_2}
  \end{pmatrix}.
  $$
  So, the quantum Knizhnik-Zamolodchikov equations are given by:
  $$\Phi(z,qa_1,a_2) = R(a_1/a_2) \begin{pmatrix}
      z\\
      &z^{-1}
  \end{pmatrix} \Phi(z,a_1,a_2).$$
  To simplify our notation, we will write 
  $$u:= a_1/a_2.$$
  We keep in mind that $\Phi$ is a function of $z$, $a_1$, and $a_2$, where $a_i$
  are the equivariant parameters, and $z$ is the dynamical parameter. But by 
  abuse of notation we will write $\Phi(z,u)$. Thus we have:
  \begin{equation}\label{eqn_qkz}
      \begin{pmatrix}
          \Phi_1(z,qu)\\
          \Phi_2(z,qu)
          \end{pmatrix} = 
          \renewcommand\arraystretch{1.5}\begin{pmatrix}
              z\displaystyle\frac{(1-u)\hbar^{1/2}}{\hbar - u} & z\displaystyle\frac{u(\hbar - 1)}{\hbar - u}\\
              z^{-1}\displaystyle\frac{\hbar - 1}{\hbar - u} & z^{-1}\displaystyle\frac{(1-u)\hbar^{1/2}}{\hbar-u}
          \end{pmatrix}.
      \begin{pmatrix}
          \Phi_1(z,u)\\
          \Phi_2(z,u)
      \end{pmatrix},
  \end{equation}
  Note from Proposition \ref{prop_qkz_daha} 
  that the dynamical parameters arising in the qKZ equation come from a non-trivial one-dimensional representation of 
  $\C[\mathbf{Y}]$.
  \begin{remark}
      This is a $q$-analogue of the trigonometric Knizhnik-Zamolodchikov connection
      that is studied in \cite{vv04} ($\nabla_j$ in \cite[Lemma 3.1(ii)]{vv04}), 
      in the case of the degenerate double affine Hecke algebra.
      The rational double affine Hecke algebra case is treated in \cite{ggor03},
      in which case the Knizhnik-Zamolodchikov equation arises.
      The affine Knizhnik-Zamolodchikov was studied by Cherednik in \cite{che90}
      --- see in particular \cite[(21)]{che90} in relation to affine Hecke algebra
      representations. The hypergeometric equation arises as a solution to these
      equations, and its monodromy has been extensively well-studied.
  \end{remark}
  
  \begin{remark}
      Moreover, these equations also appear when one considers of elliptic stable envelopes,
      which are objects that are related to the equivariant elliptic cohomology 
      of Nakajima quiver varieties. These objects were considered in \cite{ao16}.
      In particular, the rank one qKZ equations arise as vertex operators in
      some localisation of $K(T^\ast\mathbb{P}^1)$ (see \cite[\S 6]{ao16}). 
      Under this identification, the variables $z$ and $u$ arise by considering the 
      torus-equivariant elliptic cohomology of a Nakajima variety, and are called \emph{dynamical} and 
      \emph{equivariant} variables, respectively (\cite[(27)]{ao16}).
  \end{remark}
  
  \section{From DAHA to the qKZ Equation}
  
  The following proposition shows that we can produce a two-dimensional DAHA
  representation of the $q$-shift operator $\tau(\rho)^{-1}$ that is 
  precisely the matrix defining the qKZ equation. This is an explicit
  realisation of Proposition \ref{prop_cat_o_Fuchsian}.
  
  \begin{proposition}\label{prop_qkz_daha}
      Let $\C_m$ be a one-dimensional $\C[Y^\pm]$-module acting by 
      $m\in \C^\times$, and consider the standard 
      module $$\delta_m = \on{Ind}_{\C[Y^\pm]}^{\mathbf{\dot{H}}^{\mathbf{Y}}} \C_m,$$
      and choose a standard ordered basis $\lbrace 1\otimes 1, T\otimes 1\rbrace$. 
      Then, the image of the $q$-shift operator $\tau(\rho)^{-1}$ in the 
      induced module of the standard module:
      $$\Delta(\delta_m) := \on{Ind}_{\mathbf{\dot{H}}^{\mathbf{Y}}}^{\mathbf{\ddot{\mathbf{H}}}} \delta_m,$$
      is of the form:
      $$\tau(\rho)^{-1} = R(X) \cdot \on{diag}(m,m^{-1}),$$
      where $R$ is the $R$-matrix of the evaluation module of 
      $\mcal{U}_q(\widehat{\mathfrak{sl}_2})$ (see Example \ref{ex_qKZ}). 
  \end{proposition}
  \begin{proof}
      Fix some $m \in \C^\times$, and define a one-dimensional 
      representation $\C_m$ where $Y$ acts by $m$. 
      Let us consider the induced representation
      $$\delta_m := \on{Ind}_{\C[Y^\pm]}^{\mathbf{\dot{H}}^Y} \C_m = \mathbf{\dot{H}}^Y \otimes_{\C[Y^\pm]} \C_m.$$
      Then, by the PBW theorem \eqref{eqn_tri_decomp}, we have that 
      $$\delta_m \cong \C[t^\pm][T] \otimes_\C \C_m.$$ 
      Choose a basis $\lbrace 1\otimes 1, T\otimes 1\rbrace$ 
      for $\delta_m$.
      Next, let us induce to the DAHA by taking
      $$\Delta(\delta_m) = \on{Ind}_{\mathbf{\dot{H}}^Y}^{\mathbf{\ddot{\mathbf{H}}}} \delta_m = \ddot{\mathbf{H}} \otimes_{\mathbf{\dot{H}}^Y} \delta_m \cong \C[t^\pm,q^\pm][X^\pm] \otimes_{\C[t^\pm,q^\pm]} \delta_m,$$
      where the isomorphism follows again from the PBW theorem for DAHA. 
      Then, using our equation for $\tau(\rho)^{-1}$ from 
      \eqref{eqn_taurho}, and computing directly, we have:
      \begin{align*}
          \tau(\rho)^{-1}(1\otimes 1) &= \frac{t-t^{-1}}{t-t^{-1}X^2} (T\otimes m^{-1}) + \frac{1-X^2}{t-t^{-1}X^2} \otimes m^{-1}\\
                                      &= m^{-1}\left(\frac{t-t^{-1}}{t-t^{-1}X^2}\right)(T \otimes 1) + m^{-1}\left(\frac{1-X^2}{t-t^{-1}X^2}\right)(1\otimes 1),
      \end{align*} and 
      \begin{align*}
          \tau(\rho)^{-1} (T\otimes 1) &= \left(\frac{1-X^2}{t-t^{-1}X^2} + \frac{t-t^{-1}}{t-t^{-1}X^2}T\right)Y^{-1}(T\otimes 1)\\
                                       &= \frac{1-X^2}{t-t^{-1}X^2}(TY-(t-t^{-1})Y)(1\otimes 1) + \frac{t-t^{-1}}{t-t^{-1}X^2}T(TY - (t-t^{-1})Y)(1\otimes 1)\\
                                       &= m\frac{1-X^2}{t-t^{-1}X^2}(T\otimes 1) + m\frac{(X^2-1)(t-t^{-1})}{t-t^{-1}X^2}(1\otimes 1)\\
                                       &+ m\frac{(t-t^{-1})\left( (t-t^{-1})T + 1 -(t-t^{-1})T\right)}{t-t^{-1}X^2}(1\otimes 1)\\
                                       &= m\frac{1-X^2}{t-t^{-1}X^2}(T\otimes 1) + m\frac{1}{t-t^{-1}X^2}\left( (X^2-1)(t-t^{-1}) + (t-t^{-1})\right)\\
                                       &= \frac{m(1-X^2)}{t-t^{-1}X^2}(T\otimes 1) + \frac{1}{t-t^{-1}X^2} (mX^2(t-t^{-1}))
      \end{align*}
      and thus the representation of $\tau(\rho)^{-1}$ is given by 
      \begin{equation}\label{eqn_taurho_matrix}
          \tau(\rho)^{-1} = \renewcommand\arraystretch{2.5}\begin{pmatrix}
              \displaystyle\frac{1-X^2}{t-t^{-1}X^2} & \displaystyle\frac{X^2(t-t^{-1})}{t-t^{-1}X^2}\\
              \displaystyle\frac{t-t^{-1}}{t-t^{-1}X^2} & \displaystyle\frac{1-X^2}{t-t^{-1}X^2}
              \end{pmatrix} \begin{pmatrix}
              m^{-1}\\
                       &m
          \end{pmatrix},
      \end{equation}
      as claimed.
  \end{proof}
  Indeed, setting $X^2 = u$, $t = \hbar^{1/2}$, and $m = z^{-1}$, we 
  see that after multiplying each entry by $\hbar^{1/2}$ on the nominator 
  and denominator we obtain: $$\tau(\rho)^{-1} = \renewcommand\arraystretch{2.5}\begin{pmatrix}
      z\displaystyle\frac{(1-u)\hbar^{1/2}}{\hbar - u} & z\displaystyle\frac{u(\hbar-1)}{\hbar - u}\\
      z^{-1} \displaystyle\frac{\hbar - 1}{\hbar - u} & z^{-1} \displaystyle\frac{(1-u)\hbar^{1/2}}{\hbar - u}
  \end{pmatrix} = R(u) \cdot \on{diag}(z,z^{-1}),$$
  which is precisely the coefficient matrix of the qKZ equation \eqref{eqn_qkz}.\\\\
  One may also compute the corresponding representation for $\tau(\rho)$. Note that this is not as simple 
  as taking the inverse of $\tau(\rho)^{-1}$, since $\tau(\rho)$ is not a linear operator over $\mathbf{\ddot{\mathbf{H}}}$.
  As such, one obtains a twist by a factor of $q$:
  
  \begin{lemma}\label{lem_taurho2}
      Let $A(X) = \tau(\rho)^{-1}$.
      Then, $\tau(\rho) = A(qX)^{-1}$.
  \end{lemma}
  
  \begin{proof}
      Let $C(X)$ denote the image of $\tau(\rho)$ under the $\mathbf{\ddot{\mathbf{H}}}$-module 
      $\Delta(\delta_m)$. Let $\mathbf{e}_1 = 1\otimes 1$, and $\mathbf{e}_2 = T\otimes 1$.
      Then, computing directly,
      \begin{align*}
          \mathbf{e}_1 &= \left(\tau(\rho)^{-1} \cdot \tau(\rho)\right) (e_1)\\ 
          &= \tau(\rho)^{-1} \left(C_{11}(X) \mathbf{e}_1 + C_{21}(X) \mathbf{e}_2\right)\\
          &= C_{11}(q^{-1}X) \tau(\rho)^{-1} \mathbf{e}_1 + C_{21}(q^{-1}X) \tau(\rho)^{-1} \mathbf{e}_2\\ 
          &= C_{11}(q^{-1}X) \left(A_{11}(X) \mathbf{e}_1 + A_{21}\mathbf{e}_2\right) + C_{21}(q^{-1}X) \left(A_{12}(X) e_1 + A_{22}(X) \mathbf{e}_2\right)\\
          &= \left(C_{11}(q^{-1}X)A_{11}(X) + C_{21}(q^{-1}X) A_{12}(X)\right)\mathbf{e}_1 + \left(C_{11}(q^{-1}X) A_{21}(X) + C_{21}(q^{-1}X) A_{22}(X)\right)\mathbf{e}_2,
      \end{align*}
      where the third equality follows from the commutation relation seen in 
      Lemma \ref{lem_comm_relation}.
      Similarly,
      \begin{align*}
          \mathbf{e}_2 &= \left(\tau(\rho)^{-1} \cdot \tau(\rho)\right)(\mathbf{e}_2) \\
          &= \tau(\rho)^{-1} \left(C_{12}(X) \mathbf{e}_1 + C_{22}(X) \mathbf{e}_2\right)\\ 
          &= C_{12}(q^{-1}X) \tau(\rho)^{-1} \mathbf{e}_1 + C_{22}(q^{-1}X) \tau(\rho)^{-1}\mathbf{e}_2\\
          &= C_{12}(q^{-1}X) \left(A_{11}(X) \mathbf{e}_1 + A_{21}(X) \mathbf{e}_2\right) + C_{22}(q^{-1}X) \left(A_{12}(X) \mathbf{e}_1 + A_{22}(X)\right) \mathbf{e}_2\\ 
          &= \left(C_{12}(q^{-1}X) A_{11}(X) + C_{22}(q^{-1}X) A_{12}(X) \right)\mathbf{e}_1 + \left(C_{12}(q^{-1}X) A_{21}(X) + C_{22}(q^{-1}X) A_{22}(X)\right)\mathbf{e}_2
      \end{align*}
      This calculations tells us that $$C(q^{-1}X)^T \cdot A(X)^T = I,$$
      where $I$ is the identity matrix. Shifting both sides by $q$ and then re-arranging shows 
      us that: 
      $$C(X) = A(qX)^{-1},$$
      as claimed.
  \end{proof}
  
  \section{Solutions of the qKZ Equation}
  The solutions of the qKZ equation have been well-studied, so no originality
  is claimed here. We follow \cite[Chapter 11, 12]{efk98} to produce 
  solutions around $0$ for the rank one qKZ equation. \\\\
  The equation \eqref{eqn_qkz} then gives us two $q$-difference equations
  \begin{equation}\label{eqn_vertex1}
      \Phi_1(z,qu) = z\frac{(1-u)\hbar^{1/2}}{\hbar-u} \Phi_1(z,u) + z\frac{u(\hbar-1)}{\hbar-u}\Phi_2(z,u),
  \end{equation}
  \begin{equation}\label{eqn_vertex2}
      \Phi_2(z,qu) = z^{-1}\frac{\hbar-1}{\hbar-u} \Phi_1(z,u) + z^{-1}\frac{(1-u)\hbar^{1/2}}{\hbar - u}\Phi_2(z,u).
  \end{equation}
  Applying the $q$-difference operator to (\ref{eqn_qkz}), we obtain a 
  second-order $q$-difference equation:
  $$
  \begin{pmatrix}
      \Phi_1(z,q^2u)\\
      \Phi_2(z,q^2u)
      \end{pmatrix} = 
      \renewcommand\arraystretch{1.5}\begin{pmatrix}
          z\displaystyle\frac{(1-qu)\hbar^{1/2}}{\hbar - qu} & z\displaystyle\frac{qu(\hbar - 1)}{\hbar - qu}\\
          z^{-1}\displaystyle\frac{\hbar - 1}{\hbar - qu} & z^{-1}\displaystyle\frac{(1-qu)\hbar^{1/2}}{\hbar-qu}
      \end{pmatrix}.
  \begin{pmatrix}
      \Phi_1(z,qu)\\
      \Phi_2(z,qu)
  \end{pmatrix},$$
  which gives us:
  $$\Phi_1(z,q^2u) = z\frac{(1-qu)\hbar^{1/2}}{\hbar - qu} \Phi_1(z,qu) + z \frac{qu(\hbar-1)}{\hbar - qu} \Phi_2(z,qu).$$
  We wish to write this as a second order $q$-difference equation with only 
  $\Phi_1$. It is sufficient to only do this for $\Phi_1$, since $\Phi_2$ 
  can be written as a linear combination of $\Phi_1(z,qu)$ and $\Phi_1(z,u)$.
  In particular, using \eqref{eqn_vertex1}, we can re-arrange this into: 
  \begin{equation}\label{eqn_operator_E}
      \Phi_2(z,u) = E\left(\Phi_1(z,u)\right) := \frac{\hbar-u}{\hbar - 1}\Phi_1(z,qu) - \frac{\hbar-1}{(1-u)\hbar^{1/2}} \Phi_1(z,u).
  \end{equation}
  Let us drop the indices on $\Phi_1$ and simply write $\Phi := \Phi_1$.
  This equation can be put into a second order $q$-difference equation (c.f. \cite[(11.33)]{efk98}):
  \begin{equation}\label{eqn_qkz_vertex}
      \begin{aligned}
      \left(qu-\hbar\right)\Phi(z,q^2u) + \hbar^{1/2}\left(-(qz+qz^{-1})u + qz+z^{-1} \right) \Phi(z,qu) + q\left(\hbar u - 1\right)\Phi(z,u)=0.
      \end{aligned}
  \end{equation}
  The $q$-hypergeometric equation 
  $$\pFq{2}{1}{q^a,q^b}{q^c}{q\,; u},$$ can be 
  put into a second order $q$-difference equation by:
  \begin{equation}\label{eqn_qhypergeom}
      (q^{a+b}u - q^{c-1})f(q^2z) + (-(q^a+q^b)u + q^{c-1} + 1)f(qu) + (u-1)f(u) = 0,
  \end{equation}
  (c.f. \cite{koe18}, \cite[(11.21)]{efk98}). 
  Generally, one can show that any second order $q$-difference equation in the form: 
  \begin{equation}\label{eqn_efk_qkz}
      (A_0z + B_0)f(q^2u) + (A_1z+B_1)f(qu) + (A_2z+B_2)f(u) = 0,
  \end{equation}
  reduces to 
  the $q$-hypergeometric equation \eqref{eqn_qhypergeom}.
  So, in the case of \eqref{eqn_qkz_vertex}, we have:
  $$A_0 = q,\quad B_0 = -\hbar,\quad A_1 = -\hbar^{1/2}q(z+z^{-1}),\quad B_1 = \hbar^{1/2}(qz + z^{-1}),\quad A_2 = q\hbar,\quad B_2 = -q.$$
  \subsection{Solutions Around $0$}
  From this, \cite[Proposition 11.4.2]{efk98} tells us how one may 
  produce solutions for this equation in a neighbourhood of $0$.
  
  \begin{theorem}[Proposition 11.4.2., \cite{efk98}]\label{thm_prop11.4.2}
      Let $s_i$, $a_i$, $b_i$, $c_i$, where $i=1,2$ be 
      the two distinct solutions of the system of equations
      \begin{align}
          &q^{2s}B_0 + q^sB_1 + B_2 = 0, \label{eq11.4.2.1}\\
          &q^{2s}\frac{A_0}{A_2} = q^{a+b}, \label{eq11.4.2.2}\\
          &q^s\frac{A_1}{A_2} = -(q^a+q^b), \label{eq11.4.2.3}\\
          &q^{2s}\frac{B_0}{B_2} = q^{c-1}, \label{eq11.4.2.4}.
      \end{align}
      Then, the functions 
      $$f(u) := u^{s_i} \pFq{2}{1}{q^{a_i},q^{b_i}}{q^{c_j}}{q;-ua_2/b_2}$$
      are a basis of fundamental solutions in a neighbourhood of 
      $0$.
  \end{theorem}
  
  It remains to solve the series of $4$ equations listed in Theorem 
  \ref{thm_prop11.4.2} for the coefficients $a$, $b$, $c$, and $s$ to find 
  solutions of the quantum Knizhnik-Zamolodchikov connection around $0$
  and $\infty$.\\\\
  Substituting into \eqref{eq11.4.2.1}, \eqref{eq11.4.2.2}, \eqref{eq11.4.2.3} 
  and\eqref{eq11.4.2.4}, we get:
  \begin{align}
      &-\hbar q^{2s} + \hbar^{1/2}(qz + z^{-1})q^s - q = 0,\label{eqn_poly1}\\
      &q^{2s}\hbar^{-1} = q^{a+b},\label{eqn_poly2}\\
      &q^s \hbar^{-1/2}(z + z^{-1}) = q^a+q^b,\label{eqn_poly3}\\
      &q^{2s} \hbar= q^c \label{eqn_poly4}.
  \end{align}
  Observe that the coefficients $s,a,b,c$ are uniquely defined up to a 
  permutation $(a,b)\mapsto (b,a)$. 
  Solving the first quadratic equation \eqref{eqn_poly1}
  gives us: $$s_1 = -\log_qz - \frac{1}{2}\log_q\hbar,\quad s_2 = 1+\log_qz - \frac{1}{2}\log_q\hbar.$$ Using \eqref{eqn_poly2},
  \eqref{eqn_poly3}, we find that there are two solutions for $q^{a_1}$, given by 
  $$q^{a_1} = \hbar^{-1}, \hbar^{-1}z^{-2}.$$
  But $q^a + q^b = \hbar^{-1} + \hbar^{-1}z^{-2}$, corresponding to the two possible solutions of $q^{a_1}$.
  As aforementioned, since the solution is unique up to a permutation
  $(a,b)\mapsto (b,a)$, it is sufficient to choose just one of these solutions
  for $a$ and $b$. Solving for $c_1$ then gives us: $c_1 = -2\log_qz$.
  Repeating the same for $s_2$, and $a_2,b_2,c_2$ then gives us solutions:
  \begin{align*}
      &s_1 = -\log_qz - \frac{1}{2}\log_q\hbar,\quad a_1 = -\log_q\hbar,\quad b_1 = -2\log_qz-\log_q\hbar,\quad c_1 = -2\log_qz,\\
      &s_2 = 1 + \log_qz - \frac{1}{2}\log_q\hbar, \quad a_2 = 1-\log_q\hbar,\quad b_2 = 1-\log_q\hbar + 2\log_qz,\quad c_2 = 2(1+\log_qz).
  \end{align*}
  Given a set of solutions for 
  $\Phi_1$, one automatically obtains a 
  set of solutions for $\Phi_2$ 
  since by \eqref{eqn_qkz}, we may write $\Phi_2$
  as a linear combination of $\Phi_1$.
  Thus, we have proved the following:
  \begin{proposition}\label{prop_soln_zero}
      The equations 
      $$\Phi^{(0)}_1 := \renewcommand\arraystretch{1.5}\begin{pmatrix}
          \Phi_{11}^{(0)}\\
          \Phi_{12}^{(0)}
          \end{pmatrix}, \quad \renewcommand\arraystretch{1.5}\Phi_2^{(0)} := \begin{pmatrix}  
          \Phi_{21}^{(0)}\\
          \Phi_{22}^{(0)}
      \end{pmatrix},$$
      where 
      $$\Phi_{11}^{(0)}(z,u) = u^{-\log_qz - 1/2\log_q\hbar} 
      \pFq{2}{1}{\hbar^{-1},\hbar^{-1}z^{-2}}{z^{-2}}{q\,;\hbar u},$$
      $$\Phi_{12}^{(0)}(z,u) = u^{1+\log_qz-1/2\log_q\hbar} 
      \pFq{2}{1}{q\hbar^{-1},q\hbar^{-1}z^{2}}{(qz)^2}{q\,;\hbar u},$$
      and
      $$\Phi_{21}^{(0)}(z,u) = E\left(\Phi_{11}^{(0)}\right),$$
      $$\Phi_{22}^{(0)}(z,u) = E\left(\Phi_{12}^{(0)}\right),$$
      gives a basis of asymptotic solutions of the quantum Knizhnik-Zamolodchikov equation
      $\Phi$ in a neighbourhood of $0$. Further,
      $$\mathbf{X}^{(0)} = \renewcommand\arraystretch{1.5}\begin{pmatrix}
          \Phi_{11}^{(0)}(z,u) &\Phi_{12}^{(0)}(z,u)\\
          \Phi_{11}^{(0)}(z,qu)& \Phi_{12}^{(0)}(z,qu)
      \end{pmatrix},$$
      is a fundamental solution of the quantum Knizhnik-Zamolodchikov equation
      in a neighbourhood of $0$.
  \end{proposition}
  
  
  For simplicity, let us write $\Phi_1^{(0)} := \Phi_{11}^{(0)}$, and $\Phi_{12}^{(0)} := \Phi_2^{(0)}$.
  Observe that functions of the form $u^{\log_q\hbar}$ has the same periodicity as 
  $$e_{q,\hbar} = \frac{\Theta_q(u)}{\Theta_q(\hbar^{-1}u)},$$
  and thus the following equations:
  $$\Phi_1^{(0)} = \frac{\Theta_q(u)}{\Theta_q(z\hbar^{1/2}u)} \pFq{2}{1}{\hbar^{-1},\hbar^{-1}z^{-2}}{z^{-2}}{q\,;\hbar u},$$
  $$\Phi_2^{(0)} = \frac{\Theta_q(u)}{\Theta_q(z^{-1}\hbar^{1/2}u)} \pFq{2}{1}{q\hbar^{-1},q\hbar^{-1}z^{2}}{(qz)^2}{q\,;\hbar u},$$
  also give a basis of solutions of the qKZ equation in a neighbourhood of $0$.\\\\
  \cite[\S 12.2]{efk98} gives a procedure for how one can produce a basis of 
  fundamental solutions in a neighbourhood of $\infty$. The procedure is similar
  to the one detailed above.
  As before, we obtain $q$-hypergeometric equations, whose coefficients are given by $u$ 
  raised the power of $\log_q$ applied to the eigenvalues of $A(\infty)$:
  
  \begin{proposition}[Proposition 12.2.1, \cite{efk98}]\label{prop_soln_infty}
      The functions 
      $$u^{\log_q\lambda_1} \cdot \pFq{2}{1}{q^a, q^{a-c+1}}{q^{a-b+1}}{q\,;q^{c+1-a-b}z^{-1}},$$
      $$u^{\log_q\lambda_2} \cdot \pFq{2}{1}{q^{b}, q^{b-c+1}}{q^{b-a+1}}{q\,;q^{c+1-a-b}z^{-1}},$$
      where $\lambda_i$ are the eigenvalues of the matrix $A(\infty)$, are a basis of solutions 
      of the quantum Knizhnik-Zamolodchikov equation $\Phi$ in a neighbourhood of $\infty$.
  \end{proposition}
  Here, the constants $a,b,c$ are the same ones seen in the $q$-hypergeometric equations appearing in 
  Proposition \ref{prop_soln_zero}.
  
  \section{Monodromy of the qKZ Equation}
  
  We begin by first studying the monodromy of the second order $q$-difference
  equation (\ref{eqn_qkz_vertex}), following the techniques outlined in
  \cite{sauloy03}. In particular, our technique here uses 
  \cite[Theorem 2.3.2.1]{sauloy03} to derive the monodromy matrix $M$ for the 
  qKZ equation. This matrix can be derived explicitly using more classical 
  results in the theory of $q$-difference equations (in particular, from
  \cite[Chapter 11-12]{efk98}), as we will see in the next section.
  \begin{lemma}
      The quantum Knizhnik-Zamolodchikov equation is 
      strictly Fuchsian at $0$ and $\infty$ in the sense of 
      \cite[\S 1.2.1]{sauloy03}.
  \end{lemma}
  \begin{proof}
      Computing directly:
      $$A(u) := R\left(u\right) \cdot \on{diag}(z,z^{-1})= 
      \renewcommand\arraystretch{2.5}\begin{pmatrix}
          z\displaystyle\frac{(1-u)\hbar^{1/2}}{\hbar - u} & z\displaystyle\frac{u(\hbar - 1)}{\hbar - u}\\
          z^{-1}\displaystyle\frac{\hbar - 1}{\hbar - u} & z^{-1}\displaystyle\frac{(1-u)\hbar^{1/2}}{\hbar-u}
      \end{pmatrix}.$$
      Then, we see that 
      $$A(0) = \renewcommand\arraystretch{2}\begin{pmatrix}
          z\hbar^{-1/2} & 0 \\
          \displaystyle z^{-1}\frac{\hbar-1}{\hbar} & z^{-1}\hbar^{-1/2}
      \end{pmatrix},$$
      which has determinant $\hbar^{-1} \neq 0$, and thus 
      $A(0) \in \on{GL}_2(\C)$. Similarly,
      $$A(\infty) = \renewcommand\arraystretch{1.5}\begin{pmatrix}
          z\hbar^{1/2} & z(1-\hbar)\\
          0 & z^{-1}\hbar^{1/2}
      \end{pmatrix},$$
      which has determinant $\hbar \neq 0$, and thus $A(\infty)\in\on{GL}_2(\C)$.
  \end{proof}
  
  Let
  $$\mathbf{X}^{(0)} = \renewcommand\arraystretch{1.5}\begin{pmatrix}
      \Phi_{11}^{(0)}(z,u) &\Phi_{12}^{(0)}(z,u)\\
      \Phi_{11}^{(0)}(z,qu)& \Phi_{12}^{(0)}(z,qu)
  \end{pmatrix}, \quad \mathbf{X}^{(\infty)} = \renewcommand\arraystretch{1.5}\begin{pmatrix}
      \Phi_{11}^{(\infty)}(z,u) & \Phi_{12}^{(\infty)}(z,u)\\
      \Phi_{11}^{(\infty)}(z,qu) & \Phi_{12}^{(\infty)}(z,qu)
  \end{pmatrix},$$
  denote the fundamental solutions around $0$ and $\infty$, respectively.
  For simplicity, we will write $\mcal{T}_{u}(-)$ for the operator that shifts 
  a variable $u$ by $q$.
  By Theorem \ref{thm_const_coeff_solns}, we know that these constant 
  coefficients can be written in the form
  $$\mathbf{X}^{(0)} = M^{(0)}e_{q,A(0)},\quad \mathbf{X}^{(\infty)} = M^{(\infty)}e_{q,A(\infty)},$$
  where $M^{(0)}$ and $M^{(\infty)}$ are the monodromy matrices satisfying a gauge
  transform relation. We wish to obtain an expression for these matrices.
  Observe that since $A(0)$ is lower triangular, and has two distinct entries on
  its diagonal, it is diagonalisable, and thus semi-simple. Thus, its 
  Jordan-Chevalley decomposition consists of just the semi-simple portion.
  Thus, 
  $$A(0) = S \renewcommand\arraystretch{1.5}\begin{pmatrix}
      z^{-1}\hbar^{-1/2} & 0\\ 
      0 & z\hbar^{-1/2}
  \end{pmatrix} S^{-1},$$
  is the Jordan-Chevalley decomposition for $A(0)$, where $$S = \renewcommand\arraystretch{1.5}\begin{pmatrix}
      0 & \displaystyle\frac{\hbar^{1/2}(z^{2}-1)}{\hbar - 1}\\
      1 & 1
  \end{pmatrix}.$$
  \begin{remark}
      Observe that the powers of $u$ $\Phi_{1}^{(0)}$ and $\Phi_{2}^{(0)}$ in Proposition \ref{prop_soln_zero} correspond 
      to the diagonal entries of the diagonalisation of $A(0)$.
      In particular, $q^{-\log_qz - 1/2\log_q\hbar} = z^{-1}\hbar^{-1/2}$,
      and $q^{\log_qz - 1/2\log_q\hbar} = z\hbar^{-1/2}$. 
  \end{remark}
  Then, applying the recipe from Section \ref{sec_qdiff_soln} gives us:
  \begin{align*}
      e_{q,A(0)} &= S \renewcommand\arraystretch{2.5}\begin{pmatrix}
          \displaystyle\frac{\Theta_q(u)}{\Theta_q(z\hbar^{1/2}u)} & 0\\
          0 & \displaystyle\frac{\Theta_q(u)}{\Theta_q(z^{-1}\hbar^{1/2}u)}
          \end{pmatrix} S^{-1}
  %               &= \renewcommand\arraystretch{3.5}\begin{pmatrix}
  %                   \displaystyle\frac{\Theta_q(u)}{\Theta_q(z\hbar^{-1/2}u)} & 0\\
  %        \displaystyle\frac{(\hbar-1)\left(\displaystyle\frac{\Theta_q(u)}{\Theta_q(z^{-1}\hbar^{1/2}u)} - \displaystyle\frac{\Theta_q(u)}{\Theta_q(z\hbar^{1/2}u)}\right)}{\hbar^{1/2}(z^2-1)} & \displaystyle\frac{\Theta_q(u)}{\Theta_q(z^{-}\hbar^{1/2}u)} 
  %    \end{pmatrix},
  \end{align*}
  Similarly, for $A(\infty)$, 
  $$A(\infty) = Q \renewcommand\arraystretch{1.5}\begin{pmatrix}
      z^{-1} \hbar^{1/2} & 0\\
      0 & z\hbar^{1/2}
  \end{pmatrix}Q^{-1},$$
  where $$Q = \renewcommand\arraystretch{2.5}\begin{pmatrix}
      \displaystyle\frac{(\hbar-1)z^2}{\hbar^{1/2}(z^2-1)} & 1\\
      1 & 0 
  \end{pmatrix}.$$
  So,
  $$e_{q,A(\infty)} = Q \renewcommand\arraystretch{2.5}\begin{pmatrix}
      \displaystyle\frac{\Theta_q(u)}{\Theta_q(z\hbar^{-1/2}u)} & 0\\
      0 & \displaystyle\frac{\Theta_q(u)}{\Theta_q(z^{-1}\hbar^{-1/2}u)} 
  \end{pmatrix} Q^{-1}.$$
  %$$e_{q,A(\infty)} = \renewcommand\arraystretch{2.5}\begin{pmatrix}
  %    \displaystyle\frac{\Theta_q(u)}{\Theta_q(z\hbar^{-1/2}u)} & 
  %    \displaystyle\frac{z^2(\hbar - 1)}{\hbar^{1/2}(z^2-1)} \cdot 
  %    \left(\displaystyle\frac{\Theta_q(u)}{\Theta_q(z^{-1}\hbar^{-1/2}u)} - \displaystyle\frac{\Theta_q(u)}{\Theta_q(z\hbar^{-1/2}u)}\right)\\
  %    0 & \displaystyle\frac{\Theta_q(u)}{\Theta_q(z^{-1}\hbar^{-1/2}u)}
  %\end{pmatrix}.$$
  
  From \cite{sauloy03}, we have a monodromy matrix 
  \begin{equation}\label{eqn_sauloy_monodromy_mat}
      M = \left( M^{(\infty)}\right)^{-1} M^{(0)} = e_{q,A(\infty)}\left(\mathbf{X}^{(\infty)}\right)^{-1}\mathbf{X}^{(0)} \left(e_{q,A(0)}\right)^{-1},
  \end{equation}
  where the matrix $\left(\mathbf{X}^{(\infty)}\right)^{-1}\mathbf{X}^{(0)}$ is called \emph{Birkhoff's connection matrix} 
  \cite[\S 1.2.3]{sauloy03}. The matrix $e_{q,A(0)}$ is a section of the following
  rank two vector bundle on the elliptic curve $E = \C^\times/q^{\mathbb{Z}}$:
  $$\mcal{F}_0 := \mcal{L}_1^{(0)} \oplus \mcal{L}_2^{(0)} := \mcal{O}(z^{-1}\hbar^{-1/2})\oplus \mcal{O}(z\hbar^{-1/2}).$$
  Similarly, $e_{q,A(\infty)}$ is a section of the rank two bundle on $E$:
  $$\mcal{F}_\infty := \mcal{L}_1^{(\infty)}\oplus \mcal{L}_2^{(\infty)} := \mcal{O}(z^{-1}\hbar^{1/2}) \oplus \mcal{O}(z\hbar^{1/2}).$$
  The monodromy matrix $M$ then corresponds to a meromorphic map of sheaves $\varphi : \mcal{F}_0 \dashrightarrow \mcal{F}_\infty$.
  %Using \cite[Theorem 2.3.2.1]{sauloy03}, the matrix $M$ should correspond to 
  %a map $$\varphi : \mcal{F}_0\longrightarrow \mcal{F}_\infty.$$
  %There is a canonical isomorphism of $\mcal{O}_E$-modules:
  %$$\mathscr{H}om_{\mcal{O}_E} (\mcal{F}_0,\mcal{F}_\infty) \cong \mcal{F}_0^\vee\otimes_{\mcal{O}_E}\mcal{F}_\infty.$$
  %Thus, $\varphi$ is a meromorphic section of $\mcal{F}_0^\vee \otimes_{\mcal{O}_E}\mcal{F}_\infty$. 
  %Observe that the two vector bundles are dual to one another --- that is, 
  %$\mcal{F}_0^\vee \cong \mcal{F}_\infty$. 
  %That is, $\varphi$ should be expressable as a $2\times 2$ matrix 
  %whose entries are given entirely in terms of Jacobi theta functions.
  %Specifically, $\varphi_{ij}$ should be a section of 
  %$(\mcal{L}^{(\times)}_i)^\vee \otimes_{\mcal{O}_E} \mcal{L}^{(\times)}_j$, where $i,j=1,2$,
  %and $\times =0,\infty$.
  %According to \cite[\S 3]{sauloy03}, this matrix should be equal to $M$.
  %We have that 
  %$$\mcal{F}_0^\vee \otimes_{\mcal{O}_E} \mcal{F}_\infty = \mcal{O}(\hbar)\oplus \mcal{O}(z^{-2}\hbar)\oplus \mcal{O}(z^2\hbar) \oplus \mcal{O}(\hbar).$$
  %So, we have:
  %\begin{equation}\label{eqn_predict_mon_mat}
  %    M = \renewcommand\arraystretch{2.5}\begin{pmatrix}
  %    c_1 \cdot \displaystyle\frac{\Theta_q(u)}{\Theta_q(\hbar^{-1}u)} & c_2 \cdot \displaystyle\frac{\Theta_q(u)}{\Theta_q(z^2\hbar^{-1}u)}\\
  %    c_3 \cdot \displaystyle\frac{\Theta_q(u)}{\Theta_q(z^{-2}\hbar^{-1}u)} & c_4 \cdot \displaystyle\frac{\Theta_q(u)}{\Theta_q(\hbar^{-1}u)}
  %\end{pmatrix},
  %\end{equation}
  %where the constants $c_i$ can be constants, or $q$-periodic functions. 
  %One can compute the $q$-periodic constants explicitly by using the following result:
  \begin{proposition}[Proposition 12.2.2, \cite{efk98}]\label{prop_12.2.2}
      The $q$-hypergeometric equation can be written in the form:
      $$\pFq{2}{1}{q^a,q^b}{q^c}{q\,;u} = \alpha(u) \Phi_{11}^{(\infty)}(u) + \beta(u)\Phi_{12}^{(\infty)}(u),$$
      where 
      $$\alpha(u) = \frac{\Gamma_q(c)\Gamma_q(b-a)}{\Gamma_q(b)\Gamma_q(c-a)} \cdot \frac{u^{-\log_q\lambda_1} \Theta_q(q^{-a}u)}{\Theta_q(u)},$$
      and 
      $$\beta(u) = \frac{\Gamma_q(c)\Gamma_q(a-b)}{\Gamma_q(a)\Gamma_q(c-b)} \cdot \frac{u^{-\log_q\lambda_2} \Theta_q(q^{-b}u)}{\Theta_q(u)},$$
      where $\Phi^{(\infty)}_{11}$ and $\Phi^{(\infty)}_{12}$ are a basis 
      of solutions in a neighbourhood of $\infty$.
      $\Gamma_q(a)$ is the \emph{$q$-Gamma function}.
  \end{proposition}
  
  \begin{proof}
      The \emph{$q$-Gamma function} is given by:
      \begin{equation}\label{eqn_qgamma}
          \Gamma_q(x) = (1-q)^{1-x} \frac{(q;q)_\infty}{(q^x;q)_\infty}.
      \end{equation}
      Then, \cite[(4.3.2)]{gr09}, \cite[(2.30)]{koe18} has the following formula:
      \begin{align*}
          \pFq{2}{1}{q^a, q^b}{q^b}{q\,; u} 
          &= \frac{(q^b;q)_\infty (q^{c-a};q)_\infty}{(q^c;q)_\infty (q^{b-a};q)_\infty} \cdot \frac{(q^au;q)_\infty (q^{1-a}u^{-1};q)_\infty}{(u;q)_\infty (qu^{-1};q)_\infty} \cdot \pFq{2}{1}{q^a, q^{a-c+1}}{q^{a-b+1}}{q\,; q^{c+1-a-b}u^{-1}} \\
          &+ \frac{(q^a;q)_\infty (q^{c-b};q)_\infty}{(q^c;q)_\infty (q^{a-b};q)_\infty} \cdot \frac{(q^b u;q)_\infty (q^{1-b}u^{-1};q)_\infty}{(u;q)_\infty (qu^{-1};q)_\infty} \cdot \pFq{2}{1}{q^b, q^{b-c+1}}{q^{b-a+1}}{q\,;q^{c+1-a-b}u^{-1}}.
      \end{align*}
      Then, using Proposition \ref{prop_soln_infty}, 
      and the Jacobi triple product identity, and \eqref{eqn_qgamma}:
      \begin{align*}
          \pFq{2}{1}{q^a, q^b}{q^b}{q\,; u} &= \frac{\Gamma_q(c)\Gamma_q(b-a)}{\Gamma_q(b)\Gamma_q(c-a)} \cdot u^{-\log_q\lambda_1} \frac{\Theta_q(q^{a}u^{-1})}{\Theta_q(u^{-1})} \cdot \Phi_1^{(\infty)} \\ 
          &+ \frac{\Gamma_q(c) \Gamma_q(a-b)}{\Gamma_q(a) \Gamma_q(c-b)} \cdot u^{-\log_q\lambda_2} \frac{\Theta_q(q^{b}u^{-1})}{\Theta_q(u^{-1})} \cdot \Phi_2^{(\infty)},
      \end{align*}
      and the result follows from using the fact that $$\frac{\Theta_q(a^{-1}u^{-1})}{\Theta_q(u^{-1})} = \frac{\Theta_q(au)}{\Theta_q(u)}.$$
  \end{proof}
  
  Let $\mathbf{X}$ be the connection matrix. Then, we have: 
  $$\Phi^{(0)}_i = \sum_j \mathbf{X}_{ij} \Phi^{(\infty)}_j.$$
  We may replace $u^{-\log_q\lambda_i}$ with the ratio $\frac{\Theta_q(\lambda_i^{-1}u)}{\Theta_q(u)}$. 
  Then, using Proposition \ref{prop_soln_zero}, Proposition \ref{prop_soln_infty}, and 
  Proposition \ref{prop_12.2.2}, we obtain the connection matrix:
  $$\mathbf{X} = \renewcommand\arraystretch{2.5}\begin{pmatrix}
      X_{11} \cdot \displaystyle\frac{\Theta_q(z\hbar^{-1/2}u)}{\Theta_q(z\hbar^{1/2}u)} \cdot \frac{\Theta_q(\hbar^2u)}{\Theta_q(\hbar u)} & X_{12} \cdot \displaystyle\frac{\Theta_q(z^{-1}\hbar^{-1/2}u)}{\Theta_q(z\hbar^{1/2}u)} \cdot \displaystyle\frac{\Theta_q(\hbar^2z^2u)}{\Theta_q(\hbar u)}\\
      X_{21} \cdot \displaystyle\frac{\Theta_q(z\hbar^{-1/2}u)}{\Theta_q(z^{-1}\hbar^{1/2}u)}\cdot \displaystyle\frac{\Theta_q(q\hbar^2z^{-2}u)}{\Theta_q(\hbar u)} & X_{22} \cdot \displaystyle\frac{\Theta_q(z^{-1}\hbar^{-1/2}u)}{\Theta_q(z^{-1}\hbar^{1/2}u)} \cdot \frac{\Theta_q(q\hbar^2u)}{\Theta_q(\hbar u)}
  \end{pmatrix},$$
  %$$\mathbf{X} = \renewcommand\arraystretch{2.5}\begin{pmatrix}
  %    X_{11} \cdot \displaystyle\frac{\Theta_q(u)}{\Theta_q(z\hbar^{1/2}u)} \cdot \frac{\Theta_q(u)}{\Theta_q(z^{-1}\hbar^{1/2}u)} \cdot \frac{\Theta_q(\hbar^2u)}{\Theta_q(\hbar u)} & X_{12} \cdot \displaystyle\frac{\Theta_q(u)}{\Theta_q(z\hbar^{1/2}u)} \cdot \frac{\Theta_q(u)}{\Theta_q(z\hbar^{1/2}u)} \cdot \frac{\Theta_q(\hbar^2z^2u)}{\Theta_q(\hbar u)}\\ 
  %    X_{21} \cdot \displaystyle\frac{\Theta_q(u)}{\Theta_q(z^{-1}\hbar^{1/2}u)} \cdot \frac{\Theta_q(u)}{\Theta_q(z^{-1}\hbar^{1/2}u)} \cdot \frac{\Theta_q(q\hbar^2z^{-2}u)}{\Theta_q(\hbar u)} & X_{22} \cdot \displaystyle\frac{\Theta_q(u)}{\Theta_q(z^{-1}\hbar^{1/2}u)} \cdot \frac{\Theta_q(u)}{\Theta_q(z\hbar^{1/2}u)}\cdot \frac{\Theta_q(q\hbar^2u)}{\Theta_q(\hbar u)}
  %\end{pmatrix},$$
  where 
  \begin{align*}
      X_{11} &= \frac{\left(\Gamma_q(-2\log_qz)\right)^2}{\Gamma_q(-\log_q\hbar - 2\log_qz)\Gamma_q(\log_q\hbar)},\\
      X_{12} &= \frac{\Gamma_q(-2\log_qz)\Gamma_q(2\log_qz)}{\Gamma_q(-\log_q\hbar) \Gamma_q(\log_q\hbar)},\\
      X_{21} &= \frac{\Gamma_q(2+2\log_qz) \Gamma_q(2\log_qz)}{\Gamma_q(1-\log_q\hbar - 2\log_qz) \Gamma_q(1 + \log_q\hbar + 2\log_qz)},\\ 
      X_{22} &= \frac{\Gamma_q(2 + 2\log_qz)\Gamma_q(-2\log_qz)}{\Gamma_q(1-\log_q\hbar) \Gamma_q(1 + \log_q\hbar)},
  \end{align*}
  and are treated as constants, since they only depend on $z$ and $\hbar$.
  Choose a basis such that the $e_{q,A(0)}$ and $e_{q,A(\infty)}$ are diagonal -- that is:
  \begin{align*}
      e_{q,A(0)} &= \renewcommand\arraystretch{2.5}\begin{pmatrix}
          \displaystyle\frac{\Theta_q(u)}{\Theta_q(z\hbar^{1/2}u)} & 0\\
          0 & \displaystyle\frac{\Theta_q(u)}{\Theta_q(z^{-1}\hbar^{1/2}u)}
          \end{pmatrix} 
  \end{align*}
  and 
  $$e_{q,A(\infty)} = \renewcommand\arraystretch{2.5}\begin{pmatrix}
      \displaystyle\frac{\Theta_q(u)}{\Theta_q(z\hbar^{-1/2}u)} & 0\\
      0 & \displaystyle\frac{\Theta_q(u)}{\Theta_q(z^{-1}\hbar^{-1/2}u)} 
  \end{pmatrix}.$$
  We can determine the monodromy matrix $M$ explicitly using the formula 
  $$M = e_{q,A(\infty)} \cdot \mathbf{X}^T \cdot \left(e_{q,A(0)}\right)^{-1},$$
  from which we obtain: 
  \begin{equation}\label{eqn_mon_matrix}
      M = \renewcommand\arraystretch{2.5}\begin{pmatrix}
          X_{11} \cdot \displaystyle\frac{\Theta_q(\hbar^2u)}{\Theta_q(\hbar u)} & X_{21} \cdot \displaystyle\frac{\Theta_q(q\hbar^2z^{-2}u)}{\Theta_q(\hbar u)} \\
          X_{12} \displaystyle\frac{\Theta_q(\hbar^2z^2u)}{\Theta_q(\hbar u)} & X_{22} \displaystyle\frac{\Theta_q(q\hbar^2u)}{\Theta_q(\hbar u)}
      \end{pmatrix},
  \end{equation}
  as the monodromy matrix for the qKZ equation. 
  \begin{remark}
      Note that the theta terms containing the dynamical parameters $z$ on the denominator are cancelled out. 
      This tells us that the only poles of the monodromy matrix $M$ are $\hbar^{-1}$. As we will see in the next section, the 
      morphism $\Delta_s : \mcal{F} \to s^\ast\mcal{F}$ from Lemma \ref{lem_module_over_ellaha} can be expressed 
      as the product of a constant matrix and the monodromy matrix -- that is, the zeroes and poles of $\Delta_s$ are controlled entirely by 
      $M$. It follows readily that the pole condition Lemma \ref{lem_module_over_ellaha}(iii) 
      is satisfied.\\\\
      Generally, it is possible 
      that the monodromy of any arbitrary DAHA module will contain poles at dynamical parameters, which could give 
      rise to representations of the dynamical AHA. We expect that it is possible to recover an ellAHA module from this 
      dynAHA module by "forgetting" the dynamical parameter. This informs our formulation of Conjecture \ref{main_thm3}.
  \end{remark}
  %\begin{equation}\label{eqn_mon_matrix}
  %    M = \renewcommand\arraystretch{2.5}\begin{pmatrix}
  %        X_{11} \cdot \displaystyle\frac{\Theta_q(u)}{\Theta_q(z\hbar^{-1/2}u)} \cdot \frac{\Theta_q(u)}{\Theta_q(z^{-1}\hbar^{1/2}u)} \cdot \frac{\Theta_q(\hbar^2u)}{\Theta_q(\hbar u)} & X_{21} \cdot \displaystyle\frac{\Theta_q(u)}{\Theta_q(z^{-1}\hbar^{1/2}u)} \cdot \frac{\Theta_q(u)}{\Theta_q(z^{-1}\hbar^{1/2}u)} \cdot \frac{\Theta_q(q\hbar^2z^{-2}u)}{\Theta_q(\hbar u)} \\
  %        X_{12} \cdot \displaystyle\frac{\Theta_q(u)}{\Theta_q(z\hbar^{1/2}u)} \cdot \frac{\Theta_q(u)}{\Theta_q(z\hbar^{1/2}u)} \cdot \frac{\Theta_q(\hbar^2z^2u)}{\Theta_q(\hbar u)} & X_{22} \cdot \displaystyle\frac{\Theta_q(u)}{\Theta_q(z^{-1}\hbar^{-1/2}u)} \cdot \frac{\Theta_q(u)}{\Theta_q(z\hbar^{1/2}u)} \cdot \frac{\Theta_q(q\hbar^2u)}{\Theta_q(\hbar u)}
  %    \end{pmatrix}.
  %\end{equation}
  
  %We can convert this back into the matrix we derived in \eqref{eqn_predict_mon_mat}
  %by using the following Lemma:
  %
  %\begin{lemma}\label{silly_lemma}
  %    Let $a,b,c \in \C^\times$. Then, there exists a $q$-periodic function $F(u)$ such that:
  %    $$\frac{\Theta_q(au) \cdot \Theta_q(bu)}{\Theta_q(cu)} = F(u) \cdot \Theta_q( (abc^{-1})u).$$
  %\end{lemma}
  %
  %\begin{proof}
  %    Let $$F(u) := \frac{\Theta_q(au) \cdot \Theta_q(bu)}{\Theta_q(cu)} \cdot \frac{1}{\Theta_q( (abc^{-1})u)}.$$
  %    Clearly, $F$ is $q$-periodic by construction, and thus a mermorphic section
  %    of $E$. 
  %\end{proof}
  %
  %As aforementioned, functions of the form $u^{\log_q\hbar}$ have the same $q$-periodicity as 
  %the theta functions:
  %$$\frac{\Theta_q(u)}{\Theta_q(\hbar^{-1} u)},\quad \frac{\Theta_q(\hbar u)}{\Theta_q(u)},$$ and thus may be substituted into 
  %$M$ without changing its $q$-periodicity. Upon making this substitution, we have:
  %$$M = \renewcommand\arraystretch{2.5}\begin{pmatrix}
  %    \displaystyle\frac{X_{11}}{\Theta_q(z\hbar^{-1/2}u)} & X_{21} \cdot \displaystyle\frac{\Theta_q(u)}{\Theta_q(qz^{-1}\hbar^{1/2})}\\
  %    X_{12} \cdot \displaystyle\frac{\Theta_q(u)}{\Theta_q(z\hbar^{1/2}u)} & X_{22} \cdot \displaystyle\frac{\Theta_q(u)}{\Theta_q(z^{-1}\hbar^{-1/2}u)}
  %\end{pmatrix}.$$
  %Applying Lemma \ref{silly_lemma}, one readily checks that \eqref{eqn_mon_matrix}
  %defines the same matrix as \eqref{eqn_predict_mon_mat}.
  
  \section{Rank One qKZ Functor}
  
  Let $s \in W \cong \mathbb{Z}/2\mathbb{Z}$ be the only non-trivial simple 
  reflection in the rank one Weyl group. As aforementioned, we obtain 
  two flat vector bundles $\mcal{F}_0$ and $\mcal{F}_\infty$ that are
  dual to one another. 
  Let $\qTor_{\mathfrak{S}_2}^{\on{fuch}}$ be the $\mathfrak{S}_2$-equivariant
  Fuchsian quantum torus. Then, the only non-trivial generator 
  $s\in \mathfrak{S}_2$ acts on the equivariant variables by 
  $s : u \mapsto u^{-1}$, and acts on the $q$-difference operator 
  $\sigma_q := \on{D}_q^{\rho^\vee}$ by $s(\sigma_q) : x \mapsto q^{-1}x$.\\\\
  Moreover, given a line bundle $\mcal{O}(z)$ over $E$
  for some divisor $z$, a section of $\mcal{O}(z)$ is given by Jacobi theta
  functions of the form $\frac{\Theta_q(u)}{\Theta_q(zu)}$. But since $\Theta_q$
  is an odd function, it satisfies the property that 
  \begin{equation}\label{eqn_self_dual}
      \frac{\Theta_q(u)}{\Theta_q(zu)} = \frac{\Theta_q(u^{-1})}{\Theta_q(z^{-1}u^{-1})}.
  \end{equation}
  It follows then that there is an isomorphism 
  $$\mcal{O}(z^{-1}) \cong s^\ast\mcal{O}(z).$$
  Thus, relative to a choice of basis, there are isomorphisms of vector bundles:
  $$\mcal{F}_0^\vee \cong s^\ast\mcal{F}_0,\quad \mcal{F}_\infty \cong s^\ast\mcal{F}_\infty.$$
  Let $M$ be the $\qTor_{\mathfrak{S}_2}^{\on{fuch}}$-module coming from 
  the $\mathbf{\ddot{\mathbf{H}}}_{\on{loc}}$-module in Proposition \ref{prop_qkz_daha}.
  Given such a module, we wish to explicitly describe the module $s^\ast M$.
  Then, given a basis for $M$, we can write an element as a $q$-difference
  system $\sigma_qV(x)A(x) = V(x)$, where $A \in \on{GL}_2(\C[X^\pm])$, and 
  $V$ is a column vector. Then, it follows that there exists a module
  $s^\ast M$ corresponding to the $q$-difference equation:
  \begin{equation}\label{eqn_dual_system}
      s(\sigma_q) V(u^{-1}) = A(qu^{-1})^{-1} V(u^{-1}),
  \end{equation}
  where $A(qu^{-1})^{-1}$ comes from the relation in DAHA: 
  $$s\tau(\rho)^{-1} s^{-1} = \tau(\rho),$$
  (c.f. Lemma \ref{lem_taurho2}).
  We also use the fact that $s$ acts on $V(u))$ by sending $V(u) \mapsto V(u^{-1})$.
  Taking the limit $x\to 0$ and $x\to\infty$, and using the theory of 
  \cite{sauloy03} gives us vector bundles $\mcal{F}_0^\vee$ and $\mcal{F}_\infty$
  over $E$ associated to the 
  $q$-difference system \eqref{eqn_dual_system}. Taking monodromy gives us a 
  morphism $\varphi^\vee : \mcal{F}_0^\vee \to \mcal{F}_\infty^\vee$.
  Equivalently, 
  $$\varphi^\vee = s^\ast\varphi : s^\ast\mcal{F}_0 \to s^\ast\mcal{F}_\infty,$$
  and since $\mcal{F}_0$ and $\mcal{F}_\infty$ are dual to one another,
  the $q$-monodromy morphism $s^\ast\varphi$ just gives the monodromy going 
  from local solutions around $\infty$ to local solutions around $0$.
  It thus follows that there is an isomorphism:
  $$B_s : M \stackrel{\simeq}{\longrightarrow} s^\ast M.$$
  This corresponds to the morphism on the level of the $\mathfrak{S}_2$-equivariant
  connection category:
  $$B_s : (\mcal{F}_0,\mcal{F}_\infty,\varphi) \longrightarrow s^\ast(\mcal{F}_0,\mcal{F}_\infty,\varphi),$$
  where $$s^\ast (\mcal{F}_0,\mcal{F}_\infty,\varphi) = (s^\ast\mcal{F}_\infty, s^\ast\mcal{F}_0, s^\ast\varphi^{-1}),$$
  and the map $B_s$ is given by the commutative diagram:
  \begin{equation}\label{eqn_Bs}
      \begin{tikzcd}
          \mathcal{F}_0 \arrow[d, "\varphi"', dashed] \arrow[r, "B_s^{(0)}", dashed] & s^\ast \mathcal{F}_\infty \arrow[d, "s^\ast \varphi^{-1}", dashed] \\
          \mathcal{F}_\infty \arrow[r, "B_s^{(\infty)}"', dashed]                    & s^\ast\mathcal{F}_0                                               
      \end{tikzcd}
  \end{equation}
  Let us define the maps $\Delta_s$ to be the following composition:
  $$\begin{tikzcd}
      \mcal{F}_0 \arrow[r, "\varphi"] \arrow[rr, "\Delta_s^{(0)}"', bend right] & \mcal{F}_\infty \arrow[r, "B_s^{(\infty)}"] & s^\ast\mcal{F}_0
  \end{tikzcd}$$
  and 
  $$\begin{tikzcd}
      \mcal{F}_\infty \arrow[r, "B_s^{(\infty)}"] \arrow[rr, "\Delta_s^{(\infty)}"', bend right] & s^\ast\mcal{F}_0 \arrow[r, "s^\ast\varphi"] & s^\ast\mcal{F}_\infty
  \end{tikzcd}$$
  Our goal now is to construct a suitable map $B_s^{(\infty)}$ that takes us from 
  $\mcal{F}_0$ to $s^\ast\mcal{F}_\infty$. Concretely, the existence of such a 
  map is equivalent to the existence of some some gauge transformation $B$ 
  taking us from $A(u)$ to $A(qu^{-1})^{-1}$, which corresponds to the map of 
  $\qTor_{\mathfrak{S}_2}^{\on{fuch}}$-modules $M \to s^\ast M$. This will then 
  elucidate $\Delta_s^{(0)}$, which will be given by $M \cdot B_s^{(\infty)}$, where by 
  abuse of notation we take $B_s^{(\infty)}$ to be a section of the sheaf morphism 
  $\mcal{F}_\infty \to s^\ast\mcal{F}_0$. We will then analyse the poles of 
  $\Delta_s^{(0)}$, and use this to prove Lemma \ref{lem_module_over_ellaha}, 
  which will show that $\mcal{F}_0$ is a module over $\mcal{H}^{\on{ell}}$.
  
  \begin{lemma}\label{lem_s_qkz}
      Let $B(X)$ be the image of the simple reflection $s$ in the localised standard module
      $\Delta(\delta_m)_{\on{loc}}$, written with respect to the ordered basis 
      $\lbrace 1\otimes 1, T\otimes 1\rbrace$. Then, $B(X)$ is of the form:
      $$B(X) = R(X)\begin{pmatrix}
          0 & 1\\
          1 & 0
      \end{pmatrix},$$
      where $R(X)$ is the trigonometric $R$-matrix.
  \end{lemma}
  \begin{proof}
      Using \eqref{eqn_dl}, we have: 
      \begin{equation}\label{eqn_s}
          s(X) = 
          \frac{1-X^2}{t-t^{-1}X^2}(T-t) + 1.
      \end{equation}
      Then, computing directly, we have:
      \begin{align*}
          \left(\frac{1-X^2}{t-t^{-1}X^2}(T-t) + 1\right) \otimes 1 &= \frac{1-X^2}{t-t^{-1}X^2} T\otimes 1 + \left(1- \frac{t(1-X^2)}{t-t^{-1}X^2}\right)1\otimes 1\\
          &= \frac{(t-t^{-1})X^2}{t-t^{-1}X^2} 1\otimes 1 + \frac{1-X^2}{t-t^{-1}} T\otimes 1,
      \end{align*}
      and \begin{align*}
          \left(\frac{1-X^2}{t-t^{-1}X^2}(T-t) + 1\right) &= \frac{1-X^2}{t-t^{-1}X^2} T^2 \otimes 1 +\left(1 - \frac{t(1-X^2)}{t-t^{-1}X^2}\right)T\otimes 1\\
          &= \frac{1-X^2}{t-t^{-1}X^2} ((t-t^{-1})T \otimes 1 + 1\otimes 1) + \frac{(t-t^{-1})X^2}{t-t^{-1}X^2} T\otimes 1\\
          &= \frac{1-X^2}{t-t^{-1}X^2} 1\otimes 1 + \frac{t-t^{-1}}{t-t^{-1}X^2} T\otimes 1,
      \end{align*}
      and so we obtain a matrix:
      \begin{equation}
          B(X) = \renewcommand\arraystretch{2.5}\begin{pmatrix}
              \displaystyle\frac{(t-t^{-1})X^2}{t-t^{-1}X^2} & \displaystyle\frac{1-X^2}{t-t^{-1}X^2}\\
              \displaystyle\frac{1-X^2}{t-t^{-1}X^2} & \displaystyle\frac{t-t^{-1}}{t-t^{-1}X^2}
          \end{pmatrix},
      \end{equation}
      which we observe is equal to $R(X) \cdot \begin{pmatrix}
          0 & 1\\
          1 & 0
      \end{pmatrix}$.
  \end{proof}
  Written using our notation for the qKZ equation, this is given by 
  $$B(u) = \renewcommand\arraystretch{2.5}\begin{pmatrix}
      \displaystyle\frac{(h-1)u}{h-u}  & \displaystyle\frac{(1-u)\hbar^{1/2}}{h-u}\\
      \displaystyle\frac{(1-u)\hbar^{1/2}}{\hbar - u} & \displaystyle\frac{\hbar - 1}{\hbar - u}
  \end{pmatrix}.$$
  One readily checks that $R(u)^{-1} = R(u^{-1})^T$. So, we have that $\begin{pmatrix}
      0 & 1\\
      1 & 0
  \end{pmatrix}\cdot B(u^{-1}) = R(u)^{-1}$. The following Lemma will show that 
  $B(u)$ is a gauge transform that takes us from $A(u)$ to $A(qu^{-1})^{-1}$. 
  That is, $B(u)$ corresponds precisely to the $\qTor_{\mathfrak{S}_2}^{\on{fuch}}$-module
  isomorphism $B_s : M \to s^\ast M$.
  \begin{corollary}\label{cor_B_gauge}
      The matrix $B(u)$ gives a $q$-gauge transformation of $A(u)$ to $A(qu^{-1})^{-1}$. That is, 
      $$B(u)^{-1} \cdot A(u) B(q^{-1}u) = A(qu^{-1})^{-1}.$$
  \end{corollary}
  \begin{proof}
      Computing directly,
      \begin{align*}
          B(u)^{-1} A(u) B(q^{-1}u) &= \begin{pmatrix}
              0 & 1\\
              1 & 0
          \end{pmatrix} \cdot R(u)^{-1} \cdot R(u) \cdot \begin{pmatrix}
              z & 0 \\
              0 & z^{-1}
          \end{pmatrix} \cdot R(q^{-1}u) \begin{pmatrix}
              0 & 1\\ 
              1 & 0
          \end{pmatrix}\\ 
          &= \begin{pmatrix}
              0 & 1\\
              1 & 0
          \end{pmatrix} \cdot \left( R(q^{-1}u) \begin{pmatrix}
              z & 0\\ 
              0 & z^{-1}
          \end{pmatrix} \right) \cdot \begin{pmatrix}
              0 & 1\\
              1 & 0
          \end{pmatrix}\\
          &= R(q^{-1}u)^T \begin{pmatrix}
              z^{-1} & 0\\ 
              0 & z
          \end{pmatrix}\\ 
          &= R(qu^{-1})^{-1} \cdot \begin{pmatrix}
              z^{-1} & 0\\
              0 & z
          \end{pmatrix}\\
          &= A(qu^{-1})^{-1},
      \end{align*}
      where the third equality follows from the fact that:
      $$\begin{pmatrix}
          0 & 1\\
          1 & 0
      \end{pmatrix}\begin{pmatrix}
          a & b\\
          c & d
      \end{pmatrix} \begin{pmatrix}
          0 & 1\\
          1 & 0
      \end{pmatrix}
      = \begin{pmatrix}
          d & c\\ 
          b & a
      \end{pmatrix}.$$
      The diagonal entries of $R(q^{-1}u)$ are equal, and so conjugating by this matrix 
      gives us preicsely $R(q^{-1}u)^T$. Conjugating $\on{diag}(z,z^{-1})$ by this matrix 
      gives us $\on{diag}(z^{-1},z)$, which is its inverse. The fourth equality 
      follows from the equality $R(u)^{-1} = R(u^{-1})^T$. The last equality is by 
      definition of $A(u)$ as $A(u) = R(u) \cdot \on{diag}(z,z^{-1})$.
  \end{proof}
  Corollary \ref{cor_B_gauge} shows that $B(u)$ is a $q$-gauge transformation in the 
  category of $\qTor_{\mathfrak{S}_2}^{\on{fuch}}$-modules. However, the following lemma 
  will additionally show that this gauge transformation relation also holds in the category 
  of $\mathbf{\ddot{\mathbf{H}}}_{\on{loc}}$-modules:
  \begin{lemma}\label{lem_gauge_trans}
      Let $B(X)$ be the $2\times 2$ matrix given by the image of the simple reflection $s$ 
      in the $\mathbf{\ddot{\mathbf{H}}}_{\on{loc}}$-module $\Delta(\delta_m)_{\on{loc}}$. 
      Then, $$B(X)^{-1} A(X) B(q^{-1}X) = A(q^{-1}X^{-1})^{-1}.$$
      That is, $B$ gives a gauge transformation from $A(X)$ to $A(qX^{-1})^{-1}$.
  \end{lemma}
  \begin{proof}
      We use the commutation relation from the localised DAHA:
      $$\tau(\rho)^{-1} s = s\tau(\rho),$$ 
      together with the equation for $\tau(\rho)$ seen in Lemma \ref{lem_taurho2}.
      We know that $\tau(\rho) = A(qX)^{-1}$, but let us write $C(X) = A(qX)^{-1}$ 
      for simplicity. Let $\mathbf{e}_1 := 1\otimes 1$, and $\mathbf{e}_2 := T\otimes 1$.
      Then, computing directly,
      \begin{align*}
          (s\cdot \tau(\rho))(\mathbf{e}_1) &= s (C_{11}(X) \mathbf{e}_1 + C_{21}(X) \mathbf{e}_2)\\
          &= C_{11}(X^{-1}) s \mathbf{e}_1 + C_{21}(X^{-1}) s \mathbf{e}_2\\ 
          &= C_{11}(X^{-1}) \left(B_{11}(X) \mathbf{e}_1 + B_{21}(X) \mathbf{e}_2\right) + C_{21}(X^{-1})\left(B_{12}(X) \mathbf{e}_1 + B_{22}(X) \mathbf{e}_2\right)\\ 
          &= \left(C_{11}(X^{-1}) B_{11}(X) + C_{21}(X^{-1}) B_{12}(X)\right) \mathbf{e}_1 + \left(C_{11}(X^{-1}) B_{21}(X) + C_{21}(X^{-1}) B_{22}(X)\right)\mathbf{e}_2,
      \end{align*}
      \begin{align*}
          (s\cdot \tau(\rho))(\mathbf{e}_2) &= s (C_{12}(X) \mathbf{e}_1 + C_{22}(X) \mathbf{e}_2)\\
          &= C_{12}(X^{-1}) s \mathbf{e}_1 + C_{22}(X^{-1}) s \mathbf{e}_2\\ 
          &= C_{12}(X^{-1}) \left(B_{11}(X) \mathbf{e}_1 + B_{21}(X) \mathbf{e}_2\right) + C_{22}(X^{-1})\left(B_{12}(X) \mathbf{e}_1 + B_{22}(X) \mathbf{e}_2\right)\\ 
          &= \left(C_{12}(X^{-1}) B_{11}(X) + C_{22}(X^{-1}) B_{12}(X)\right) \mathbf{e}_1 + \left(C_{12}(X^{-1}) B_{21}(X) + C_{22}(X^{-1}) B_{22}(X)\right)\mathbf{e}_2,
      \end{align*}
      \begin{align*}
          (\tau(\rho)^{-1} \cdot s) (\mathbf{e}_1) &= \tau(\rho)^{-1} (B_{11}(X) \mathbf{e}_1 + B_{21}(X) \mathbf{e}_2)\\
          &= B_{11}(q^{-1}X) \tau(\rho)^{-1} \mathbf{e}_1 + B_{21}(q^{-1}X) \tau(\rho)^{-1} \mathbf{e}_2\\ 
          &= B_{11}(q^{-1}X) \left(A_{11}(X) \mathbf{e}_1 + A_{21}(X) \mathbf{e}_2\right) + B_{21}(q^{-1}X) \left(A_{12}(X) \mathbf{e}_1 + A_{22}(X) \mathbf{e}_2\right)\\ 
          &= \left(B_{11}(q^{-1}X) A_{11}(X) + B_{21}(q^{-1}X) A_{12}(X)\right)\mathbf{e}_1 + \left(B_{11}(q^{-1}X) A_{21}(X) + B_{21}(q^{-1}X) A_{22}(X)\right)\mathbf{e}_2
      \end{align*}
      \begin{align*}
          (\tau(\rho)^{-1} \cdot s) (\mathbf{e}_2) &= \tau(\rho)^{-1} (B_{12}(X) \mathbf{e}_1 + B_{22}(X) \mathbf{e}_2)\\
          &= B_{12}(q^{-1}X) \tau(\rho)^{-1} \mathbf{e}_1 + B_{22}(q^{-1}X) \tau(\rho)^{-1} \mathbf{e}_2\\ 
          &= B_{12}(q^{-1}X) \left(A_{11}(X) \mathbf{e}_1 + A_{21}(X) \mathbf{e}_2\right) + B_{22}(q^{-1}X) \left(A_{12}(X) \mathbf{e}_1 + A_{22}(X) \mathbf{e}_2\right)\\ 
          &= \left(B_{12}(q^{-1}X) A_{11}(X) + B_{22}(q^{-1}X) A_{12}(X)\right)\mathbf{e}_1 + \left(B_{12}(q^{-1}X) A_{21}(X) + B_{21}(q^{-1}X) A_{22}(X)\right)\mathbf{e}_2
      \end{align*}
      Then, we see that we have relations 
      $C(u^{-1})^T B(u)^T = B(q^{-1}u)^T A(u)^T$. From Lemma \ref{lem_taurho2}, we know that 
      $C(X) = A(qX)^{-1}$, and so we have the relation:
      $$B(u) A(qu^{-1})^{-1} = A(u) B(q^{-1}u).$$
  \end{proof}
  Computing directly,
  $$B(0) = \renewcommand\arraystretch{2.5}\begin{pmatrix}
      0 & \hbar^{-1/2}\\
      \hbar^{-1/2} & \displaystyle\frac{\hbar - 1}{\hbar} 
  \end{pmatrix}, \quad B(\infty) = \renewcommand\arraystretch{2.5}\begin{pmatrix}
      1 - \hbar & \hbar^{1/2}\\ 
      \hbar^{1/2} & 0
  \end{pmatrix}.$$
  As a consequence of Corollary \ref{cor_B_gauge} or Lemma \ref{lem_gauge_trans},
  we have the relations:
  $$B(0)^{-1} A(0) B(0) = A(\infty)^{-1}, \quad B(\infty)^{-1} A(\infty) B(\infty) = A(0)^{-1}.$$
  Recall that $A(0)$ and $A(\infty)$ give rise to divisors (and thus line bundles) on $E$ 
  by diagonalising them. In particular, $A(0) = S\cdot  E_0 \cdot S^{-1}$, and $A(\infty) = Q\cdot  E_\infty \cdot Q^{-1}$,
  where $E_0$ and $E_\infty$ are the diagonal entries obtained from diagonlising $A(0)$ and 
  $A(\infty)$, respectively. Then, using the gauge transformation relation, we have:
  $$\left(Q^{-1}B(0)^{-1}S\right) \cdot E_0 \cdot \left(Q^{-1}B(0)^{-1}S\right)^{-1} = E_\infty^{-1},$$
  and thus $QB(0)^{-1} S$ corresponds to a section of a map $\mcal{F}_0 \to s^\ast \mcal{F}_\infty$. Thus, 
  \begin{equation}\label{eqn_Bs0}
      B_s^{(0)} = Q^{-1} B(0)^{-1} S = \renewcommand\arraystretch{2.5}\begin{pmatrix}
          0 & \displaystyle\frac{\hbar(z^2-1)}{\hbar - 1}\\ 
          \hbar^{1/2} & 2\hbar^{1/2}(1-z^2)
  \end{pmatrix}.
  \end{equation}
  Analogously, one sees that 
  \begin{equation}\label{eqn_Bsinfty}
      B_s^{(\infty)} = S^{-1} B(\infty)^{-1} Q = \renewcommand\arraystretch{2.5}\begin{pmatrix}
          \displaystyle\frac{2(\hbar - 1)}{\hbar} & \hbar^{-1/2}\\ 
          \displaystyle\frac{\hbar - 1}{\hbar(z^2-1)} & 0
  \end{pmatrix}.
  \end{equation}
  Indeed, one expects these matrices to be constant matrices, since $s^\ast\mcal{F}_\infty \cong \mcal{F}_\infty^\vee \cong \mcal{F}_0$.
  So, $B_s^{(0)}$ is actually a map $B_s^{(0)} : \mcal{F}_0 \to \mcal{F}_0$.
  However, these two matrices are inverses of one another, which implies that $B_s^{(0)} = \left(B_s^{(\infty)}\right)^{-1}$, 
  which is an unexpected relation.
  Then, it follows then that 
  \begin{equation}\label{eqn_deltas0}
      \Delta_s^{(0)} = M(u) \cdot B_s^{(\infty)}.
  \end{equation}
  This gives us:
  $$\Delta_s^{(0)} = \renewcommand\arraystretch{2.5}\begin{pmatrix}
      X_{11} \cdot \displaystyle\frac{2(\hbar - 1)}{\hbar} \cdot \frac{\Theta_q(\hbar^2u)}{\Theta_q(\hbar u)} + X_{21} \cdot \frac{\hbar - 1}{\hbar(z^2-1)} \cdot \frac{\Theta_q(q\hbar^2 z^{-2}u)}{\Theta_q(\hbar u)} & X_{11} \cdot \hbar^{-1/2} \cdot \displaystyle\frac{\Theta_q(\hbar^2u)}{\Theta_q(\hbar u)} + X_{21}\\
      X_{12} \cdot \displaystyle\frac{2(\hbar - 1)}{\hbar} \cdot \frac{\Theta_q(\hbar^2z^2u)}{\Theta_q(\hbar u)} + X_{22} \cdot \frac{\hbar-1}{\hbar(z^2-1)} \cdot \frac{\Theta_q(\hbar^2u)}{\Theta_q(\hbar u)} & X_{12} \cdot \hbar^{-1/2} \cdot \displaystyle\frac{\Theta_q(\hbar^2z^2u)}{\Theta_q(\hbar u)}
  \end{pmatrix}.$$
  %$$\Delta_s^{(0)} = \renewcommand\arraystretch{2.5}\begin{pmatrix}
  %    X_{11} \cdot \displaystyle\frac{2(\hbar - 1)}{\hbar} \cdot \displaystyle\frac{\Theta_q(\hbar^2u)}{\Theta_q(\hbar u)} + X_{21} \cdot \hbar^{-1/2} \cdot \frac{\Theta_q(\hbar^2z^2u)}{\Theta_q(\hbar u)} & X_{21} \displaystyle\frac{2(\hbar - 1)}{\hbar} \cdot \frac{\Theta_q(q\hbar^2z^{-2}u)}{\Theta_q(\hbar u)} + X_{22} \cdot \hbar^{-1/2} \cdot \frac{\Theta_q(\hbar^2u)}{\Theta_q(\hbar u)}\\ 
  %    X_{11} \cdot \displaystyle\frac{\hbar - 1}{\hbar(z^2-1)} \cdot \frac{\Theta_q(\hbar^2z^2u)}{\Theta_q(\hbar u)} & X_{22} \cdot \displaystyle\frac{\hbar - 1}{\hbar(z^2-1)} \cdot \frac{\Theta_q(q\hbar^2u)}{\Theta_q(\hbar u)}
  %\end{pmatrix}.$$
  \begin{proposition}
      \leavevmode
      \begin{itemize}
          \item[(i)] $\left(\pi_\ast\Delta_s^{(0)}\right)^2 = \on{id}_{\pi_\ast\mcal{F}_0}$,
          \item[(iii)] $\Delta_s^{(0)}$ has a pole of order at most $1$ along $T_{\alpha,\hbar}$.
      \end{itemize}
  \end{proposition}
  \begin{proof}
      \leavevmode
      \begin{itemize}
          \item[(i)] By construction, we have that $s^\ast\Delta_s^{(0)} \circ \Delta_s^{(0)} = \on{id}_{s^\ast\mcal{F}_0}$.
          Then, $\pi_\ast (s^\ast\Delta_s^{(0)} \circ \Delta_s^{(0)}) = \left(\pi_\ast \Delta_s^{(0)}\right)^2 = \on{id}_{\pi_\ast\mcal{F}_0}$.
          \item[(ii)] 
          From \eqref{eqn_mon_matrix}, the only poles of $M$ are $\hbar^{-1}$, which is of order $1$.
          Recall that we have a map 
          $$\chi_\alpha : E \otimes \mathbf{Y} \longrightarrow E,\quad u \otimes \mu^\vee \longmapsto u^{\langle \mu^\vee,\alpha \rangle}.$$
          The element $\hbar^{-1} \otimes (-\rho^\vee)$ is in $T_{\alpha,\hbar}$. Then, $\Delta_s^{(0)}$ 
          can only have poles of order at most $1$ along the divisor $T_{\alpha,\hbar}$.
          Note that the element $u\in \mathfrak{A}$ is identified to be $u \otimes \rho^\vee$, since 
          $\rho^\vee$ is the only fundamental coweight, and thus a basis of the coweight lattice $\mathbf{Y}$.
      \end{itemize}
  \end{proof}
  
  \begin{conjecture}\label{annoying}
      $\pi_\ast\Delta_s^{(0)} \vert_{T_\alpha} = -\pi_\ast\Delta_e^{(0)}\vert_{T_\alpha}$.
  \end{conjecture}
  We briefly outline a strategy that one may use to prove Conjecture \ref{annoying}.
  Recall from \eqref{eqn_Bs} that we have a commutative diagram:
  $$\begin{tikzcd}
      \mathcal{F}_0 \arrow[d, "\varphi"', dashed] \arrow[r, "B_s^{(0)}", dashed] & s^\ast \mathcal{F}_\infty \arrow[d, "s^\ast \varphi^{-1}", dashed] \\
      \mathcal{F}_\infty \arrow[r, "B_s^{(\infty)}"', dashed]                    & s^\ast\mathcal{F}_0                                               
  \end{tikzcd}$$ 
  Note that from \eqref{eqn_Bs0} and \eqref{eqn_Bsinfty} that 
  $B_s^{(0)} = \left(B_s^{(\infty)}\right)^{-1}$.
  Using the commutativity of the diagram \eqref{eqn_Bs}, we see that 
  $$\Delta_s^{(0)} = B_s^{(0)} \circ s^\ast \varphi^{-1} = B_s^{(\infty)} \circ \varphi.$$ 
  Note that $$\Delta_s^{(0)}\vert_{T_\alpha} = \left(B_s^{(0)} \circ s^\ast\varphi^{-1}\right)\vert_{T_\alpha} = B_s^{(0)} \circ s^\ast \varphi^{-1}\vert_{T_\alpha} = B_s^{(0)} \circ \varphi^{-1}\vert_{s\cdot T_\alpha} = \left(B_s^{(\infty)}\right)^{-1} \circ \varphi^{-1}\vert_{T_\alpha}.$$
  Using the fact that $B_s^{(0)} = \left(B_s^{(\infty)}\right)^{-1}$, we have:
  $$\Delta_s^{(0)} \vert_{T_\alpha} = \left(\varphi\vert_{T_\alpha} \circ B_s^{(\infty)}\right)^{-1} = \left(\varphi\vert_{T_\alpha} \circ B_s^{(\infty)}\right),$$
  and thus $(\Delta_s^{(0)}\vert_{T_\alpha})^2 = \on{id}$.
  Thus, we know that the eigenvalues of $\Delta_s^{(0)}\vert_{T_\alpha}$ are given by $\pm 1$. \\\\
  The issue now is to determine 
  these eigenvalues, but this turns out to be a non-trivial computation, and we leave it as a conjecture for the timebeing. We expect that there should be 
  some way to simplify the connection matrix so the eigenvalues can be easily calculated. In particular, these connection and monodromy matrices 
  are known to be related to elliptic $R$-matrices. Felder's $R$-matrix is given in \cite[(85)]{ao16} as:
  %$$R_{\on{standard}} = \renewcommand\arraystretch{2.5}\begin{pmatrix}
  %    \displaystyle\frac{\Theta_q(z\hbar) \Theta_q(u^{-1})}{\Theta_q(\hbar u^{-1}) \Theta_q(z)} & \displaystyle\frac{\Theta_q(zu) \Theta_q(\hbar)}{\Theta_q(\hbar u^{-1})\Theta_q(z)}\\
  %    \displaystyle\frac{\Theta_q(zu^{-1}) \Theta_q(\hbar)}{\Theta_q(\hbar u^{-1})\Theta_q(z)} & \displaystyle\frac{\Theta_q(\hbar z^{-1}) \Theta_q(u)}{\Theta_q(\hbar u^{-1})\Theta_q(z)}
  %\end{pmatrix}.$$ 
  $$R_{\on{standard}}(u) = \renewcommand\arraystretch{2.5}\begin{pmatrix}
      \displaystyle\frac{\Theta_q(z\hbar) \Theta_q(z\hbar^{-1}) \Theta_q(u)}{\Theta_q(z)^2 \Theta_q(\hbar^{-1} u)} & -\displaystyle\frac{\Theta_q(\hbar)\Theta_q(zu)}{\Theta_q(z) \Theta_q(\hbar^{-1}u)}\\ 
      -\displaystyle\frac{\Theta_q(\hbar) \Theta_q(zu^{-1})}{\Theta_q(z) \Theta_q(\hbar^{-1}u)} & \displaystyle\frac{\Theta_q(u)}{\Theta_q(\hbar^{-1}u)}
  \end{pmatrix}.$$ 
  The $2$-torsion points of the elliptic curve $E = \C^\times/q^{\mathbb{Z}}$ are given by
  $\pm 1$, and $\pm q^{-1/2}$.
  However, note that $M(u)$ is really a matrix $M(X^2)$, where $X = X^\rho$ in the fuchsian quantum torus 
  $\qTor^{\on{fuch}}_{\mathfrak{S}_2}$ (or, the rank one localised DAHA). Thus, choosing $X = \pm q^{1/2}$, we 
  see that $X^2 = q$, which is identified with $1$ in $E$. It follows then that the only $2$-torsion points we need 
  to consider for $M(u)$ is the case for which $u = 1$. Indeed, substituting $u = 1$ in $R_{\on{standard}}$ gives us:
  $$R_{\on{standard}}(1) = \begin{pmatrix}
      0 & -1\\ 
      -1 & 0
  \end{pmatrix},$$
  which has two distinct eigenvalues, given by $1,-1$.
  Conjecture \ref{annoying} will imply that $\pi_\ast \mcal{F}_0$ has the structure of a $\mcal{H}^{\on{ell}}$-module by 
  Lemma \ref{lem_module_over_ellaha}. Since $\mcal{F}_\infty \cong \mcal{F}_0^\vee$, it follows that 
  $\pi_\ast\mcal{F}_\infty \cong \pi_\ast\mcal{F}_0^\vee$ also has the structure of a $\mcal{H}^{\on{ell}}$-module.
  Thus, we have maps:
  $$\begin{tikzcd}
      \Delta(\delta_m)_{\operatorname{loc}} \arrow[rr, "\qRH", maps to] \arrow[rd, "\qKZ^{(0)}"'] &                             & {(\mcal{F}_0,\mcal{F}_\infty,\varphi)} \\
                                                                                            & \pi_\ast\mcal{F}_0 \arrow[ru, hook] &                                       
  \end{tikzcd}$$ 
  This will show the following:
  
  \begin{conjecture}\label{std_mod}
      Given a localised standard module $\Delta(\delta_m)_{\on{loc}} \in \mcal{O}_{\mathbf{\ddot{\mathbf{H}}}}$, let 
      $(\mcal{F}_0, \mcal{F}_\infty, \varphi)$ be its image under the $q$-Riemann-Hilbert functor 
      $\qRH$. Then, $\pi_\ast\mcal{F}_0$ and $\pi_\ast\mcal{F}_\infty$ have the structure of a flat 
      module over $\mcal{H}^{\on{ell}}$. Moreover, there exists unique functors $\qKZ^{(0)}$ and 
      $\qKZ^{(\infty)}$ factoring through $\qRH$ uniquely.
  \end{conjecture}
  
  \chapter{Further Directions}
  
  We expect to be able to prove Conjecture \ref{annoying}, given slightly more time. From this, Conjecture \ref{std_mod}
  would immediately follow, which is a weaker version of Conjecture \ref{main_thm1}.
  The expectation is that Conjecture \ref{main_thm1} should be easily deducable from Conjecture \ref{std_mod} 
  by applying a d\'evissage argument 
  using the $\Delta$-filtration structure on $\mcal{O}_{\mathbf{\ddot{\mathbf{H}}}}$. Following this, we expect that the 
  restriction of $\qKZ$ to the categorical quotient $\mcal{O}_{\mathbf{\ddot{\mathbf{H}}}}/\mcal{O}_{\ddot{\mathbf{H}}}^{\on{tor}}$ 
  should give a fully faithful and essentially surjective functor into the module category 
  $\coh^{\on{flat}}(\mcal{H}^{\on{ell}})$ -- this is outlined in Conjecture \ref{main_thm2}. By construction, we already know that $\qKZ$ restricted to 
  $\mcal{O}_{\mathbf{\ddot{\mathbf{H}}}}/\mcal{O}_{\ddot{\mathbf{H}}}^{\on{tor}}$ is faithful.\\\\ 
  The full conjecture we have relates this idea of monodromy to the setting of the dynamical affine Hecke algebra, 
  which is an object in $\coh(\mathfrak{A} \times \mathfrak{A}^\vee)$. One constructs this object by considering the 
  elliptic affine Hecke algebra equipped with a \emph{dynamical parameter} arising from $\mathfrak{A}^\vee$ (see Appendix \ref{app_D}).
  In the case of the standard module, one observes these dynamical parameters $z$ arising from the character of the non-trivial 
  one-dimensional representation of $\C[\mathbf{Y}]$.\\\\
  These dynamical parameters vanish when we compute the monodromy matrix $M$, which allows us to cleanly 
  apply Lemma \ref{lem_module_over_ellaha} in order to obtain an $\mcal{H}^{\on{ell}}$-module. 
  Generally, one should observe these dynamical parameters appearing in the poles of the monodromy matrix, 
  in which case we should be able to obtain a $\mcal{H}^\dyn$-module structure. Forgetting the dynamical parameter 
  should then give us a $\mcal{H}^{\on{ell}}$-module with zero-dimensional support -- this is outlined in Conjecture \ref{main_thm3}. From this, we expect to obtain a 
  commutative diagram:
  $$\begin{tikzcd}
      {\C[\mathbf{Y}]\lmod} \arrow[r, "\on{Induction}"] \arrow[d] & \mcal{O}_{\mathbf{\ddot{\mathbf{H}}}} \arrow[d, "\qKZ"]                                      \\
      \coh(\mcal{H}^\dyn) \arrow[rd, "\on{forget}"', bend right]  & \coh^{\on{flat}}(\mcal{H}^{\on{ell}}) \arrow[d, "\widehat{\mcal{FM}}"', bend right] \\
                                                                  & \coh^{\on{fin}}(\mcal{H}^{\on{ell}}) \arrow[u, "\mcal{FM}"', bend right]           
  \end{tikzcd}$$
  where ${}^L\mcal{H}^{\on{ell}}$ is the Langlands dual of $\mcal{H}^{\on{ell}}$, and $\mcal{FM}$ is the Fourier-Mukai transform constructed in \cite{LZZ23}.
  There is also an inverse Fourier-Mukai functor $\widehat{\mcal{FM}}$ mapping in the other direction. \\\\ 
  Indeed, the expectation is that the ideas in this thesis should be generalisable beyond the rank one 
  DAHA. The primary limitation that we experienced was the fact that the $q$-Riemann-Hilbert correspondence is 
  only known for $q$-difference systems of one variable; the multivariate case is still conjectural. 
  Otherwise, the techniques employed in this thesis would generalise easily to the general rank case. \\\\
  Vasserot in \cite{vas05} classified the irreducible, integrable representations of the DAHA using 
  perverse sheaves. The irreducible modules of $\mcal{H}^{\on{ell}}$ have also been classified using 
  the equivariant elliptic cohomology of the Springer resolution in \cite{ZZ21}. 
  Using these classifications, we expect to be able to relate irreducible objects in $\mcal{H}^{\on{ell}}$
  to non-torsion irreducible objects in $\mcal{O}_{\mathbf{\ddot{\mathbf{H}}}}$ using the qKZ functor. 
  Conjecture \ref{main_thm2} posits that the restriction of $\qKZ$ to the Serre quotient should be fully faithful 
  and essentially surjective, in which case the composition $\widehat{\mcal{FM}} \circ \qKZ\vert_{\mcal{O}_{\mathbf{\ddot{\mathbf{H}}}}/\mcal{O}_{\mathbf{\ddot{\mathbf{H}}}}^{\on{tor}}} : \mcal{O}_{\mathbf{\ddot{\mathbf{H}}}}/\mcal{O}_{\ddot{\mathbf{H}}}^{\on{tor}} \to \coh^{\on{fin}}({}^L\mcal{H}^{\on{ell}})$
  should give a one-to-one correspondence between non-torsion irreducible objects in $\mcal{O}_{\mathbf{\ddot{\mathbf{H}}}}$ 
  and irreducible ${}^L\mcal{H}^{\on{ell}}$-modules. \cite{ZZ21} showed that the irreducible representations of $\mcal{H}^{\on{ell}}$ 
  may be parametrised by certain nilpotent Higgs bundles. As a corollary, one would obtain a similar parametrisation of irreducible non-torsion 
  DAHA-modules by these nilpotent Higgs bundles.\\\\
  Our work also ties in to a larger body of work around $q$-difference equations and the elliptic cohomology of Nakajima varieties.
  Aganagic-Okounkov in \cite{ao16} constructed $q$-difference equations arising from the elliptic cohomology 
  of a Nakajima variety $X$. In the case for which $X = T^\ast\mathbb{P}^1$, one obtains the 
  qKZ equations. The solutions of the qKZ equations -- the $q$-hypergeometric equations -- then admit an interpretation via 
  vertex operators (see \cite[\S 6]{ao16}). The monodromy of these equations are constructed as a morphism on elliptic cohomology:
  $$\varphi : \mcal{E}\ell\ell (T^\ast\mathbb{P}^1) \longrightarrow \mcal{E}\ell\ell( (T^\ast\mathbb{P}^1)_{\on{flop}}).$$
  
  \bibliography{bibliography} 
  \bibliographystyle{alpha}
  
  \appendix 
  
  \chapter{Monodromy of Differential Equations} 
  
  Let $\mathbb{S} := \C \cup \lbrace \infty\rbrace$ be the Riemann sphere.
  Consider a linear differential equation of the form 
  \begin{equation}\label{eqn_linear_diffeq}
      \frac{dY}{dx} = A(x)Y,
  \end{equation}
  of holomorphic functions on some open domain $D$ of $\mathbb{S}$,
  and a fundamental system of solutions $Y(x)$. Then, one may define a loop
  $\gamma$ in $D$ with basepoint $x$. Analytically continuing the family of solutions 
  along $\gamma$
  gives rise to a new family of solutions --- call it $F(x)$ --- 
  satisfying $Y(x) = M_\gamma(x)F(x)$, where $M_\gamma \in \on{GL}_n(\C)$.
  Each homotopy class $[\gamma]$ uniquely determines such a $M_\gamma$, see 
  \cite[Theorem 3.3]{hara20}.
  The map $\gamma \mapsto M_\gamma$ defines a representation of the fundamental group $\pi_1(D,x)$, called the \emph{monodromy
  representation} associated to (\ref{eqn_linear_diffeq}) \cite[Definition 5.1]{hara20}.\\\\
  Let $X$ be a complex manifold, or a variety (that is, reduced, irreducible scheme of finite type) over $\C$. Let
  $\mcal{O}_X$ be its structure sheaf, and $\mcal{F}$
  be a vector bundle (that is, a locally free, coherent $\mcal{O}_X$-module) 
  on $X$. 
  %Then, 
  %a \emph{connection} on $X$ is a map:
  %$$\nabla : \mcal{F} \longmapsto \mcal{F} \otimes_{\mcal{O}_X} \Omega_{X/\C}^1.$$
  %satisfying the Leibniz rule: $$\nabla (sf) = ds \otimes f + s\nabla(f),$$
  %where $f$ and $s$ are local sections in $\mcal{F}$ and $\mcal{O}_X$, 
  Let $\mathbf{Vect}(X)^\nabla$ denote the category of vector bundles
  on $X$ equipped with a flat connection $\nabla$ (see Stacks Project 3.60.15).\\\\ 
  Given any local system $\mcal{L}$ on $\C$, the tensor product $\mcal{L}\otimes_\C \mcal{O}_X$ defines a vector bundle on $X$. 
  The $\C$-linear differential $d : \mcal{O}_X \to \Omega_{X/\C}^1$ then gives a 
  flat connection $d\otimes 1$ on $\mcal{L} \otimes_\C \mcal{O}_X$. Then, the \emph{Riemann-Hilbert
  correspondence} posits the following equivalence of categories: 
  $$\mathbf{LocSys}(X) \longrightarrow \mathbf{Vect}(X)^\nabla, \quad \mcal{L} \longmapsto (\mcal{L} \otimes_\C \mcal{O}_X, 1\otimes d).$$
  There is a quasi-inverse given by by the \emph{solution functor}
  $(\mcal{F},\nabla) \mapsto \ker \nabla$, where 
  $\ker \nabla$ is the sheaf of horizontal sections on $\mcal{F}$.
  Further, there is a \emph{monodromy functor} 
  $$\mathbf{LocSys}(X) \longrightarrow \pi_1(X,x), \quad \mcal{L}\longmapsto \mcal{L}_x,$$
  obtained by analytically continuing the stalks of $\mcal{L}$ along a loop in $X$.
  
  \chapter{Monodromy of Degenerate Double Affine Hecke Algebra Representations}\label{app_B}
  
  This idea of monodromy turns out to be an important tool in studying the 
  representations of various Hecke algebras. Affine Hecke algebras 
  \cite[Definition 3]{kir97} and degenerate affine Hecke algebras (\cite[Definition 1]{kir97}, \cite[Definition 4]{che90}) 
  both give rise to systems of differential equations, whose 
  monodromy can be studied.\\\\
  The case of the degenerate affine Hecke algebra
  case was studied by \cite{che90}, who studied the monodromy of the affine 
  Knizhnik-Zamolodchikov connection \cite[(21)]{che90}, 
  and showed that the aforementioned 
  monodromy functor factors through representations of the affine Hecke algebra 
  \cite[Theorem 10]{che90}.\\\\
  The double affine Hecke algebra (DAHA) $\mathbf{\ddot{\mathbf{H}}}$ \cite[Definition 4]{kir97}, was first introduced by \cite{che95}. From this, there are two degenerations that one can obtain. One level
  of degeneration gives us the degenerate (or trigonometric) DAHA \cite[(2.1.1)]{vv04}, and another level of degeneration gives us the rational DAHA \cite[Section 3.1]{ggor03}. Denote these resulting algebras by $\mathbf{\ddot{\mathbf{H}}}'$ and 
  $\mathbf{\ddot{\mathbf{H}}}''$, respectively. \\\\
  The $\mathbf{\ddot{\mathbf{H}}}'$ case is treated in \cite{vv04}
  Given some affine root datum \cite[Section 3]{kir97} $(X,X^\vee,\Phi,\Phi^\vee)$, define the torus $T := \mathbb{G}_{\on{m}}\otimes_\mathbb{Z} X^\vee$.
  Following \cite[Section 1]{gkv95}, for each $\lambda \in X$, there is a morphism of group schemes
  $T \to \mathbb{G}_{\on{m}}$ defined by $q \otimes x \mapsto q^{\langle \lambda,x\rangle}$. For some $\lambda \in X$, we denote by $T_\lambda$
  the kernel divisor of this map. Then, define $$T^{\on{reg}} := T\setminus \left(\bigcup_{\alpha \in \Phi} T_\alpha\right).$$
  We may localise the degenerate DAHA by taking $\mathbf{\ddot{\mathbf{H}}}'_{\on{reg}} := \ddot{\mathbf{H}}' \otimes_{\C[X]} \C[T^{\on{reg}}]$.
  Modules over $\mathbf{\ddot{\mathbf{H}}}$ 
  are localised in the same way: given a $\mathbf{\ddot{\mathbf{H}}}'$-
  module $M$, we define $M_{\on{reg}} := M \otimes_{\C[X]} \C[T^{\on{reg}}]$. 
  Let $\mcal{D}_{T^{\on{reg}}}$ denote the sheaf of differential operators on $T^{\on{reg}}$,whose 
  sections are the usual $D_{T^{\on{reg}}}$-modules (see \cite{hot98}, \cite{eti17}) . 
  There is a natural 
  $W$-action on $D_{T^{\on{reg}}}$, which allows us to form the semi-direct product
  $D_{T^{\on{reg}}} \rtimes \C[W]$.
  Then, there is a unique ring isomoprhism \cite[Lemma 3.1(ii)]{vv04}: 
  \begin{equation}\label{eqn_lemma3.1}
      D_{T^{\on{reg}}} \rtimes \C[W] \stackrel{\simeq}{\longrightarrow} \mathbf{\ddot{\mathbf{H}}}'_{\on{reg}},
  \end{equation}
  which maps the differential operator to the \emph{trigonometric Knizhnik-Zamolodchikov connection} ($\nabla_j$ in \cite[Lemma 3.1(ii)]{vv04}). 
  Composing the the localisation functor $\mcal{O}' \to \text{$\mathbf{\ddot{\mathbf{H}}}'_{\on{reg}}$-Mod}$, the isomorphism (\ref{eqn_lemma3.1}), and the sheafification
  functor $\textbf{$D_{T^{\on{reg}}}$-Mod} \to \textbf{$\mcal{D}_{T^{\on{reg}}}$-Mod}$, we obtain a functor 
  $\mcal{L} : \mcal{O}' \to \textbf{$\mcal{D}_{T^{\on{reg}}} \rtimes \C[W]$-Mod}$.
  We compose the solution functor from the Riemann-Hilbert correspondence
  with the monodromy functor from before to obtain: 
  $$\textbf{$\mcal{D}_{T^{\on{reg}}} \rtimes \C[W]$-Mod} \longrightarrow
  \mathbf{LocSys}(T^{\on{reg}}/W) \longrightarrow \textbf{$\pi_1(T^{\on{reg}}/W)$-Mod} \cong \textbf{$\on{Br}_{\on{aff}}$-Mod},$$
  where $\on{Br}_{\on{aff}}$ is the affine Braid group. 
  Composing $\mcal{L}$ together with the above functor then gives us the 
  \emph{monodromy functor}
  of a $\mathbf{\ddot{\mathbf{H}}}'$-module: $$\mcal{M}on : \mcal{O}' \longrightarrow \textbf{$\on{Br}_{\on{aff}}$-Mod}.$$
  The $\mathbf{\ddot{\mathbf{H}}}''$ case is treated by \cite{ggor03}.
  Define a new torus $T_{\on{a}} := \mathbb{G}_{\on{a}} \otimes_\mathbb{Z} X^\vee$. 
  The scheme $T_{\on{a}}^{\on{reg}}$ is defined using 
  kernel divisors $T_\alpha$ in the same way as before. As before, there is a unique 
  ring isomorphism \cite[Theorem 5.6]{ggor03} 
  $$D_{T_{\on{a}}^{\on{reg}}} \rtimes W \stackrel{\simeq}{\longrightarrow} \mathbf{\ddot{\mathbf{H}}}_{\on{reg}}.$$ 
  mapping differential operators on $T_{\on{a}}^{\on{reg}}$ to a $W$-equivariant
  connection on a $\mathbf{\ddot{\mathbf{H}}}''$-module $M$ \cite[Proposition 5.7]{ggor03}.
  The same argument from the prior section is used to produce a monodromy functor
  $$\mcal{O}'' \longrightarrow \textbf{$\on{Br}$-Mod},$$
  where $\on{Br}$ is now the usual braid group. In \cite{ggor03}, this functor
  is called the \emph{Knizhnik-Zamolodchikov} functor, and is denoted by $\on{KZ}$.
  
  \chapter{Convolution Structure on $\coh(\mathfrak{A} \times_{\mathfrak{A}/W}\mathfrak{A})$}\label{app_C}
  
  Let $\pi : \mathfrak{A} \to \mathfrak{A}/W$, and 
  $\pi_\dyn : \mathfrak{A}^\vee \to \mathfrak{A}^\vee/W^\dyn$ be the projection
  maps. We wish to define a new category: $$\coh(\mathfrak{A}\underset{\mathfrak{A}/W}{\times} \mathfrak{A}^\vee),$$
  in which the dynamic elliptic Hecke algebra lives. This will have the benefit
  of the pushforward map $\pi_\ast$ having some nicer properties. In particular,
  this category will be equipped with a monoidal structure that is similar
  to the convolution algebra construction originally given in the paper 
  of Kazhdan-Lusztig's proof of the Deligne-Langlands conjecture \cite{kl87} 
  (see also \cite{cg09}).\\\\
  Let us also consider projection maps:
  $$\pi \times \pi^\vee : \mathfrak{A} \underset{\mathfrak{A}/W}{\times} \mathfrak{A}^\vee \longrightarrow \mathfrak{A}/W \times \mathfrak{A}^\vee,$$ and 
  $$\pi \times \pi_\dyn : \mathfrak{A} \underset{\mathfrak{A}^\vee/W^\dyn}{\times} \mathfrak{A}^\vee \longrightarrow \mathfrak{A} \times \mathfrak{A}^\vee/W^\dyn.$$
  Now, we equip the category of coherent sheaves
  $$\coh(\mathfrak{A} \underset{\mathfrak{A}/W}{\times} \mathfrak{A}),$$
  with a monoidal structure in the following way. Let
  $$p_{ij} : \mathfrak{A} \underset{\mathfrak{A}/W}{\times} \mathfrak{A} \underset{\mathfrak{A}/W}{\times} \mathfrak{A} \longrightarrow \mathfrak{A} \underset{\mathfrak{A}/W}{\times} \mathfrak{A},$$
  denote the projection onto the $(i,j)$-th factor. 
  We also define a map:
  $$\Delta_2 : \mathfrak{A} \underset{\mathfrak{A}/W}{\times} \mathfrak{A} \underset{\mathfrak{A}/W}{\times} \mathfrak{A} \longrightarrow \mathfrak{A} \underset{\mathfrak{A}/W}{\times} \mathfrak{A} \underset{\mathfrak{A}/W}{\times} \mathfrak{A} \underset{\mathfrak{A}/W}{\times}\mathfrak{A},\quad (z_1,z_2,z_3) \longmapsto (z_1,z_2,z_2,z_3),$$
  where the subscript in $\Delta_2$ reminds us that the coordinate $z_2$ 
  is repeated. Then, the convolution product in $\coh(\mathfrak{A}\underset{\mathfrak{A}/W}{\times}\mathfrak{A})$ is defined to be:
  $$\mcal{F} \star \mcal{G} := \left(p_{13}\right)_\ast \Delta_2^\ast (\mcal{F} \boxtimes \mcal{G}).$$
  \begin{lemma}\label{lem_monoidal_functor}
      The functor $$\phi : \coh_W(\mathfrak{A}) \longrightarrow \coh(\mathfrak{A} \underset{\mathfrak{A}/W}{\times}\mathfrak{A}),\quad \mcal{F}_w \longmapsto \widetilde{w}_\ast\mcal{F}_w,$$
      is a monoidal functor.
      Here, $\widetilde{w}_\ast$ is the pushforward along the map 
      $$\widetilde{w} : \mathfrak{A} \longrightarrow \mathfrak{A} \underset{\mathfrak{A}/W}{\times} \mathfrak{A},\quad z\longmapsto (z,w^{-1}(z)).$$
  \end{lemma}
  
  \begin{proof}
      Let $\mcal{F}_w, \mcal{G}_v \in \coh_W(\mathfrak{A})$. We wish to 
      show that 
      $$\phi(\mcal{F}_w \star \mcal{G}_v) = \phi(\mcal{F}_w) \star \phi(\mcal{G}_v).$$
      On the left-hand side, we have:
      $$\phi(\mcal{F}_w \star \mcal{G}_v) = \phi(\mcal{F}_w \otimes (w^{-1})^\ast\mcal{G}_v) = (\widetilde{wv})_\ast \left(\mcal{F}_w \otimes (w^{-1})^\ast\mcal{G}_v\right) = (\widetilde{wv})_\ast \widetilde{w}^\ast (\mcal{F}_w \boxtimes \mcal{G}_v),$$
          and on the right-hand side,
          $$\phi(\mcal{F}_w) \star \phi(\mcal{G}_v) = \widetilde{w}_\ast \mcal{F}_w \star \widetilde{v}_\ast \mcal{G}_v = \left(p_{13}\right)_\ast \Delta_2^\ast(\widetilde{w}_\ast\mcal{F}_w \boxtimes \widetilde{v}_\ast\mcal{G}_v).$$
          We now show that these two relations coincide.
          $$\begin{tikzcd}
                                                   & \mathfrak{A} \arrow[r, "\widetilde{w}"] \arrow[d, "p"] \arrow[ld, "\widetilde{wv}"']                                                                             & \mathfrak{A} \underset{\mathfrak{A}/W}{\times}\mathfrak{A} \arrow[d, "\widetilde{w} \times \widetilde{v}"]                                                           \\
  \mathfrak{A} \underset{\mathfrak{A}/W}{\times}\mathfrak{A} & \mathfrak{A} \underset{\mathfrak{A}/W}{\times}\mathfrak{A}\underset{\mathfrak{A}/W}{\times}\mathfrak{A}\underset{\mathfrak{A}/W}{\times}\mathfrak{A} \arrow[l, "p_{13}"] \arrow[r, "\Delta_2"] & \mathfrak{A} \underset{\mathfrak{A}/W}{\times}\mathfrak{A}\underset{\mathfrak{A}/W}{\times}\mathfrak{A}\underset{\mathfrak{A}/W}{\times}\mathfrak{A} \underset{\mathfrak{A}/W}{\times}\mathfrak{A}
      \end{tikzcd}$$
      Here, we defined $$p(z) := (z,w^{-1}(z), (wv)^{-1}(z)),$$
      so $(\widetilde{wv})_\ast = (p_{13})_\ast p_\ast,$
      and $p_\ast\widetilde{v}^\ast = \Delta_2^\ast(\widetilde{w}\times\widetilde{v})$.
      Thus,
      \begin{align*}
          (\widetilde{wv})_\ast \widetilde{w}^\ast (\mcal{F}_w\boxtimes \mcal{G}_v) &= (p_{13})_\ast p_\ast \widetilde{w}^\ast (\mcal{F}_w \boxtimes \mcal{G}_v)\\
                                                                                    &= (p_{13})_\ast \Delta_2^\ast (\widetilde{w} \times \widetilde{v})_\ast (\mcal{F}_w \boxtimes \mcal{G}_v)\\
                                                                                    &= (p_{13})_\ast \Delta_2^\ast (\widetilde{w}_\ast\mcal{F}_w \boxtimes \widetilde{v}_\ast\mcal{G}_v),
      \end{align*}
      as claimed.
  \end{proof}
  
  The proof of the following is similar, and we omit the proofs:
  
  \begin{lemma}
      The following functors are monoidal, where the monoidal structures on the 
      domains and target are similar to the ones seen in Lemma \ref{lem_monoidal_functor}. By abuse of notation, we call all of these functors $\phi$.
      \begin{itemize}
          \item[(i)] $$\phi : \coh_{W\times W^\dyn} (\mathfrak{A} \times \mathfrak{A}^\vee) \longrightarrow \coh_{W^\dyn}(\mathfrak{A}_{\mathfrak{A}/W}\mathfrak{A}\times \mathfrak{A}^\vee), \quad \mcal{F}_{w,v^\dyn} \longmapsto \widetilde{w}_\ast\mcal{F}_{w,v^\dyn},$$
          \item[(ii)] 
              $$\phi : \coh_{W^\dyn} (\mathfrak{A} \underset{\mathfrak{A}/W}{\times} \mathfrak{A} \times \mathfrak{A}^\vee) \longrightarrow \coh(\mathfrak{A} \underset{\mathfrak{A}/W}{\times} \mathfrak{A} \times \mathfrak{A}^\vee \underset{\mathfrak{A}^\vee/W^\dyn}{\times} \mathfrak{A}^\vee),\quad \mcal{F}_v \longmapsto (\widetilde{v^\dyn})_\ast \mcal{F}_v.$$
      \end{itemize}
  \end{lemma}
  
  \begin{lemma}\label{lem_lax_monoidal}
      The pushforward functors are lax monoidal, with the target 
      equipped with the usual tensor structure of coherent sheaves:
      \begin{itemize}
          \item[(i)] $\coh(\mathfrak{A} \underset{\mathfrak{A}/W}{\times} \mathfrak{A}) \longrightarrow \coh(\mathfrak{A}/W)$,
          \item[(ii)] $\coh(\mathfrak{A} \underset{\mathfrak{A}/W}{\times} \mathfrak{A} \times \mathfrak{A}^\vee \underset{\mathfrak{A}^\vee/W^\dyn}{\times} \mathfrak{A}^\vee) \longrightarrow \coh(\mathfrak{A}/W \times \mathfrak{A}^\vee \times_{\mathfrak{A}/W^\dyn} \mathfrak{A}^\vee)$,
          \item[(iii)] $\coh_{W^\dyn}(\mathfrak{A} \underset{\mathfrak{A}/W}{\times} \mathfrak{A} \times \mathfrak{A}^\vee) \longrightarrow \coh_{W^\dyn}(\mathfrak{A}/W \times \mathfrak{A}^\vee)$.
      \end{itemize}
  \end{lemma}
  
  %Let us consider the pushforward $\pi_\ast\phi : \coh_W(\mathfrak{A}) \to \coh(\mathfrak{A}/W)$, and the map
  %$$\pi_\ast \phi(\mcal{F}_w) \otimes \phi_\ast\phi(\mcal{G}_v) \longrightarrow \pi_\ast\phi(\mcal{F}_w \star \mcal{G}_v).$$
  %Given $\mcal{F}_w$ and $\mcal{G}_v$ of degree $w$, $v$ in $\coh_W(\mathfrak{A})$,
  %we have that:
  %$$\pi_\ast(\phi(\mcal{F}_w) \star \phi(\mcal{G}_v)) = \pi_\ast\phi(\mcal{F}_w\star \mcal{G}_v) = \pi_\ast\phi(\mcal{F}_w\otimes (w^{-1})^\ast \mcal{G}_v).$$
  %With this construction, we can recover the elliptic twisted group algebra 
  %$\mcal{O}[W]$ of \cite{gkv95}. 
  
  \section{Localisation to $\mathfrak{A}^{\on{reg}}$}
  
  Construct the root hyperplanes $T_\alpha$, divisors $T_{\alpha,\hbar}$,
  and $\mathfrak{A}^{\on{reg}}$ as before. Let $j : \mathfrak{A}^{\on{reg}} \to\mathfrak{A}$ be the embedding. Similarly, we also have co-root hyperplanes
  given by $T_{\alpha^\vee}$, and divisors $T_{\alpha^\vee,\hbar}$ of 
  $\mathfrak{A}^\vee$. Define $(\mathfrak{A}^\vee)^{\on{reg}}$ similarly,
  and let $j^\vee  : (\mathfrak{A}^\vee)^{\on{reg}} \to \mathfrak{A}^\vee$
  denote the embedding. The open subsets $\mathfrak{A}^{\on{reg}}$ and 
  $(\mathfrak{A}^\vee)^{\on{reg}}$ are invariant under the actions of 
  $W$ and $W^\dyn$, respectively. The categories $\coh_W(\mathfrak{A}^{\on{reg}})$
  and $\coh(\mathfrak{A}^{\on{reg}} \underset{\mathfrak{A}/W}{\times} \mathfrak{A}^{\on{reg}})$ are equipped with induced monoidal structures, and are furthermore 
  categorically equivalent by the freeness of $W$ and $W^\dyn$:
  $$\coh_W(\mathfrak{A}^{\on{reg}}) \cong \coh(\mathfrak{A}^{\on{reg}} \underset{\mathfrak{A}/W}{\times} \mathfrak{A}^{\on{reg}}).$$
  The induced monoidal structures make the localisation functors 
  $$\coh_W(\mathfrak{A}) \longrightarrow \coh_W(\mathfrak{A}^{\on{reg}}),$$
  and
  $$\coh(\mathfrak{A} \underset{\mathfrak{A}/W}{\times} \mathfrak{A}) \longrightarrow \coh(\mathfrak{A}^{\on{reg}} \underset{\mathfrak{A}/W}{\times} \mathfrak{A}^{\on{reg}}),$$
  monoidal. Together, we have a commutative diagram:
  $$\begin{tikzcd}
  \coh_W(\mathfrak{A}^{\on{reg}}) \arrow[r, "\cong"]                & \coh(\mathfrak{A}^{\on{reg}} \underset{\mathfrak{A}/W}{\times} \mathfrak{A}^{\on{reg}})       \\
  \coh_W(\mathfrak{A}) \arrow[u, "\on{localise}"] \arrow[r, "\phi"] & \coh(\mathfrak{A} \underset{\mathfrak{A}/W}{\times} \mathfrak{A}) \arrow[u, "\on{localise}"']
  \end{tikzcd}$$
  We also have the following commutative diagram:
  $$\begin{tikzcd}
  \coh_W(\mathfrak{A}^{\on{reg}} \times (\mathfrak{A}^\vee)^{\on{reg}}) \arrow[r, "\cong"]                   & \coh(\mathfrak{A}^{\on{reg}} \underset{\mathfrak{A}/W}{\times} \mathfrak{A}^{\on{reg}} \times (\mathfrak{A}^\vee)^{\on{reg}} \underset{\mathfrak{A}^\vee/W^\dyn}{\times} (\mathfrak{A}^\vee)^{\on{reg}}) \\
  \coh_{W \times W^\dyn}(\mathfrak{A} \times \mathfrak{A}^\vee) \arrow[u, "\on{localise}"] \arrow[r, "\phi"] & \coh(\mathfrak{A} \underset{\mathfrak{A}/W}{\times} \mathfrak{A} \times \mathfrak{A}^\vee \underset{\mathfrak{A}^\vee/W^\dyn}{\times} \mathfrak{A}^\vee) \arrow[u, "\on{localise}"']                    
  \end{tikzcd}$$
  From this, one may recover the elliptic twisted group algebra 
  $\mcal{O}[W]$ of \cite{gkv95} by 
  $$\pi_\ast \phi(\mcal{O}_{\mathfrak{A}^{\on{reg}} \underset{\mathfrak{A}/W}{\times} \mathfrak{A}^{\on{reg}}}) \cong \pi_\ast \mcal{O}_{\mathfrak{A}^{\on{reg}}} \rtimes \C[W] = \mcal{O}[W],$$
  from which we construct the ellAHA $\mcal{H}^{\on{ell}}$ 
  as as subsheaf of $j_\ast \mcal{O}[W]$ satisfying some residue 
  conditions.
  
  \section{Convolution Construction of $\coh(\mcal{H}^\dyn)$ and $\coh(\mcal{H}^{\on{ell}})$}
  
  The monoidal category $\coh(\mathfrak{A} \underset{\mathfrak{A}/W}{\times} \mathfrak{A})$
  acts on $\coh(\mathfrak{A})$ via convolution, similar to the aforementioned 
  convolution category construction. We outline this construction here.
  Let $p_i : \mathfrak{A} \times \mathfrak{A} \to \mathfrak{A}$ 
  denote the projection map onto the $i$-th factor, and 
  $\Delta_2 : \mathfrak{A} \times \mathfrak{A} \to \mathfrak{A} \times \mathfrak{A} \times \mathfrak{A}$ by $(u_1,u_2) \mapsto (u_1,u_2,u_2)$. 
  Then, given a coherent sheaf $\mcal{F}$ on $\mathfrak{A}$, and 
  $\mcal{G}$ on $\mathfrak{A} \underset{\mathfrak{A}/W}{\times} \mathfrak{A}$, 
  define $$\mcal{G} \star \mcal{F} := (p_1)_\ast \Delta_2^\ast \left(\mcal{G} \boxtimes \mcal{F}\right).$$
  In this case, we say that $\coh(\mathfrak{A})$ has the structure of a 
  \emph{(categorical) $\coh(\mathfrak{A}\underset{\mathfrak{A}/W}{\times} \mathfrak{A})$-module}.
  Recall that the ellAHA $\mcal{H}^{\on{ell}}$ is an object in 
  $\coh(\mathfrak{A}\underset{\mathfrak{A}/W}{\times}\mathfrak{A})$. We define 
  $$\coh(\mcal{H}^{\on{ell}}),$$ to be the category of 
  (categorical) $\mcal{H}^{\on{ell}}$-modules in $\coh(\mathfrak{A})$.
  We observe that this notion is quivalent to the notion of coherent sheaves of 
  $\mcal{H}^{\on{ell}}$ on $\mathfrak{A}/W$. Indeed, using the 
  lax monoidal property of the pushforward $\pi_\ast$ in Lemma \ref{lem_lax_monoidal}, any objects in $\coh(\mathfrak{A})$ gives rise to an object in
  $\coh(\mathfrak{A}/W)$ via the pushforward $\pi_\ast$. Conversely,
  any $\mcal{H}^{\on{ell}}$-module has the action of the 
  subsheaf of algebras $\pi_\ast\mcal{O}_{\mathfrak{A}} \hookrightarrow \mcal{H}^{\on{ell}}$, which equips it with the structure of a coherent 
  sheaf on $\mathfrak{A}$. Indeed, the category $\coh(\mcal{H}^{\on{ell}})$
  is the appropriate representation category that we wish to utilise for the 
  ellAHA.\\\\
  Let $$\coh^{\on{fin}}(\mcal{H}^{\on{ell}}),$$
  be the full subcategory of $\coh(\mcal{H}^{\on{ell}})$ whose underlying
  coherent sheaf has zero-dimensional support in $\mathfrak{A}$. Let 
  $$\coh^{\on{flat}}(\mcal{H}^{\on{ell}}),$$
  be the full subcategory of $\coh(\mcal{H}^{\on{ell}})$ whose underlying coherent
  sheaf is a homogeneous vector bundle (i.e. locally free, coherent sheaf)
  on $\mathfrak{A}$.\\\\
  Similarly, the monoidal category $\coh(\mathfrak{A} \times \mathfrak{A}^\vee)$
  obtains the structure of a $\coh(\mathfrak{A} \underset{\mathfrak{A}/W}{\times} \mathfrak{A} \times \mathfrak{A}^\vee \underset{\mathfrak{A}^\vee/W^\dyn}{\times} \mathfrak{A}^\vee)$-module 
  via convolution. 
  Let us define a projection map 
  $$p_{ij} : (\mathfrak{A} \underset{\mathfrak{A}/W}{\times} \mathfrak{A}) \times (\mathfrak{A}^\vee \underset{\mathfrak{A}^\vee/W^\dyn}{\times} \mathfrak{A}^\vee) \longrightarrow \mathfrak{A} \times \mathfrak{A}^\vee, \quad (u_1,u_2,z_1,z_2) \longmapsto (u_i,z_j),$$
  and 
  \begin{align*}
      \Delta_{ij} : (\mathfrak{A} \underset{\mathfrak{A}/W}{\times} \mathfrak{A}) \times (\mathfrak{A}^\vee \underset{\mathfrak{A}^\vee/W^\dyn}{\times} \mathfrak{A}^\vee) &\longrightarrow (\mathfrak{A} \underset{\mathfrak{A}/W}{\times} \mathfrak{A}) \times (\mathfrak{A}^\vee \underset{\mathfrak{A}^\vee/W^\dyn}{\times} \mathfrak{A}^\vee) \times (\mathfrak{A} \times \mathfrak{A}^\vee)\\
      (u_1,u_2,z_1,z_2) &\longmapsto (u_1,u_2,z_1,z_2,u_i,z_j).
  \end{align*}
  Then, the convolution action is given by:
  $$\mcal{G} \star \mcal{F} := (p_{11})_\ast \Delta_{22}^\ast (\mcal{G} \boxtimes \mcal{F}).$$
  Since $\mcal{H}^\dyn$ is an object in $\coh(\mathfrak{A} \underset{\mathfrak{A}/W}{\times} \mathfrak{A} \times \mathfrak{A}^\vee \underset{\mathfrak{A}^\vee/W^\dyn}{\times} \mathfrak{A}^\vee)$, let us define $$\coh(\mcal{H}^\dyn),$$
  to be the category of $\mcal{H}^\dyn$-modules in $\coh(\mathfrak{A} \times\mathfrak{A}^\vee)$. 
  
  \chapter{The Dynamical Elliptic Affine Hecke Algebra}\label{app_D}
  
  In this section, we follow the construction outlined in \cite[\S 4.3]{ZZ21}, \cite[\S 1, \S 3]{ZZ24},
  and \cite{LZZ23}.
  The idea is that instead of defining a Hecke algebra on an elliptic curve 
  $E$, we wish to take the product $E \times E^\vee$ instead. Here, 
  $E^\vee$ denotes the dual abelian variety, and is define to be the Picard
  group of degree zero line bundles on $E$ --- denoted $E^\vee := \on{Pic}^0(E)$ 
  \cite[\S 9.3]{pol09}. \\\\
  A key benefit of the dynamical ellAHA is that one can define elliptic analogues
  of Demazure-Lusztig operators that satisfy the usual braid relations (see 
  \cite[(16) and Proposition 4.11]{ZZ21}). This is not a feature that is 
  available in the usual ellAHA.\\\\
  In this case, $E$ is self-dual --- that is,
  $E \cong \on{Pic}^0(E)$ \cite[\S 9.4]{pol09}. Then, the dual abelian variety of 
  $\mathfrak{A}$ is given by 
  $$\mathfrak{A}^\vee := \on{Pic}^0(\mathfrak{A}) \cong E \otimes \mathbb{X}^\ast(T),$$
  and similarly we have that $\mathfrak{A} \cong \mathfrak{A}^\vee$ by the 
  self-duality of $E$. We use $u$ to denote elements in $\mathfrak{A}$, and 
  $z$ to denote elements in $\mathfrak{A}^\vee$. The degree zero line bundle
  determined by the divisor $z \in \mathfrak{A}^\vee$ is denoted by 
  $\mcal{O}(z)$. On the product $\mathfrak{A} \times \mathfrak{A}^\vee$, there exists an universal line bundle $\mcal{P}$ called the 
  \emph{Poincar\'e line bundle} \cite[\S 9.4]{pol09}, which is characterised by 
  the property that for any $z \in \mathfrak{A}^\vee$, the restriction sheaves
  have the property:
  $$\mcal{P}\vert_{\mathfrak{A} \times \lbrace z \rbrace} \cong \mcal{O}(z),\quad \mcal{P}\vert_{\lbrace u \rbrace \times \mathfrak{A}^\vee} \cong \mcal{O}(u).$$
  Consider the origin $0 \in E$. Then, the line bundle $\mcal{O}(0)$ is lifted to the cover 
  $\C^\times$ of $E$, it becomes trivial. Given any line bundle $\mcal{L}$ over $E$, denote the 
  lift by $\mcal{L} \to \C^\times$, and let $\widetilde{s} : \C^\times \to \mcal{L}$ be a section 
  of the line bundle. Then, the local trivialisation 
  $$\begin{tikzcd}
      \mathcal{L} \arrow[rr, "\varphi"] \arrow[rd, bend left] &                                                          & \mathbb{C}^\times \times \mathbb{C} \arrow[ld] \\
                                                              & \mathbb{C}^\times \arrow[lu, "\widetilde{s}", bend left] &                                               
  \end{tikzcd}$$
  is uniquely determined by two conditions:
  \begin{itemize}
      \item[(i)] the isomorphism $\varphi$ must commute with multiplication by $q^{\mathbb{Z}}$,
      \item[(ii)] the derivative of $\varphi \circ \widetilde{s}$ should be equal to $1$ at $1 \in \C^\times$ 
      \cite[pg. 38]{sie80}.
  \end{itemize}
  With this in mind, the $q$-Jacobi theta function is given by:
  $$\Theta_q(u) = (q^{-1};q^{-1})_\infty (-q^{-1}u;q^{-1})\infty (-u^{-1};q^{-1})\infty,\quad u \in \C^\times,\quad \vert q \vert < 1,$$
  defines a holomorphic function on a double cover of $\C^\times$. Let us denote the coordinates of 
  $\mathfrak{A}$ and $\mathfrak{A}^\vee$ as $u$ and $z$, respectively. Then, the ratio of Jacobi theta functions:
  $$\frac{\Theta_q(z^{-1}u))}{\Theta_q(u)},$$ defines a rational section of $\mcal{P}$.\\\\
  The Weyl group actions on $\mathbb{X}_\ast(T)$ and $\mathbb{X}^\ast(T)$
  induce actions on $\mathfrak{A}$ and $\mathfrak{A}^\vee$, respectively.
  To distinguish between the Weyl group actions, let us write $W$ for the 
  Weyl group acting on $\mathfrak{A}$, and $W^{\on{dyn}}$ for the Weyl group
  acting on $\mathfrak{A}^\vee$, called the \emph{dynamic Weyl group}. 
  Let us write $w^{-1} : \mathfrak{A} \to \mathfrak{A}$. Then, 
  for any $z\in \mathfrak{A}^\vee$, 
  $$\mcal{O}(w^\dyn \cdot z) = (w^{-1})^\ast \mcal{O}(z),$$
  which is to say that the Poincar\'e line bundle $\mcal{P}$ is preserved by the
  diagonal action of $W\times W^\dyn$. Moreover, $\mcal{P}$ is a $W$-equivariant 
  line bundle.\\\\
  Recall from our discussion of the ellAHA that there is a map 
  $$\chi_\alpha : \mathfrak{A} \longrightarrow E,\quad u \otimes \mu^\vee \longmapsto u^{\langle \mu^\vee, \alpha\rangle}.$$
  Dually, we may define a map 
  $$\chi_{\alpha^\vee} : \mathfrak{A}^\vee \longrightarrow E,\quad z \otimes \mu \longmapsto z^{\langle \alpha^\vee, \mu\rangle}.$$
  Let $T_\alpha = \ker \chi_\alpha$, and $T_{\alpha^\vee} := \ker \chi_{\alpha^\vee}$. As before, 
  $T_{\alpha,\hbar} = \ker (\chi_\alpha - \hbar)$, and $T_{\alpha^\vee, z} = \ker (\chi_{\alpha^\vee} - z)$.
  
  \section{The Dynamical Elliptic Twisted Group Algebra}
  In this section, we construct a dynamic version of the sheaf of algebras
  $\mcal{O}[W]$ from the previous section. 
  Let $\coh_{W\times W^\dyn}(\mathfrak{A} \times \mathfrak{A}^\vee)$ to be the 
  category of $W\times W^\dyn$-equivariant coherent sheaves on 
  $\mathfrak{A} \times \mathfrak{A}^\vee$.
  Objects in this category are coherent sheaves with a direct sum decomposition
  $$\mcal{F} = \bigoplus_{\substack{w\in W\\ v^\dyn \in W^\dyn}} \mcal{F}_{w,v^\dyn},$$ where each component $\mcal{F}_{w,v^\dyn}$ is a coherent sheaf 
  called the degree $(w,v^\dyn)$ component of $\mcal{F}$. Morphisms in this 
  category are such that the grading is respected. One may equip this 
  category with a monoidal structure by defining:
  $$\mcal{F}_{w_1,v_1^\dyn} \star \mcal{G}_{w_2, v_2^{\dyn}} := \mcal{F}_{w_1,v_1^\dyn} \otimes (w_1^{-1})^\ast ((v_1^\dyn)^{-1})^\ast \mcal{G}_{w_2,v_2^\dyn},$$
  which by construction is a coherent sheaf of degree $(w_1w_2, v_1^\dyn v_2^\dyn)$. The object $\mcal{O}_{\mathfrak{A} \times \mathfrak{A}^\vee}$ concentrated
  in degree $0$ is treated as the unit object. Recall that given two schemes
  $X$ and $Y$ over a scheme $S$, and $\mcal{F}$ an $\mcal{O}_X$-module, and $\mcal{G}$ an $\mcal{O}_Y$-module, the \emph{external tensor product} is defined to be
  the $\mcal{O}_{X\times_S Y}$-module:
  $$\mcal{F} \boxtimes_{\mcal{O}_S} \mcal{G} := \on{pr}_X^\ast (\mcal{F}) \otimes_{\mcal{O}_{X\times_SY}} \on{pr}_Y^\ast (\mcal{G}),$$
  where $\on{pr}_X$ and $\on{pr}_Y$ denote projections onto the first and second 
  factor of $X\times_SY$, respectively. \\\\
  Now, let us define a line bundle 
  $$\mcal{L} := \mcal{P} \otimes \left(\mcal{O}(\hbar\otimes \rho) \boxtimes \mcal{O}(-\hbar \otimes \rho^\vee)\right).$$
  One regards the line bundle $\mcal{L}$ as a certain $\rho$-shift of the Poincar\'e line bundle $\mcal{P}$.
  Using the (dynamic) Weyl group actions, one obtains a new line bundle: $$\mcal{O}_{w,v^\dyn} := \mcal{L} \otimes (w^{-1})^\ast ((v^\dyn)^{-1})^\ast \mcal{L}^{-1}.$$ 
  \begin{definition}
      The object $$\mcal{O}[W\times W^\dyn] := \bigoplus_{\substack{w\in W\\v^\dyn\in W^\dyn}}\mcal{O}_{w,v^\dyn},$$ is called the \emph{dynamical 
      elliptic twisted group algebra}.
  \end{definition}
  
  As the name suggests, $\mcal{O}[W\times W^\dyn]$ in $\coh_{W\times W^\dyn}(\mathfrak{A} \times \mathfrak{A}^\vee)$
  defines an algebra (a.k.a. monad object) in a suitable category of coherent sheaves. In particular, it is 
  an algebra object in the category 
  $$\coh_{W \times W^\dyn}(\mathfrak{A} \times \mathfrak{A}^\vee).$$
  The algebra structure is given by the following:
  
  \begin{lemma}[Lemma 2.2, \cite{LZZ23}]
      \leavevmode
      \begin{itemize}
          \item[(i)] We have the following composition properties:
              \begin{align*}
                  \mcal{O}_{w_1w_2, v_1^\dyn v_2^\dyn} &= \mcal{O}_{w_1,v_1^\dyn} \otimes (w_1^{-1})^\ast (v_1^{-1})^{\dyn\ast} \mcal{O}_{w_2,v_2^\dyn},\\
                  \mcal{O}_{w,v^\dyn}^{-1} &= (w^{-1})^\ast (v^{-1})^{\dyn\ast}\mcal{O}_{w^{-1},(v^{-1})^\dyn}.
              \end{align*}
          \item[(ii)] The objects $\mcal{O}[W\times W^\dyn]$ 
              and $\mcal{O}[W\times W^\dyn]^{-1} := \bigoplus_{\substack{w\in W\\v^\dyn \in W^\dyn}} \mcal{O}_{w,v^\dyn}^{-1}$ are 
              monoidal objects, where the degrees of $\mcal{O}_{w,v^\dyn}$
              and $\mcal{O}_{w,v^\dyn}^{-1}$ are both $(w,v)$.
      \end{itemize}
  \end{lemma}
  
  \begin{remark}
      \cite{LZZ23} calls this the "Kostant-Kumer" twisted product, but augmented by the 
      additional dynamic Weyl group action (see \cite[\S 2.6]{LZZ23}).
  \end{remark}
  
  Similarly, we also have the monoidal categories $\coh_W(\mathfrak{A})$, and 
  $\coh_{W^\dyn}(\mathfrak{A}^\vee)$. The twisted group algebra of 
  \cite{gkv95}:
  $$\mcal{O}_{\mathfrak{A}} \rtimes \C[W] := \bigoplus_{w\in W} \mcal{O}_{\mathfrak{A}},$$
  is an algebra object of $\coh_W(\mathfrak{A})$. The dynamic version 
  $\mcal{O}_{\mathfrak{A}^\vee} \rtimes \C[W^\dyn]$ is similarly also an 
  algebra object of $\coh_{W^\dyn}(\mathfrak{A}^\vee)$.
  
  \section{Residue Construction of the Dynamical Hecke Algebra}
  
  \begin{definition}
      Let $\mcal{L}$ be a line bundle over the abelian variety $\mathfrak{A} \times \mathfrak{A}^\vee$, and
      let $f$ be a rational section of $\mcal{L}$, with a pole along $T_\alpha$ of order at most $1$.
      Then, the \emph{residue} of $f$ at $T_\alpha$ is given by:
      $$\on{Res}_\alpha(f) := \left(\Theta_q(u) f\right)\vert_{T_\alpha},$$
      where $\Theta_q$ is the Jacobi $q$-theta function. 
  \end{definition}
  This definition says that taking the residue is equivalent to evaluating $\Theta_q(u)f$ 
  at $u = 0$. More explicitly, for an open set $U \subseteq \mathfrak{A} \times \mathfrak{A}^\vee$, 
  and $f \in \mcal{L}(U)$, then $\on{Res}_\alpha(f)$ is a rational section of 
  $$\left(\mcal{L} \otimes \mcal{O}(-T_\alpha)\right)\vert_{T_\alpha},$$ 
  on $T_\alpha \cap U$. Moreover, if $f$ is a rational section of $\mcal{O}_{w,v^\dyn}$, and 
  $g$ is a rational section of $\mcal{O}_{s_\alpha w,v^\dyn}$, then 
  $$\on{Res}_\alpha(f+g) = \on{Res}_\alpha(f) + \on{Res}_\alpha(g),$$ 
  (see \cite[Lemma 2.1]{ZZ21}).
  Consider the map
  $$j\times j^\vee : \mathfrak{A}^{\on{reg}} \times (\mathfrak{A}^\vee)^{\on{reg}} \longrightarrow \mathfrak{A} \times \mathfrak{A}^\vee,$$
  and the map $$\pi \times \pi_\dyn : \mathfrak{A} \times \mathfrak{A}^\vee \longrightarrow \mathfrak{A}/W \times \mathfrak{A}^\vee/W^\dyn.$$
  Let us write rational sections of $\mcal{O}[W\times W^\dyn]$ as 
  $$\sum_{\substack{w\in W\\v^\dyn \in W^\dyn}} a_{w,v^\dyn} \delta_w \delta_v^\dyn,$$
  where $a_{w,v^\dyn}$ is a rational section of $\mcal{O}_{w,v^\dyn}$, and 
  $\delta_w$, and $\delta_v^\dyn$ denotes its degree. We can now 
  define the dynamical Hecke algebra:
  
  \begin{definition}
      Let $\mcal{H}^\dyn$ be the subsheaf of $(j\times j^\vee)_\ast(j\times j^\vee)^\ast \mcal{O}[W\times W^\dyn]$, so that on any open subset 
      $U\subseteq \mathfrak{A} \times \mathfrak{A}^\vee$, 
      sections of $\mcal{H}^\dyn(U)$ are of the form $$\sum_{\substack{w\in W\\v^\dyn \in W^\dyn}} a_{w,v^\dyn} \delta_w \delta_v^\dyn,$$
      satisfying the following conditions:
      \begin{itemize}
          \item[(i)] $a_{w,v^\dyn}$ only has poles at $T_\alpha$ or $T_{\alpha^\vee}$ of at most order $1$, for finitely many $\alpha \in \Phi^+$,
          \item[(ii)] $\on{Res}_\alpha(a_{w,v^\dyn} + a_{s_\alpha w,v^\dyn}) = 0$,
              and $\on{Res}_{\alpha^\vee}(a_{w,v^\dyn} + a_{w,s_\alpha v^\dyn}) = 0$,
          \item[(iii)] For any $\alpha \in \Phi(w) = \Phi^+ \cap w^{-1}\Phi^-$, as a rational section
              of $\mcal{O}_{w,v} \otimes \widetilde{\mcal{O}}(-\hbar \otimes \alpha)$, the function 
              $$a_{w,v} \frac{\Theta_q(z_\alpha)}{\Theta_q(\hbar z_\alpha^{-1})},$$
              is regular at $T_{\alpha,\hbar}$.
      \end{itemize}
  \end{definition}
  
  \section{The Module Category $\coh(\mcal{H}^\dyn)$}
  \begin{definition}
      Let $\pi^\vee : \mathfrak{A}^\vee \to \mathfrak{A}^\vee/W^\dyn$ be the dual natural map.
      A \emph{$\mcal{H}^\dyn$-module} $\mcal{F}$ is an object in 
      $\coh(\mathfrak{A}/W \times_{\spec \C} \mathfrak{A}^\vee/W^\dyn)$ 
      for which there exists a multiplication map:
      $$(\pi \times \pi^\vee)_\ast\mcal{H}^\dyn \otimes_{\mcal{O}_{\mathfrak{A}/W \times \mathfrak{A}^\vee/W^\dyn}} \mcal{F} \longrightarrow \mcal{F},$$ 
      such that for each $V \subseteq \mathfrak{A}/W \times_{\spec \C} \mathfrak{A}^\vee/W^\dyn$,
      each regular section $\mcal{F}(U)$ has the structure of a $(\pi \times \pi^\vee)_\ast\mcal{H}^\dyn(U)$-module.
  \end{definition}
  Let $\coh(\mcal{H}^\dyn)$ denote the category of $\mcal{H}^\dyn$-modules. Define 
  the full subcategories:
  $$\coh^{\on{flat}}(\mcal{H}^\dyn), \quad \coh^{\on{fin}}(\mcal{H}^\dyn),$$
  analogously.

\end{document}